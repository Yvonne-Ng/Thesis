\chapter{Concluding Words and Future Direction for Resonance hunting}
% add projected exclusion plot 


\epigraph{\textit{The bud disappears when the blossom breaks through, and we might say that the former is refuted by the latter; in the same way when the fruit comes, the blossom may be explained to be a false form of the plant’s existence, for the fruit appears as its true nature in place of the blossom. The ceaseless activity of their own inherent nature makes these stages moments of an organic unity, where they not merely do not contradict one another, but where one is as necessary as the other; and constitutes thereby the life of the whole.}}{--Georg Hegel}

Particle physics has come a long way since its early days. The Standard Model advanced fundamental physics and human understanding of the constituent of the matter world. Along with the more completed model, along comes more questions. In particular, the enigma regarding the nature of Dark Matter continue to challenge our understanding to the . The analyses presented in Chapter~\ref{} and Chapter~\ref{} both extends beyond the previous ground and explored new tools for future resonance hunting effort. In this chapter, future direction in the area of resonance hunting for new physics will be discussed.


%Figure here shows the dark matter exclusion limit plots from the dark matter chapter.


\section{Gaussian Process as a Background Modelling Method}
The Gaussian Process based background estimation method is currently being finalized in ATLAS with the dimuon analysis in Chapter~\ref{chapter:dimuon}. It provides an simpler method for smooth background modelling for other resonance hunting analyses where MC is limited. Possible searches future include channels not searched for covered in Chapter~\ref{chapter:topo}.Gaussian Process can also be extended to perform searched in the 2-dimensional search space for increased signal sensitivity.

\section{Data Scouting}
Due to limited band width, trigger described in Chapter~\ref{} set a lower bound in the mass of the resonance that could be searched for. Searching for a signal with an initial state radiation mitigate the problem but also lower the sensitivity to the signal. Data scouting is a method proposed to create special triggering stream and object to mitigate the limit triggering band width by storing only partial events. Only information directly related to the analysis will be stored. Currently the
dimuon data scouting analysis is under study in ATLAS and will be a future area with promising improved sensitivity. 

\section{Anomaly Detection with Machine Learning Method}
The concept of “model independent” searches is not unfamiliar to LHC. Searches for new particles often include so- called model independent results in the form of excesses beyond certain statistical significance LHC also has its own dedicated general search [6]. However, these current approaches are not truly model independent: they are either signal model dependent in the sensitivity optimal kinematic cut for targeted search or LHC background dependent in the general search. Search sensitivity
is greatly diminished if the anomaly is not as predicted by the signal or background models. Improved methods include Classification Without Label (CWoLA), all utilizing different methods of 

An improved method will instead teach data to perform optimal selection on its own based only on the anomaly observed in data. One recent proposal is the weakly supervised Classification Without Labels (CWoLa) hunting method [7]. This technique discards the usual supervised Signal-over-background (S/B) kinematic strategy, where the optimal selection is made based on maximal S/B ratio of a specific model combination. In utilizing CWoLa for resonance hunting, the bump-like signal produces a signal-rich center-band and signal-deprived side-bands optimal for optimal sensitivity kinematic cuts training. If a new particle is embedded in the dataset, CWoLa training will produce a selection cut that maximizes events in the central-band-bump without any underlying S/B model assumptions. The data is made to reveal surprising anomalies on its own from the data alone.
The existing two-quark final state resonance CWoLa search [8] to other final states including muons, photons, and other uncovered resonance signatures. While the dijet analysis has proven the feasibility of the method in ATLAS, many detailed technical aspects, further generalizations, and expansions into larger feature spaces must still be performed. The new analyses on muon and photon final states will go beyond the predecessor by focusing training on additional features including the jet substructure of the ISR jet. More complicated signal topologies, where the primary resonant decay object can be composite, can also be explored using simpler muon/photon final state objects. The CWoLa model independent searches will serve as complementary additions to current physics analyses’ model dependent and independent search methods, and it will push sensitivity into new regions with attuned understanding driven by the data itself.


The 
