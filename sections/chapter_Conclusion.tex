\chapter{Concluding Words: Anomalous Resonance as a Question}

%So tell me, what is it that I do not know?
%Anomalous resonances as a question to what can be looked for in physics
%A question in disguise
% add projected exclusion plot 

%I had a dream where I swallowed a cheerio shaped hole. 
%In a dream I swallowed a cheerio shaped hole.
%The question in the form of a bump.%
%This is not a search but rather a question in disguise. 
% Search claims thhat we actually do know someonthing about the bump.
% queer in search of joussance 

%\epigraph{\textit{The bud disappears when the blossom breaks through, and we might say that the former is refuted by the latter; in the same way when the fruit comes, the blossom may be explained to be a false form of the plant’s existence, for the fruit appears as its true nature in place of the blossom. The ceaseless activity of their own inherent nature makes these stages moments of an organic unity, where they not merely do not contradict one another, but where one is as necessary as the other; and constitutes thereby the life of the whole.}}{--Georg Hegel}

Particle physics has come a long way since its early days. Since the discovery of the Higgs Boson in 2012, the Standard Model is considered completed. But human's quest to understand fundamental physics is far from over. Many existing issues in Standard Model points to a picture out there unknown, yet to be understood. This evidence, especially those concerning Dark Matter, drove the resonance search analyses presented in Chapter~\ref{chapter:dijetISR} and Chapter~\ref{chapter:dimuon}.
However, as a highly distinguishable signature, resonances hunting offers much more than a tool to hunt for a specific candidate. In the age where most exotic and SUSY searches returned empty results, what ought to concern physicists regarding data is no longer merely what can be searched for, but also what can be \textit{asked}: what questions can be asked of data for them to reveal to us what we do not know? This chapter summarizes some directions that can be taken in the resonance hunting regime for future studies.
%But the search for the little bump is perhaps not just a search
%for dark matter, it is also a question on what is possible to be searched for and what lies between the boundaries of the known and unknown of our understanding. What is it that we do not know about? 


%Perhaps the gauge of what a physicist
%
%But human's search for truth is far from that.  However, along with all the compelling evidence, a complete physics theory is out there to be discovered. The analyses presented in Chapter~\ref{chapter:dijet} and Chapter~\ref{chapter:dimuon} both offered results important to answering of the question of Dark Matter. But in the age where new physics
%theory seems to be out of reach with most evidence, the quest of resonance hunting, is perhaps not merely to search for a theoretical candidate to what's out there. Is not in actuality a search, but rather, is a question in disguise. 
%What is possible for us to look for? The little bumps like signature is what lies between the known and the unknown of our understanding? Future lies in good miner and searcher for new things, it lies in physicists being good question askers in coaxing data from telling us its stories. 
%As a highly distinguishable signature, resonances not just a tool for catch, but also a question we can ask on what borders between the known and unknown. 
%
%the more compelling question that these analyses is no longer a particular theory candidate, but instead, to ask what is possible to be looked for?
%
%Instead the search for resonances, in their various methods, can be thought of as asking 
%
%the search for a more complete understanding to the nature of the world is far from over. One way to further our search is through these ripple nuggets of resonances. Driven by the question of Dark Matter, the analyses presented in Chapter~\ref{chapter:dijet} and Chapter~\ref{chapter:dimuon} bothoffers additional evidence to the search of the
%Dark Matter mediator Z'. The searches are also important in the following way, namely they extends beyond the previous ground and explored new tools for future resonance hunting effort. perhaps it's offering an alternative not just in the search but also in the method itself. Instead of 
%what is it possible to be looked for? 


%Figure here shows the dark matter exclusion limit plots from the dark matter chapter.

\section{Unexplored Landscape of the Two Body Resonances}
Chapter~\ref{chapter:topo} explored uncovered two-body final states in resonance hunting in detail. The paper has since been superseded by newer results ~\cite{2020}. The surveys offer an overview of all two-body final states that are left unsearched for in collider physics that could provide a wealth of sources for possible places to look for new physics signatures.

\section{Gaussian Process as a Background Modeling Method}
The Gaussian Process-based background estimation method is currently being finalized in ATLAS with the dimuon analysis in Chapter~\ref{chapter:dimuon}. It provides a more flexible method better suited for high luminosity for smooth background modeling for other resonance hunting analyses where MC is limited. The method offers the advantage of being \textit{generalizable} for any final state with a smooth background estimation in the signal region. The method will be applicable to many other future analyses. 
%Gaussian Process can also be extended to perform searched in the 2-dimensional search space for increased signal sensitivity.

\section{Data Scouting}
Due to limited bandwidth, the trigger described in Chapter~\ref{chapter:dimuon} set a lower bound in the mass of the resonance that could be searched for. Searching for a signal with initial state radiation mitigates this but it also results in a lowered sensitivity to the signal. Data scouting is a method proposed to create a special triggering stream and object to mitigate the limit triggering bandwidth by storing only partial events. Only information directly related to the analysis will be
stored. Currently, the dimuon data scouting analysis is under study in ATLAS and will be a future area with promising improved sensitivity. 

\section{Anomaly Detection for Resonances with Machine Learning Method}
The concept of “model-independent” searches is not unfamiliar to LHC. Searches for new particles often include model-independent results in the form of excesses beyond certain statistical significance LHC also has its dedicated general search ~\cite{General2019}. However, these current approaches are not truly model-independent: they are either signal model-dependent in the sensitivity optimal kinematic cut for targeted search or LHC background dependent in the general search. Search
sensitivity is greatly diminished if the anomaly is not as predicted by the signal or background models. An improved method will instead teach data to perform optimal selection on its own based only on the anomaly observed in data. One recent proposal is the weakly supervised Classification Without Labels (CWoLa) hunting method~\cite{CWoLA2019}. This technique discards the usual supervised Signal-over-background (S/B) kinematic strategy, where the optimal selection is made based on
the maximal S/B ratio of a specific model combination. In utilizing CWoLa for resonance hunting, the bump-like signal produces a signal-rich center-band and signal-deprived side-bands optimal for optimal sensitivity kinematic cuts training. If a new particle is embedded in the dataset, CWoLa training will produce a selection cut that maximizes events in the central-band-bump without any underlying S/B model assumptions. The data is made to reveal surprising anomalies on its own from the data alone.
The existing two-quark final state resonance CWoLa search~\cite{CWOLA2020} to other final states including muons, photons, and other uncovered resonance signatures. While the dijet analysis has proven the feasibility of the method in ATLAS, many detailed technical aspects, further generalizations, and expansions into larger feature spaces must still be performed. The new analyses on muon and photon final states will go beyond the predecessor by focusing the training on additional features including
the jet substructure of the ISR jet. More complicated signal topologies, where the primary resonant decay object can be composite, can also be explored using simpler muon/photon final state objects. The CWoLa model-independent searches will serve as complementary additions to current physics analyses’ model-dependent and independent search methods, and it will push sensitivity into new regions with attuned understanding driven by the data itself. 

%otheer anomaly searches are driven by-%rthe. It could open a new chapter where data are made to tell the truth behind. 
