%\chapter{Common Elements in the Analysis of High Energy Physics Collision Data}
\chapter{Common Elements in the Search for New Physics}
\label{chap:common_search}

\epigraph{
\textit{Our `Age of Anxiety' is, in great part, the result of trying to do today's jobs with yesterday's tools!}
}{--Marshall McLuhan}


%%%%%%%%%%%%%%%%%%%%%%%%%%%%%%%%%%%%%%%%%%%%%%%%%%%%%%%%%%%%%%%%%%%%%%
%%%%%%%%%%%%%%%%%%%%%%%%%%%%%%%%%%%%%%%%%%%%%%%%%%%%%%%%%%%%%%%%%%%%%%

There are many ways in which physicists study and analyse the data collected by
the ATLAS detector.
Once the gathered data has been aggregated and the physics objects therein
have been reconstructed (Chapter~\ref{chap:objects}), an analyst has at their fingertips access
to the stuff of high energy particle physics from which they
can test their hypotheses or perform measurements of fundamental quantities.
In the present work, several analyses representing the search for BSM physics
will be discussed in detail.
The details of each of these analyses differ quite a bit, but each follow
a general \textit{analysis strategy} that will be discussed
further in Section~\ref{sec:gen_strategy}.
Physics analyses start with a stated goal in mind; for example,
a measurement of some quantity to a desired level of precision
or, as in the analyses to be presented, a statement about the existence of a new particle or theory of new physics.
The methods by which these statements are to be made are intertwined with, and generally
dictate the initial design of, the overall
analysis strategy and are statistical in nature.
In Section~\ref{sec:stat_hypo} a discussion of the statistical methods used
in the analyses to be presented will be given, highlighting how statements about
new physics are made and how broad-strokes physics conclusions can be drawn from them.
Section~\ref{sec:fakes} follows up on the topic of how one estimates particular sources
of SM backgrounds for which the MC simulation cannot be relied upon to give reasonable
predictions, particularly on those processes that lead to leptons being produced in the detector
that do not originate from the $pp$ hard-scatter processes of interest.

%%%%%%%%%%%%%%%%%%%%%%%%%%%%%%%%%%%%%%%%%%%%%%%%%%%%%%%%%%%%%%%%%%%%%%
%%%%%%%%%%%%%%%%%%%%%%%%%%%%%%%%%%%%%%%%%%%%%%%%%%%%%%%%%%%%%%%%%%%%%%
%
% GENERAL STRATEGY
%
%%%%%%%%%%%%%%%%%%%%%%%%%%%%%%%%%%%%%%%%%%%%%%%%%%%%%%%%%%%%%%%%%%%%%%
%%%%%%%%%%%%%%%%%%%%%%%%%%%%%%%%%%%%%%%%%%%%%%%%%%%%%%%%%%%%%%%%%%%%%%
\section{General Analysis Strategy}
\label{sec:gen_strategy}

In this section the analysis strategy used in the searches for new physics
to be presented in Chapters~\ref{chap:search_stop} and \ref{chap:search_hh}
will be given.
The general analysis workflow for designing an analysis is outlined in the following
sub-sections.

%%%%%%%%%%%%%%%%%%%%%%%%%%%%%%%%%%%%%%%%%%%%%%%%%%%%%%%%%%%%%%%%%%%%%%%%%%%%%
% SIGNAL PHENO
%%%%%%%%%%%%%%%%%%%%%%%%%%%%%%%%%%%%%%%%%%%%%%%%%%%%%%%%%%%%%%%%%%%%%%%%%%%%%
\subsection{Target the Signal}
\label{sec:sig_pheno}

The search for a particular source of new physics, such as a particular model of SUSY (Chapter~\ref{chap:bsm}),
begins first with the thorough understanding of the signatures that the new physics model
will leave in the ATLAS detector.
This generally requires a strict definition of the \textit{final state} of the
new physics model that one wishes to look for; for example, deciding to search for
evidence of SUSY via the production of the SUSY partners to the SM top-quark
in final states having exactly two leptons (electrons or muons) instead of
exactly zero or exactly one lepton, as in Chapter~\ref{chap:search_stop}.\footnote{Performing
searches for new physics by the partitioning of specific new physics models
by their resulting final states allows for separate, independent dedicated analyses to be carried out
for each possible final state with the idea that each one will be more sensitive
to the presence of the new physics in their respective final state than would be
a single analysis attempting to target all possible final states of the new physics production.
The results of the independent analyses' searches can be statistically combined once they are finished,
leading to enhanced sensitivities to the new physics scenario in question that is more or less independent
of the final state.}
Once a new physics model has been chosen, along with its final state, there is a well-defined
\textit{signal} to be looked for in the data recorded by the ATLAS detector.
The production and decay of the sought-for signal is then simulated via MC methods in the exact
same manner as for the SM processes, as described in Chapter~\ref{chap:simulation}.
In physics analyses, the physics processes not inclusive of the sought-for signal processes
are referred to as the \textit{background} processes.

The simulation of the signal process allows one to study the kinematics of the signal in detail, in order
to get an overall feel for what phase space the signal inhabits.
Knowledge of both the signal final state and its kinematics therein informs the analyst
about the specific SM background processes that are likely to be relevant to the analysis.
For example, if the sought-for signal decays to two leptons with opposite electric charge
that are of the same flavor (both leptons are electrons or both are muons, for example)
it is very likely that the SM processes inclusive of $Z$-boson production will be relevant,
since this is one of the main $Z$-boson decay final states, as opposed to the production of a single $W$-boson
whose decay does not lead directly to final states with two leptons.
Knowledge of the dominant SM background processes, then, allows
one to determine how the phenomenology and kinematics of the signal differ
with respect to those of the relevant backgrounds by comparing the simulated events
of each.
The aim of this is to be able to define a basis of kinematic observables that allows
for the discrimination between the signal and background.
From such a basis of observables, one can define regions of phase space in which
the signal-to-background ratio is large, such that the likelihood of observing
the presence of the signal is (ideally) maximal.
Such regions of increased signal purity\footnote{The `purity' of a process is defined
as the fraction of a given process in a region of phase space, relative to the sum
of all processes (inclusive of the process in question).} are referred
to as \textit{signal regions} (SR).
As an example, take the case where there is a single discriminating variable in our
basis of useful kinematic observables.
One would apply a selection on this observable in such a way that $pp$ collision events
satisfying this selection are likely to be enhanced in signal events.
This is one-dimensional SR case is illustrated in Figure~\ref{fig:sr_search_v}.

\begin{figure}[!htb]
    \begin{center}
        \includegraphics[width=0.65\textwidth]{figures/common_ana/sr_search_vPDF}
        \caption{
            Signal region concept illustrated in the case of a one-dimensional selection
            made on a discriminating kinematic observable.
            The dominant SM background (red) is characterised by typically small values
            of the discriminating variable whereas the signal (blue) has values that extend
            beyond that of the background.
            The signal region in this case is defined by requiring $pp$ collision events
            to have values of the discriminating variable that are larger than
            the value indicated by the dashed vertical line, where the signal purity is
            enhanced.
            The $y$-axis represents the probability distribution of the background and signal
            processes, not their absolute yield for a given range of the quantity on the $x$-axis.
        }
        \label{fig:sr_search_v}
    \end{center}
\end{figure}



%%%%%%%%%%%%%%%%%%%%%%%%%%%%%%%%%%%%%%%%%%%%%%%%%%%%%%%%%%%%%%%%%%%%%%%%%%%%%
% GATHER THE DATA
%%%%%%%%%%%%%%%%%%%%%%%%%%%%%%%%%%%%%%%%%%%%%%%%%%%%%%%%%%%%%%%%%%%%%%%%%%%%%
\FloatBarrier
\subsection{Define the Trigger Strategy}
\label{sec:gather_data}

Once an analysis understands the final state and its kinematics that it will be searching for,
the strategy for gathering the $pp$ collision data consistent with that final state must be defined.
This requires that the analysis make a choice about which triggers to use for recording
the collision events in ATLAS, a process referred to as defining the analysis' \textit{trigger strategy}.
The analyses to be presented in Chapters~\ref{chap:search_stop} and \ref{chap:search_hh} are based
on searches for signals with leptons (electrons or muons) in the final state.
As a result, the triggers these analyses use are all based on signatures of high-\pT~leptons.

The lepton triggers used in the analyses to be presented trigger are based on the \pT~of the leptons.
They are configured to have a given \pT~threshold and if a lepton is identified in the online trigger
system during the $pp$ collisions to have a \pT~value at or above this threshold, the trigger `fires' and the
event is recorded (see Section~\ref{sec:tdaq}).
Lepton triggers are characterised by a sharp `turn-on', meaning that they become fully efficient
very near the online threshold at which they operate. 
Here, `efficiency' is defined as the ability for the trigger to make a decision to record the event
when there is actually an object satisfying its requirements. A trigger is 100\% efficiency if it
has a threshold of 10\,GeV and it fires for every single lepton with a \pT~at or above 10\,GeV.
The maximum attainable trigger efficiency is never exactly 100\% due to non-100\% coverage of the
trigger system, however.
Examples of the trigger efficiencies, measured in data and in MC, are shown in Figure~\ref{fig:trig_example} which shows
the trigger efficiency turn-on curves for representative electron and muon triggers.
It can be seen, for example, that the electron trigger efficiency reaches a plateau very near
its configured threshold of 28\,GeV.
As described in Section~\ref{sec:tdaq}, the trigger system has two-levels: Level 1 (L1) and HLT.
The efficiencies for the trigger at L1 and HLT differ, primarily due to the lower lepton momentum resolution
achieved at L1.
This can be seen in the right side of Figure~\ref{fig:trig_example}, which shows the trigger
efficiency turn-on also for the L1 trigger that seeds the HLT trigger.
The rise in efficiency, with respect to the offline reconstructed object's \pT, is shallower
at L1 than at the HLT as a result of the poorer muon momentum resolution at L1.

As can be seen in the left side of Figure~\ref{fig:trig_example}, the MC description of the trigger
response is fairly good.
However, analyses typically require that their offline reconstructed objects on which they base
their trigger strategy have \pT~values that lie on the trigger efficiency plateau.
This is because the trigger efficiency plateau represents a region in which the trigger response
is stable (i.e. unchanging) in both data and MC and can therefore be easily calibrated in the offline analysis
with scale-factors that account for differences in the measured trigger efficiencies in data and in MC simulation.
The description in the MC of the region at the trigger threshold, which has non-vertical rise,
is potentially difficult to model accurately and therefore the calibrations are not typically
derived for these regions in which the efficiency is not at its plateau.
This is illustrated in Figure~\ref{fig:trig_plateau_cartoon}, showing the typical case of a stable
and flat response in both the observed data and MC in the region of the trigger efficiency
plateau but a potentially difficult to characterise response during the turn-on phase of the trigger
efficiency.
Given, however, the relatively good momentum resolution for leptons at L1, the offline \pT~requirements on leptons
can be kept very near the online \pT~thresholds, typically within 1\,GeV or so.
Triggers based on jets or on the missing transverse momentum, however, typically require offline \pT~requirements
that are significantly higher than the online thresholds due to the fact that the online reconstruction of these
objects in the trigger is not at the level of precision attainable in the offline reconstruction.
This is illustrated by the trigger efficiency curves shown in Figure~\ref{fig:met_trig_example}, showing
the efficiency curves for triggers based on the reconstruction of the missing transverse momentum.
These \met-based triggers all have online thresholds near $80\,\GeV$, but they do not reach their
efficiency plateau until offline \met values nearing 250\,GeV.

\begin{figure}[!htb]
    \begin{center}
        \includegraphics[width=0.48\textwidth]{figures/common_ana/trig/egam_trig_example} 
        \includegraphics[width=0.48\textwidth]{figures/common_ana/trig/muon_trig_example} 
        \caption{
            Figures showing the measured trigger efficiency as a function of the associated
            offline object for a representative electron trigger (\textbf{\textit{left}}, from Ref.~\cite{EgammaTrig})
            and muon trigger (\textbf{\textit{right}}, from Ref.~\cite{MuonTrig}).
        }
        \label{fig:trig_example}
    \end{center}
\end{figure}


\begin{figure}[!htb]
    \begin{center}
        \includegraphics[width=0.75\textwidth]{figures/common_ana/trig_plateauPDF}
        \caption{
            Cartoon illustrating the principle of a trigger efficiency `turn on' curve.
            The efficiency for the lepton to fire the trigger, as a function of the \pT~of the offline object, is plotted
            as a function of the offline object's \pT.
            At the online level, given the typically poorer lepton momentum resolution, there is generally
            not a perfectly sharp (i.e. vertical) turn on at the \pT~threshold of the trigger.
            Instead there is an `S'-curve, with the efficiency increasing with a steep slope
            until it reaches a point where it flattens out.
            This latter point is referred to as the trigger efficiency `plateau'.
            Offline analyses typically apply offline \pT~requirements on their objects
            such that they are always on the plateau of the associated trigger used for event selection.
        }
        \label{fig:trig_plateau_cartoon}
    \end{center}
\end{figure}

\begin{figure}[!htb]
    \begin{center}
        \includegraphics[width=0.6\textwidth]{figures/common_ana/trig/met_trig_example}
        \caption{
            Trigger efficiency turn-on curve for typical triggers based on the missing transverse momentum.
            Figure taken from \href{https://twiki.cern.ch/twiki/bin/view/AtlasPublic/MissingEtTriggerPublicResults}{AtlasPublic/MissingEtTriggerPublicResults}.
        }
        \label{fig:met_trig_example}
    \end{center}
\end{figure}

%%%%%%%%%%%%%%%%%%%%%%%%%%%%%%%%%%%%%%%%%%%%%%%%%%%%%%%%%%%%%%%%%%%%%%%%%%%%%
% THE CONTROL REGION METHOD
%%%%%%%%%%%%%%%%%%%%%%%%%%%%%%%%%%%%%%%%%%%%%%%%%%%%%%%%%%%%%%%%%%%%%%%%%%%%%
\FloatBarrier
\subsection{Background Estimation and the Control Region Method}
\label{sec:control_region_method}

The general principle behind searches for new physics is to define a SR, or a set of SRs,
and then make predictions about how the signal and background behave therein.
Such predictions can then be compared to the data actually recorded by the ATLAS detector
and the statistical procedures described in Section~\ref{sec:stat_hypo} can be used to
make statements about whether or not --- or to what degree --- the data is likely to contain the specified signal.
The emphasis, then, in physics analyses is on the understanding and precise estimation of the backgrounds.
Without being able to properly estimate the contribution of the background processes to the
events observed in the SRs, well-defined predictions cannot therein be made, resulting in ineffective
analyses.

The process of estimating the backgrounds in an analysis' SRs is aptly referred to as
\textit{background estimation}.
There are many background estimation methods that are used.
There exist general background estimation techniques, applicable to a wide range of SM processes,
as well as more dedicated estimation techniques that are specific to a smaller subset of
SM processes.
Most rely on the MC simulation of the SM processes, either as the primary source of providing
the prediction of a given SM process in an analysis' SR(s) or secondarily, as a means of providing a
cross-check on or input to a prediction obtained using the observed data obtained in auxiliary measurements as the primary source.
The high levels of accuracy imposed upon the ATLAS MC simulation infrastructure is derived
from the large and dominant role that the MC simulation plays in the background estimation
procedures in almost all analyses performed by the ATLAS experiment.

The same background estimation strategy is used in each of the analyses to be presented in the
present thesis and is as follows.
Once the SRs designed to capture the sought-for signal are defined, MC simulation can be
used to determine the overall contribution of all SM background processes.
In such a way, the MC simulation can be used to understand which SM processes in the SR(s)
are dominant and which are sub-dominant.
The former are those whose relative contribution to the total background prediction
are large and the latter are those for which this quantity is small or negligible.

As illustrated in Figure~\ref{fig:sr_search_v} for the one-dimensional example,
SR(s) are typically defined by events populating the \textit{tails} of discriminating
observables or probe extreme regions of phase space.
%The tails of such observables are regions of low cross-section and
Such regions typically exhibit low cross-section (background rates) and
are regions of phase space for which the underlying theoretical inputs
to the MC may be less well-understood as compared to the bulk of the phase space.
The MC simulation by itself, therefore, may not be able to adequately describe the background
processes in the SRs defined in these regions, failing to describe
either the overall cross-section of specific processes or the actual shape of the discriminating observables' distributions
therein.
The former results in a failure in the overall predicted \textit{rate}, or normalisation, of a specific backgrounds' contribution
to the event yields in the regions
and the latter results in a failure in predicting the background process' \textit{acceptance}.
In order to increase the confidence in the background estimates in such SR(s), the analyses in the present
thesis make use of the so-called \textit{control region method}.
This method is characterised by defining a (set of) region(s) in which there is (are) high purities
of the dominant background process(es).
These regions are referred to as \textit{control regions} (CRs) and are ideally defined using the
same basis of observables used to define the analysis' SRs.
The CRs are defined to be orthogonal to the SRs, meaning that no events that satisfy the requirements of
the SRs populate the CRs.
The observed data in the CRs, which are enriched in a specific background process, are used to derive
factors that correct the cross-section predictions of the MC estimates of the dominant background processes for
which the CRs are defined.
The per-process normalisation corrections, $\mu_p$, can essentially be thought of as those factors
that adjust the process' normalisation in such a way as to cover any discrepancy between the observed
data yield and MC prediction for the process in question:
\begin{align}
    \mu_{p} = \frac{  N_{\text{data}}^{\text{CR}} - \sum\limits_{\substack{i \\ i\ne p}} N_{\text{MC},\,i}^{\text{CR}}} { N_{\text{MC},\,p}^{\text{CR}}},
    \label{eq:mu_fac}
\end{align}
where `$p$' indicates the process for which the CR is defined, $N_{\text{data}}^{\text{CR}}$ is the observed
data yield in the CR, and $N_{\text{MC},\,j}^{\text{CR}}$ is the predicted yield in MC for the background
process $j$.
If there is more than one process for which a normalisation correction factor is being derived, and therefore
more than one CR, the normalisation factors are constrained by the process' contribution across all CRs in which
it is present and the expression in Equation~\ref{eq:mu_fac} is expanded into a system of equations,
\begin{align}
    N_{\text{data,sub}}^{\text{CR1}} &= \mu_i N_i^{\text{CR1}} + \mu_j N_j^{\text{CR1}} + ... \nonumber \\
    N_{\text{data,sub}}^{\text{CR2}} &= \mu_i N_i^{\text{CR2}} + \mu_j N_j^{\text{CR2}} + ...     \label{eq:mu_fac_expand} \\
        &\vdots \nonumber
\end{align}
where $N_{\text{data,sub}}^{a}$ is the observed data yield in the region $a$ with the MC predictions
for those processes not having a dedicated  CR subtracted (analogous to the numerator appearing in Equation~\ref{eq:mu_fac}),
$\mu_p$ are the normalisation factors for each process being solved for, and $N_p^{a}$ are the MC predictions
for process $p$ in region $a$.
Each process' dedicated CR ideally exhibits both a high purity of the given process and a relatively large
number of events\footnote{A too large difference in the numbers of events observed in the CRs, as compared to the SRs, may indicate
that the CRs are kinematically very different from the SRs, however.} and therefore
is the only CR that has any real constraining power on the process' normalisation correction factor.
If this is true, the expression in Equation~\ref{eq:mu_fac} generally holds true for each process with a
CR, even in the case of multiple CRs and
normalisation factors.

As mentioned above, the CRs are ideally defined using the same basis of kinematic observables
as used in the definition of the SR.
When this is the case, it is more likely that the constructed CRs probe a similar kinematic
phase space as that of the SRs.
It is important that the CRs are kinematically similar to the SRs so that the correction factors
derived in them are representative of the SRs; that is, that the underlying root cause of the need for the
correction is the same in both the CRs, where the corrections are derived, and the SRs, where
the corrections are applied.
If an SR requires high numbers of jets (high event activity), for example, but the CR is defined to have zero
jets (low event activity) then any normalisation correction derived in the CR may be correcting for physics effects
that are not relevant to the phase space probed by the SR.
In such a case, extrapolation uncertainties will generally be incurred in the final background estimate in the SR.

In addition to the CRs, so-called \textit{validation regions} (VRs) are typically defined.
The VRs are typically kinematically more similar to the SRs than the CRs, while still maintaining orthogonality
between the CRs and the SRs.
VRs are defined for each CR and allow for one to validate the extrapolation of the CR-derived normalisation
correction for each process in a region more similar to the SR.
As they are kinematically closer to the SRs, VRs are generally less pure in the specific process for which they are defined,
and will also have generally fewer events, as compared to the associated CRs.
The validation is done by comparing the post-corrected MC prediction of the backgrounds to the
observed data in the VRs, ensuring that both the overall normalisation of the backgrounds
agrees with the observed data as well as the overall shape of the relevant observables used
in the definition of the SRs.
The relationship between the CR, VR, and SR is illustrated in the one-dimensional case in
Figure~\ref{fig:sr_search_v_CR}.

When constructing a set of CRs and VRs, it is important to do so using the right set of observables
out of the total basis of observables from which the SRs are defined.
It is important that the shape as predicted by the simulated background process for which
the normalisation correction is being derived reproduces that of the observed data.
If this is not the case, then the extrapolation from the CR to the SR suffers.
This is illustrated in Figure~\ref{fig:crvr_extrap_shape} for two scenarios in which
the MC-based background prediction of the shape of the observable used for defining
the various regions both agrees and does not agree with the observed data.
If the MC simulation for the specific background for which a normalisation correction is being derived
has monotonic shape mis-modellings, as in the case of Scenario B in Figure~\ref{fig:crvr_extrap_shape},
the normalisation correction will generally not be applicable in the SR and may lead to
SR background estimates with false over- or under-predictions of the data.
In the latter case, a false excess in data may be observed and lead to mis-statements about the
likelihood of the existence of new physics in the SRs.
In the former case, a false over-prediction will lead to too-prematurely excluding the possibility for new
physics to arise when it may in fact exist.
For this reason, dedicated studies on the dependence of the derived normalisation corrections
on the set of observables used to define the CRs and VRs should generally be made so
that the analysis avoids these susceptibilities to shape mis-modellings in the MC.
Of course, this can become challenging in the general case where the SRs are defined using a large basis of potentially
correlated observables
and/or when one wishes to define several CRs to correct multiple processes' normalisations.

\begin{figure}[!htb]
    \begin{center}
        \includegraphics[width=0.65\textwidth]{figures/common_ana/sr_search_v_CRPDF}
        \caption{
            Illustration of the control region method, in the one-dimensional case analogous to that
            presented in Figure~\ref{fig:sr_search_v}.
            The control region (CR) is pure in the background process but is defined kinematically alongside the signal region (SR).
            A validation region (VR), ideally still with high background purity, is defined between the CR and SR and is used to validate
            the extrapolation of the background estimate from the CR to the SR.
            The $y$-axis represents the probability distribution of the background and signal
            processes, not their absolute yield for a given range of the quantity on the $x$-axis.
        }
        \label{fig:sr_search_v_CR}
    \end{center}
\end{figure}

\begin{figure}[!htb]
    \begin{center}
        \includegraphics[width=0.8\textwidth]{figures/common_ana/crvr_extrap_shape}
        \caption{
            Illustration of CR extrapolation scenarios in the control region method.
            In Scenario A (green data) the predicted shape of the discriminating variable
            used to define the CR, VR, and SR agrees well with the observed data
            in both the CR and VR, as seen by the flat data-to-background ratio in the bottom.
            In Scenario B (red data) the predicted shape of the discriminating variable
            differs with respect to that of the observed data, leading to an observed
            slope in the data-to-background ratio.
            In Scenario A, the normalisation correction derived for the background process
            in the CR will be well extrapolated into the VR and gives confidence in its applicability
            in the SR.
            In Scenario B, the CR-derived normalisation factor for the background process
            will pull the background prediction in the wrong direction when extrapolated
            to the VR, making the data-to-background agreement even worse and reducing
            its applicability in the SR.
        }
        \label{fig:crvr_extrap_shape}
    \end{center}
\end{figure}

\FloatBarrier
%%%%%%%%%%%%%%%%%%%%%%%%%%%%%%%%%%%%%%%%%%%%%%%%%%%%%%%%%%%%%%%%%%%%%%%%%%%%%
% SYSTEMATIC UNCERTAINTIES
%%%%%%%%%%%%%%%%%%%%%%%%%%%%%%%%%%%%%%%%%%%%%%%%%%%%%%%%%%%%%%%%%%%%%%%%%%%%%


%%%%%%%%%%%%%%%%%%%%%%%%%%%%%%%%%%%%%%%%%%%%%%%%%%%%%%%%%%%%%%%%%%%%%%
%%%%%%%%%%%%%%%%%%%%%%%%%%%%%%%%%%%%%%%%%%%%%%%%%%%%%%%%%%%%%%%%%%%%%%
%
% FAKES
%
%%%%%%%%%%%%%%%%%%%%%%%%%%%%%%%%%%%%%%%%%%%%%%%%%%%%%%%%%%%%%%%%%%%%%%
%%%%%%%%%%%%%%%%%%%%%%%%%%%%%%%%%%%%%%%%%%%%%%%%%%%%%%%%%%%%%%%%%%%%%%
\section{Estimation of Sources of Fake and Non-prompt Leptons}
\label{sec:fakes}

Despite both the high levels of accuracy achieved by the ATLAS simulation
infrastructure and the lepton reconstruction and identification algorithms
described in Chapter~\ref{chap:objects}, sources of misidentified reconstructed
leptons still exist and lead to an additional source of backgrounds to
the analyses discussed in Chapters~\ref{chap:search_stop} and \ref{chap:search_hh}.
These background sources of leptons are broken down into two categories:
\begin{itemize}
    \item \textbf{Fake leptons}: Cases in which signals in the ATLAS detector
        are selected as being leptons when in fact there is no real lepton present
    \item \textbf{Non-prompt leptons}: When real, genuine leptons are identified
        but they are not leptons originating from the primary $pp$ hard-scatter interaction process
        of interest
\end{itemize}
In the subsequent discussion, the term `fake' will be used in reference to the two
categories listed above, unless specified otherwise.

The contribution of backgrounds leading to sources of fake leptons are generally predicted
using methods based on the observed data --- referred to as `data-driven' methods ---
and arise from various sources and mechanisms.
The sources of fake leptons will be described in Section~\ref{sec:fake_lepton_sources}, separately for
electrons and muons.
Section~\ref{sec:fake_dd_motivation} provides some reasoning for why a data-driven
approach is generally taken for estimating these backgrounds.
Sections~\ref{sec:matrix_method} and \ref{sec:same_sign_extrap} go on to describe
the two data-driven approaches taken in the analyses to be presented in this thesis
for estimating fake lepton contributions: the so-called `Matrix Method' and the `Same-sign Extrapolation Method', respectively.

%%%%%%%%%%%%%%%%%%%%%%%%%%%%%%%%%%%%%%%%%%%%%%%%%%%%%%%%%%%%%%%%%%%
%%%%%%%%%%%%%%%%%%%%%%%%%%%%%%%%%%%%%%%%%%%%%%%%%%%%%%%%%%%%%%%%%%%
%
% SOURCES OF FAKE LEPTONS
%
%%%%%%%%%%%%%%%%%%%%%%%%%%%%%%%%%%%%%%%%%%%%%%%%%%%%%%%%%%%%%%%%%%%
%%%%%%%%%%%%%%%%%%%%%%%%%%%%%%%%%%%%%%%%%%%%%%%%%%%%%%%%%%%%%%%%%%%
\subsection{Sources of Fake Leptons}
\label{sec:fake_lepton_sources}

The types and sources of fake leptons generally have different experimental signatures
than those leptons that genuinely originate from the $pp$ hard-scatter.
However, due to the non-perfect lepton identification and isolation algorithms,
such sources are able to contaminate the various regions of an analysis.
The rates of contamination are generally quite low for the analyses to be presented, but their inclusion in the background
estimates of the analyses has measurable consequences nevertheless.

The analyses to be presented in the current thesis make use of $b$-tagging algorithms
to identify jets originating from $b$-hadrons.
Fake leptons, both electrons and muons, can originate from the semi-leptonic decays of
$b$- and $c$-quarks within these $b$-tagged jets, following $b\rightarrow \ell$ or cascade-type
$b \rightarrow c \rightarrow \ell$ decays of the $B$ hadrons within the jets.
The leptons resulting from such decays are typically embedded within or very close to
the originating reconstructed jet object and the lepton isolation requirements
are intended to reduce this type of background.
The subsequent paragraphs will describe additional sources of fake electrons
and muons, which generally differ between the two lepton species.

%%%%%%%%%%%%%%%%%%%%%%%%%%%%%%%%%%%%%%%%%%%%%%%%%%%%%%%%%%%%%%%%%%%
%%%%%%%%%%%%%%%%%%%%%%%%%%%%%%%%%%%%%%%%%%%%%%%%%%%%%%%%%%%%%%%%%%%
%
% SOURCES OF FAKE ELECTRONS
%
%%%%%%%%%%%%%%%%%%%%%%%%%%%%%%%%%%%%%%%%%%%%%%%%%%%%%%%%%%%%%%%%%%%
%%%%%%%%%%%%%%%%%%%%%%%%%%%%%%%%%%%%%%%%%%%%%%%%%%%%%%%%%%%%%%%%%%%
\subsubsection{Sources of Fake Electrons}
\label{sec:fake_electron_sources}

As described in Section~\ref{sec:electrons}, electrons are reconstructed based
on the presence of well-reconstructed tracks in the ID matched to deposited
energy clusters in the EM calorimeter.
Light-flavor jets, originating from the production of light quarks ($u$, $d$, $s$),
or gluon jets, which are associated with a large number of tracks due to their
increased radiation pattern, are able to fake electrons as they leave
tracks in the ID as well as subsequent energy depositions in both the EM and hadronic calorimeters.
This background, due to mis-identified jets, is typically suppressed by the use
of lepton isolation and by jet shower-shape information used in the electron identification: the
hadronic shower shapes and radial extent differ with respect to the electromagnetic shower
produced by a genuine electron.

An additional large source of fake electrons is due to photon conversion processes,
$\gamma \rightarrow e^+ e^-$, and other electromagnetic scattering processes
that happen as a result of detector material interactions.
These processes leave both tracks in the ID and electromagnetic energy depositions
in the EM calorimeter which are difficult to distinguish from genuine electrons.
Neutral hadron decays, such as the $\pi^0 \rightarrow e^+ e^- \gamma$ Dalitz decay,
also lead to electron-like signatures.
This decay of the $\pi^0$ only has a branching fraction of just over $1\%$~\cite{PDGRef}, but given
the large production of $\pi^0$ states in the $pp$ collision this has the potential to be
a relevant source of fake electrons.
These electromagnetic sources of fake electrons are distinguished by their generally
larger impact parameters relative to genuine prompt electrons.

\begin{figure}[!htb]
    \begin{center}
        \includegraphics[width=0.45\textwidth]{figures/common_ana/fakes/electron_brem_fake}
        \caption{
            Illustration of electron bremsstrahlung radiation with photon conversion exhibiting
            `trident' topology.
        }
        \label{fig:electron_brem_fake}
    \end{center}
\end{figure}


%%%%%%%%%%%%%%%%%%%%%%%%%%%%%%%%%%%%%%%%%%%%%%%%%%%%%%%%%%%%%%%%%%%
%%%%%%%%%%%%%%%%%%%%%%%%%%%%%%%%%%%%%%%%%%%%%%%%%%%%%%%%%%%%%%%%%%%
%
% SOURCES OF FAKE MUONS
%
%%%%%%%%%%%%%%%%%%%%%%%%%%%%%%%%%%%%%%%%%%%%%%%%%%%%%%%%%%%%%%%%%%%
%%%%%%%%%%%%%%%%%%%%%%%%%%%%%%%%%%%%%%%%%%%%%%%%%%%%%%%%%%%%%%%%%%%
\subsubsection{Sources of Fake Muons}
\label{sec:fake_muon_sources}

As described in Section~\ref{sec:muons}, muons are primarily reconstructed via the combination
of tracking information provided by the ID and MS, and, generally speaking, they should be the only particle species to reach
the MS.
In addition to the semi-leptonic decays of heavy-flavored jets described above, however,
there are several sources of fake muons.
Highly energetic jets can have elongated shower profiles that reach the outer
radii of the hadronic calorimeter, with a non-zero chance of exiting the calorimeter
and resulting in particle leakage into the MS.
Such cases are referred to as calorimeter punch-through, and have been illustrated
in Figure~\ref{fig:jet_punch_through}.
Punch-through particles can leave signatures similar to charged muons whose subsequent
MS tracks are associated with a track in the ID, leading to a reconstructed combined muon
faking a genuine muon.
An additional source of fake muons come from the in-flight decays of charged hadrons,
such as the $K^\pm$ and $\pi^{\pm}$, that can decay to $\mu^{\pm} \nu$.
These sources of non-prompt muons typically result in low-\pT~muons and are characterised
by combined tracks with exhibiting a kink topology,
briefly mentioned when discussing the muon combined reconstruction in Section~\ref{sec:muon_id}.
The \pT~dependence of muon cross-sections for various sources of muons (prompt, fake, and non-prompt)
is shown in Figure~\ref{fig:fake_muon_kink}, along with an illustration of the kinked-track topology
characterising the in-flight decays of Kaons and charged pions.

\begin{figure}[!htb]
    \begin{center}
        \includegraphics[width=0.43\textwidth]{figures/common_ana/fakes/reco_muon_sources}
        \raisebox{0.45cm}{\includegraphics[width=0.54\textwidth]{figures/common_ana/fake_muon_kinkPDF}}
        \caption{
            \textbf{\textit{Left}}:  Transverse momentum dependence of muon cross-sections for muons originating
                from various prompt and non-prompt sources.
                Figure taken from Ref.~\cite{CERN-LHCC-97-022}.
            \textbf{\textit{Right}}: Illustration of a reconstructed non-prompt muon resulting from a kinked-track topology.
                A produced $K^{\pm}$ or $\pi^{\pm}$ is produced and decays in-flight to a muon and muon-neutrino.
                The point at which the hadron decays is indicated by the yellow dot.
                The red circles indicate detector hits in the ID and MS layers indicated
                by the horizontal black lines.
        }
        \label{fig:fake_muon_kink}
    \end{center}
\end{figure}


%%%%%%%%%%%%%%%%%%%%%%%%%%%%%%%%%%%%%%%%%%%%%%%%%%%%%%%%%%%%%%%%%%%
%%%%%%%%%%%%%%%%%%%%%%%%%%%%%%%%%%%%%%%%%%%%%%%%%%%%%%%%%%%%%%%%%%%
%
% DATA DRIVEN MOTIVATION
%
%%%%%%%%%%%%%%%%%%%%%%%%%%%%%%%%%%%%%%%%%%%%%%%%%%%%%%%%%%%%%%%%%%%
%%%%%%%%%%%%%%%%%%%%%%%%%%%%%%%%%%%%%%%%%%%%%%%%%%%%%%%%%%%%%%%%%%%
\subsection{The Need for a Data-driven Approach}
\label{sec:fake_dd_motivation}

In the analyses to be presented in Chapters~\ref{chap:search_stop} and \ref{chap:search_hh},
relatively tight identification working points are used for electrons and muons.
As a result, the contamination of fake leptons in these analyses is relatively
minor.
Although small, their contamination does have measurable effects and so their contribution
must be accounted for in order to achieve accurate estimates of the backgrounds
in each analysis.

Several methods exist to estimate the background rates arising from the sources of fake
leptons, those relying on data-driven methods or those based entirely on the MC
simulation.
Relying on the MC simulation of these sources of fake leptons, described in previous
sections, means to rely entirely on the \textsc{GEANT4} simulation of the ATLAS detector
and on the MC generation and showering processes to accurately predict the
rates of these processes.
There are several problems with this approach and they are (non-exhaustively) as follows.
Given the very small region of phase space being probed by the analysis,
the number of MC events needed to appropriately sample the sources of production of fake
leptons as described above would be prohibitively large if a statistically relevant sample
is desired.
An accurate prediction of the production rates of several of these fake lepton sources
would require an accurate underlying theoretical model of many processes, such as
heavy-flavor jet fragmentation, which is challenging.
Additionally, many sources of fake leptons arise as a result of detector material interactions
or as a result of subtle and difficult-to-model failure modes of the detector response.
%or inaccurate simulation of the detector response.
The accurate prediction of the rate of electron bremsstrahlung and photon conversions, for example, requires
high levels of precision in the simulation and measurement of the active and passive material in the ATLAS detector and cavern,
which is not necessarily possible.
The rates of jets being mis-identified as electrons and jet punch-through, for example,
require that the MC simulation of the calorimeter response and shower evolution are
accurately modelled.
The MC simulation is not expected to perform to the degree at which these subtle, and comparatively rare, effects
are accurately predicted.
For this reason, data-driven approaches are typically taken for estimating the background rates
of these fake lepton sources.
In the analyses to be presented, two data-driven approaches are taken.
In the search described in Chapter~\ref{chap:search_stop}, the Matrix Method
is used.
In the search described in Chapter~\ref{chap:search_hh}, the Same-sign Extrapolation
Method is used.
These methods are introduced in Sections~\ref{sec:matrix_method} and \ref{sec:same_sign_extrap}, respectively.


\FloatBarrier
%%%%%%%%%%%%%%%%%%%%%%%%%%%%%%%%%%%%%%%%%%%%%%%%%%%%%%%%%%%%%%%%%%%
%%%%%%%%%%%%%%%%%%%%%%%%%%%%%%%%%%%%%%%%%%%%%%%%%%%%%%%%%%%%%%%%%%%
%
% THE MATRIX METHOD
%
%%%%%%%%%%%%%%%%%%%%%%%%%%%%%%%%%%%%%%%%%%%%%%%%%%%%%%%%%%%%%%%%%%%
%%%%%%%%%%%%%%%%%%%%%%%%%%%%%%%%%%%%%%%%%%%%%%%%%%%%%%%%%%%%%%%%%%%
\subsection{The Matrix Method}
\label{sec:matrix_method}

The Matrix Method, discussed thoroughly in Ref.~\cite{TOPFake}, is one of the most common
methods used in ATLAS analyses for estimating backgrounds due to processes containing
fake leptons.
It is characterised by the definition of two levels of lepton selection:
\begin{itemize}
    \item[]\textbf{Tight Leptons}: Those leptons passing all reconstruction, identification, and isolation criteria
        as the leptons used in the final analysis' results
    \item[]\textbf{Loose Leptons}: Leptons requiring similar selections as the Tight leptons but typically with either, or both, identification
        and isolation criteria relaxed
\end{itemize}
The Tight leptons are a subset of the Loose, by definition.
In the analysis described in Chapter~\ref{chap:search_stop}, the Loose leptons are defined by loosening
only the lepton identification working points.
Generally speaking, both samples of Loose and Tight leptons will contain
both fake and real leptons.\footnote{Genuine, prompt
leptons originating from the $pp$ hard-scatter interaction point are typically referred to as `real' leptons
in order to distinguish them, semantically, from fake and non-prompt leptons.
}
The Matrix Method consists of measuring a set of efficiencies: the \textit{real}
(\textit{fake}) \textit{efficiencies}, $\varepsilon_r$ ($\varepsilon_f$),
defined as the efficiency for a real (fake) electron or muon that satisfies the Loose selection criteria
to also satisfy the Tight selection criteria.
This is illustrated in Figure~\ref{fig:fake_effs}.
As illustrated, both the Loose and Tight lepton samples will contain contributions of both fake
and real leptons.
The Matrix Method can be generalised to final states with any number of leptons.
In the discussion to follow, we will discuss that of final states with two leptons: the dilepton Matrix Method.

\begin{figure}[!htb]
    \begin{center}
        \includegraphics[width=0.65\textwidth]{figures/common_ana/fakes/fake_effs_illustration}
        \caption{
            Illustration of the loose and tight lepton samples used in the Matrix Method.
        }
        \label{fig:fake_effs}
    \end{center}
\end{figure}

The real efficiencies, $\varepsilon_r$, are measured in data using $Z$-boson tag-and-probe methods, requiring the probe lepton
to satisfy the Tight lepton selection and to be matched to the trigger.
The probe lepton must satisfy the Loose lepton selection.
The fraction of Loose probe leptons to pass the Tight lepton selection then gives a measure of $\varepsilon_r$.

The fake efficiencies, $\varepsilon_f$, are measured in data using events with different-flavor leptons,
where one lepton is an electron and the other is a muon, that have the same electric charge.
A tag-and-probe method similar to that used in the measurement of $\varepsilon_r$ is used
and relies on the fact that a comparatively small amount of SM processes can result in same-sign
and different-flavor events.
Therefore, when a probe satisfies the Tight selection it is very likely to be the case that the
probe lepton is a fake lepton.
An additional component of the $\varepsilon_f$ is measured by additionally requesting that there
be at least one $b$-tagged jet in the same-sign, different-flavor selection.
This allows for $\varepsilon_f$ to be measured in a region enriched in fake leptons originating
from semi-leptonic decays of heavy-flavor jets.
The various measurements of $\varepsilon_f$ are combined following an averaging scheme, weighted according
to the composition of fake lepton sources expected to contaminate the SRs.

The real and fake efficiencies are not global quantities.
Instead, they are measured as a function of both the lepton $\pT$ and $\eta$
such that they may track the effects of changes in detector response and
material interaction over the $\pT$ and $\eta$ ranges relevant to the leptons used in the analysis.

Once $\varepsilon_r$ and $\varepsilon_f$ are obtained, the fake lepton background
can be obtained by inverting the equation relating the measured quantities (Tight versus Loose), taken
from the observed data, in terms
of those that we want to know (Fake versus Real):
\begin{equation}
    \begin{pmatrix}
        N_{TT} \\ N_{TL} \\ N_{LT} \\ N_{LL}
    \end{pmatrix}
        = M
    \begin{pmatrix}
        N_{LL}^{RR} \\ N_{LL}^{RF} \\ N_{LL}^{FR} \\ N_{LL}^{FF}
    \end{pmatrix}
    \label{eq:matrix_method}
\end{equation}
\noindent where in the sub- and super-scripts, the first (second) index refers to the leading (sub-leading) lepton.
The sub-script `$T$' (`$L$') refers to the lepton passing the Tight (Loose) lepton
selection.
The super-script `$R$' (`$F$') indicates whether or not the lepton is a real (fake) lepton.
For example, the quantity $N_{LL}^{FR}$ is the number of events in which the leading lepton
in the sample of Loose leptons is fake and the sub-leading is real.
The matrix $M$ is given by,
\begin{align}
    M = \begin{pmatrix}
            \varepsilon_{r,1}\,\varepsilon_{r,2} & \varepsilon_{r,1}\,\varepsilon_{f,2}  & \varepsilon_{f,1}\, \varepsilon_{r,2} & \varepsilon_{f,1}\, \varepsilon_{f,2} \\
            \varepsilon_{r,1}\, \overline{\varepsilon_{r,2}} & \varepsilon_{r,1}\,\overline{\varepsilon_{f,2}} & \varepsilon_{f,1}\, \overline{\varepsilon_{r.2}} & \varepsilon_{f,1}\, \overline{\varepsilon_{f,2}} \\
            \overline{\varepsilon_{r,1}} \varepsilon_{r,2} & \overline{\varepsilon_{r,1}}\, \varepsilon_{f,2} & \overline{\varepsilon_{f,1}}\, \varepsilon_{r,2} & \overline{\varepsilon_{f,1}}\, \varepsilon_{f,2} \\
            \overline{\varepsilon_{r,1}}\, \overline{\varepsilon_{r,2}} & \overline{\varepsilon_{r,1}}\, \overline{\varepsilon_{f,2}} & \overline{\varepsilon_{f,1}}\, \overline{\varepsilon_{r,2}} & \overline{\varepsilon_{f,1}}\, \overline{\varepsilon_{f,2}}
        \end{pmatrix}
    \label{eq:matrix_method_matrix}
\end{align}
where the notation $\overline{\varepsilon}$ indicates $(1 - \varepsilon)$.

The number of events with double-fake and single-fake leptons satisfying the analysis' Tight selection
($N_{TT}^{FF}$ and $N_{TT}^{RF} + N_{TT}^{FR}$, respectively) can then be obtained from the
number of events with double-fake and single-fake leptons satisfying the Loose selection
($N_{LL}^{FF}$ and $N_{LL}^{RF} + N_{LL}^{FR}$, respectively) through inversion
of Equation~\ref{eq:matrix_method}  and
noting the following relations:
\begin{align}
    N_{TT}^{RR} &= \varepsilon_{r,1}\,\varepsilon_{r,2} \times N_{LL}^{RR}  \label{eq:matrix_method_sol0}\\
    N_{TT}^{RF} &= \varepsilon_{r,1}\,\varepsilon_{f,2} \times N_{LL}^{RF}  \label{eq:matrix_method_sol1}\\
    N_{TT}^{FR} &= \varepsilon_{f,1}\,\varepsilon_{r,2} \times N_{LL}^{FR}  \label{eq:matrix_method_sol2}\\
    N_{TT}^{FF} &= \varepsilon_{f,1}\,\varepsilon_{f,2} \times N_{LL}^{FF}. \label{eq:matrix_method_sol3}
\end{align}
The quantities appearing on the right-hand-side of Equations~\ref{eq:matrix_method_sol0}-\ref{eq:matrix_method_sol3}
depend entirely on the observed data.
The number of events in the analysis' Tight selection that have at least one fake lepton is then given by the sum
$N_{TT}^{RF} + N_{TT}^{FR} + N_{TT}^{FF}$.
This gives a total integrated yield for the fake background contribution in the analysis'
Tight selection; however, from Equations~\ref{eq:matrix_method_sol1}-\ref{eq:matrix_method_sol3}
a set of per-event weighting factors (`fake weights'), depending only on the $\varepsilon_r(\pT,\eta)$ and
$\varepsilon_f(\pT,\eta)$ efficiency factors for the two leptons, can be defined.
These fake weights allow for kinematic distributions of the fake lepton background
sources to be populated
by appropriately applying them to events in the data sample consisting of leptons satisfying the analysis' Loose selection.



%%%%%%%%%%%%%%%%%%%%%%%%%%%%%%%%%%%%%%%%%%%%%%%%%%%%%%%%%%%%%%%%%%%
%%%%%%%%%%%%%%%%%%%%%%%%%%%%%%%%%%%%%%%%%%%%%%%%%%%%%%%%%%%%%%%%%%%
%
% SAME SIGN EXTRAPOLATION
%
%%%%%%%%%%%%%%%%%%%%%%%%%%%%%%%%%%%%%%%%%%%%%%%%%%%%%%%%%%%%%%%%%%%
%%%%%%%%%%%%%%%%%%%%%%%%%%%%%%%%%%%%%%%%%%%%%%%%%%%%%%%%%%%%%%%%%%%
\subsection{Same-sign Extrapolation Method}
\label{sec:same_sign_extrap}

The Same-sign Extrapolation Method is another method for
estimating the numbers of events with fake and non-prompt leptons.
This method is well described in Refs.~\cite{TOPQ-2015-09,TOPQ-2017-05}.
The method is tailored for dilepton analyses that require the two
leptons to have opposite electric charge and takes advantage of the fact that the
production mechanisms for the fake leptons described above are generally
uncorrelated with the charges of leptons in the event.
This means that the rates of the various sources of fake leptons will generally
be the same in the sample of oppositely-charged dilepton events and in the
sample of dilepton events in which the leptons have the same charge.
The former are referred to as opposite-sign (OS) events and the latter
as same-sign (SS) events.
This assumption is not entirely correct, however, since the underlying sources of
the fake and/or non-prompt leptons are not all charge symmetric.\footnote{`Charge-symmetric' means that the
dilepton final state of a given process may be either OS or SS, with equal probability of each case occurring.}
For example, a dominant source of mis-identified leptons in dileptonic events
arises from the production of semi-leptonically decaying top-quark pairs, in which the sub-leading reconstructed
electrons arise from a mis-identified jet.
This process is charge-symmetric since
converted photons produced in jets are equally likely to give rise to a reconstructed
$e^+$ or $e^-$ and are uncorrelated to the sign of the real lepton.
In trident decays whereby bremsstrahlung photons (radiated from the lepton children
of the top-quarks) undergo conversion processes, however, there is a partial
charge correlation of the reconstructed (non-prompt) electron with the charge of the original lepton,
and hence conversion processes contribute more to the OS sample of events.
These two cases are illustrated in Figure~\ref{fig:ttbar_fake_charge_sym}.
The lepton definitions within the OS and SS event samples are the same.
That is, they use the same reconstruction, identification, and isolation working points.
This is compared to the Matrix Method (Section~\ref{sec:matrix_method}), in which the two samples of leptons used in the technique's
implementation differ by their lepton definitions.

\begin{figure}[!htb]
    \begin{center}
        \includegraphics[width=0.75\textwidth]{figures/common_ana/fakes/ttbar_fake_charge_sym_semi}
        \includegraphics[width=0.7\textwidth]{figures/common_ana/fakes/ttbar_fake_charge_sym_trident}
        \caption{
            Illustration of charge-asymmetry in fake lepton production
            arising in top-quark pair production events, leading to
            the rate of photon conversion sources of fake electrons being larger in OS dilepton events.
            \textbf{\textit{Top}}: Shower photons arising from decays within one of the jets
                in semi-leptonic decays of top-quark pairs may convert to an electron-positron pair,
                leading to one of the jets being reconstructed as an electron.
                Looking at the charge possibilities of each side of the top-quark pair decay,
                the event is equally likely to be classified as either an OS or SS event.
            \textbf{\textit{Bottom}}: Trident events arising in dileptonic top-quark pair production
                events can lead to a non-prompt electron from the photon conversion being selected
                as one of the event's candidate leptons.
                The overall charge possibilities of the lepton charges on the side of the trident decay
                are correlated with the charge of the initial lepton.
                Looking at the charge combinations possible between both sides of the top-quark pair decay,
                the event is more likely to be classified as an OS event.
        }
        \label{fig:ttbar_fake_charge_sym}
    \end{center}
\end{figure}

The method also assumes that the rate of dilepton events in which \textit{both} leptons
are fake is negligible, and that the composition of the fake background contributing to
the dilepton final state is composed of events in which one of the leptons is real.
In the majority of cases, the sub-leading lepton is the fake lepton.
For this reason, when using the MC simulation to gain predictions of the composition of
the fake backgrounds, the MC events in which only one of the leptons is fake are considered.
This will become clear in the discussion to follow and in the specific implementation described
in Chapter~\ref{chap:search_hh}.


The general method works as follows.
Since, as described above, the sources of the various sources of fake backgrounds in the OS and SS samples of events
are generally the same, the method relies on using the sample of SS events to provide a template of the fake backgrounds
to be used in the OS selections.
The contribution of fakes to each of the regions (CR, VR, or SR) in the analysis is estimated
by subtracting the prediction of the prompt (real) SM backgrounds from the observed data
in the associated SS selections, defined similarly to the OS selections used in the analysis but with the
charge requirements inverted.
The ratio of the number of OS to SS events with fake leptons, $f^{SS \rightarrow OS}$,
is taken entirely from MC and is applied to the SS data that has had the prompt-MC contribution
subtracted in order to extrapolate this number to the OS regions.
This SS extrapolation is described by Equation~\ref{eq:ss_extrap}:
\begin{align}
    N_{\text{OS}}^{\text{fake}} &= f^{SS \rightarrow OS} \times N_{\text{SS}}^{\text{fake}} \nonumber \\
        &= \frac{ N_{\text{MC,OS}}^{\text{fake}} }{ N_{\text{MC,SS}}^{\text{fake}} } \times ( N_{\text{data,SS}} - N_{\text{MC,SS}}^{\text{real}} )
        \label{eq:ss_extrap}
\end{align}
In addition to providing the overall yields of the fake backgrounds in a given region,
the method described by Equation~\ref{eq:ss_extrap} provides the means to
inspect the kinematics of the predicted fake backgrounds by simply computing
Equation~\ref{eq:ss_extrap} on a bin-by-bin basis when populating histograms
of kinematic observables.

Further details on the implementation of the Same-sign Extrapolation Method,
described by Equation~\ref{eq:ss_extrap}, will be given in Chapter~\ref{chap:search_hh}.
In particular, the implicit sensitivities to the MC simulation will be described,
as well as additional extrapolations specific to the analysis.


%%%%%%%%%%%%%%%%%%%%%%%%%%%%%%%%%%%%%%%%%%%%%%%%%%%%%%%%%%%%%%%%%%%%%%
%%%%%%%%%%%%%%%%%%%%%%%%%%%%%%%%%%%%%%%%%%%%%%%%%%%%%%%%%%%%%%%%%%%%%%
%
% SYSTEMATIC UNCERTAINTIES
%
%%%%%%%%%%%%%%%%%%%%%%%%%%%%%%%%%%%%%%%%%%%%%%%%%%%%%%%%%%%%%%%%%%%%%%
%%%%%%%%%%%%%%%%%%%%%%%%%%%%%%%%%%%%%%%%%%%%%%%%%%%%%%%%%%%%%%%%%%%%%%
\FloatBarrier
\section{Systematic Uncertainties}
\label{sec:common_systematics}

There are many sources of systematic uncertainty (`systematics') that affect the results of the analyses
to be presented in Chapters~\ref{chap:search_stop} and \ref{chap:search_hh}.
The considered sources of systematic uncertainty are listed in Table~\ref{tab:syst_summary}.
There are uncertainties related to the overall event (e.g. the uncertainty
on the luminosity measurement), on the reconstruction and identification of
the physics objects described in Section~\ref{chap:objects}, and in the MC
simulation of both the SM background and signal processes.
Brief descriptions of the sources of systematic uncertainty
appearing in Table~\ref{tab:syst_summary} are given in Sections~\ref{sec:syst_experimental}-\ref{sec:syst_sig_modelling}.

\begin{table}[!htb]
    \caption{
        Summary of the sources of systematic uncertainties affecting the measurements
        in the analyses discussed in Chapters~\ref{chap:search_stop} and \ref{chap:search_hh}.
        Sources of uncertainty in {\color{red}{red}} ({\color{blue}{blue}}) pertain only to the
        search presented in Chapter~\ref{chap:search_stop} (\ref{chap:search_hh}).
        Those in black are considered in both analyses.
        For the uncertainties related to the SM background modelling, it is indicated
        whether or not they are computed using the Transfer Factor Method (Section~\ref{sec:transfer_factor}).
    }
    \label{tab:syst_summary}
    \begin{footnotesize}
    \begin{center}
        %\begin{tabularx}{\textwidth}{@{\extracolsep{\fill}}c c c c}
        %\begin{tabularx}{\textwidth}{c c c c}
        \begin{tabular}{c c c c}
        \toprule
        \hline
        \multicolumn{4}{c}{\textbf{Event-level}} \\
        \hline
        \multicolumn{4}{c}{Luminosity Measurement} \\
        \multicolumn{4}{c}{Pile-up Modelling} \\
%        \multicolumn{3}{c}{Luminosity} & \multicolumn{2}{c}{ Pile-up }  \\
        \hline
        \midrule
        \multicolumn{4}{c}{\textbf{Object Reconstruction}} \\ \hline
        \hspace{-2cm} \underline{\textbf{Jets}} &     \hspace{-1.8cm} \underline{\textbf{Flavor Tagging}} & \hspace{0.5cm}\underline{\textbf{Leptons}} & \hspace{0.3cm} \underline{\textbf{\met}} \\
        \hspace{-2cm} Jet Energy Scale (JES) &        \hspace{-1.8cm} $b$-tag Eff. & \hspace{0.5cm}Reconstruction Eff. & \hspace{0.3cm} Soft-term Resolution \\
        \hspace{-2cm} Jet Energy Resolution (JER) &   \hspace{-1.8cm} Mis-tag Eff. & \hspace{0.5cm}Identification Eff. & \hspace{0.3cm}  Soft-term Scale \\
        \hspace{-2cm} Pile-up Suppression (JVT) &     \hspace{-1.8cm}   & \hspace{0.5cm}Isolation Eff. &\hspace{0.3cm}  \\
        \hspace{-2cm} &     \hspace{-1.8cm}   & \hspace{0.5cm}Trigger Eff. &\hspace{0.3cm}  \\
        \midrule
        \midrule
        \multicolumn{4}{c}{\textbf{Background Modelling}} \\
        \hline
        \multicolumn{1}{l}{\textbf{Souce of Uncertainty}} & \multicolumn{2}{c}{\textbf{Affected Background Processes}} & \textbf{Transfer Factor Approach?} \\
        \hline
        \multicolumn{1}{l}{Hard-Scatter Generation}  & \multicolumn{2}{c}{\ttbar, \wt, \color{blue}{\zhf}} & Yes \\
        \multicolumn{1}{l}{Fragmentation} & \multicolumn{2}{c}{\ttbar, \wt} & Yes \\
        \multicolumn{1}{l}{Additional Radiation (ISR and FSR variation)} & \multicolumn{2}{c}{\ttbar, \wt} & Yes \\
        \multicolumn{1}{l}{PDF Choice and Uncertainty} & \multicolumn{2}{c}{\ttbar, \wt, \color{blue}{\zhf}, \color{red}{\vv}} & Yes \\
        \multicolumn{1}{l}{Scale ($\mu_R$, $\mu_F$) Variations} & \multicolumn{2}{c}{\ttbar, \wt, \color{red}{\vv}, \color{blue}{\zhf}} & Yes \\
        \multicolumn{1}{l}{Cross-section Uncertainty} & \multicolumn{2}{c}{\color{blue}{\ttbar}, \color{blue}{\wt}} & No \\
        \multicolumn{1}{l}{\ttbar~Interference Uncertainty} & \multicolumn{2}{c}{\wt} & Yes \\
        \multicolumn{1}{l}{Prompt Subtraction \& Fake Composition} & \multicolumn{2}{c}{Fake Lepton Background} & No \\ 
        \multicolumn{1}{l}{SS-OS Extrapolation} & \multicolumn{2}{c}{\color{blue}{Fake Lepton Background}} & No \\ 
        \midrule
        \midrule
        \multicolumn{4}{c}{\textbf{Signal Modelling}} \\
        \hline
        \multicolumn{4}{c}{{\color{blue}{Fragmentation}}} \\
        \multicolumn{4}{c}{Scale ($\mu_R$, $\mu_F$) Variations} \\
        \multicolumn{4}{c}{PDF Choice and Uncertainty} \\
        \multicolumn{4}{c}{Cross-section Uncertainty} \\
        
%        %\multicolumn{1}{l|}{ }  & \underline{\ttbar} & \underline{$Wt$} & \underline{Diboson} & \underline{$Z$+jets} & \underline{Fake Lepton Estimate} \\
%        \multicolumn{1}{l|}{Uncertainty Source }  & \ttbar & $Wt$ & Diboson & $Z$+jets & Fake Lepton \\
%        \hline
%        \multicolumn{1}{l|}{Hard-Scatter Generation}  & \checkmark & \checkmark & & & n/a\\
%        \multicolumn{1}{l|}{Fragmentation}  & \checkmark & \checkmark  & & & \\
%        \multicolumn{1}{l|}{Additional Radiation} & \checkmark & \checkmark & & & \\
%        \multicolumn{1}{l|}{PDF} & \checkmark & \checkmark &  & \checkmark & \\
%        \multicolumn{1}{l|}{Scale ($\mu_R$, $\mu_F$) Variations} & \checkmark & \checkmark & & & \\
%        \multicolumn{1}{l|}{Cross-section} & \checkmark$_{HH}$& & & & \\
%        HS + Matching & HS + Matching & & &  \\
%        Frag. + Had. & Frag. + Had. & & &  \\
        \hline
        \bottomrule
        \end{tabular}
    \end{center}
    \end{footnotesize}
\end{table}


%%%%%%%%%%%%%%%%%%%%%%%%%%%%%%%%%%%%%%%%%%%%%%%%%%%%%%%%%%%%%%%%%%%
%%%%%%%%%%%%%%%%%%%%%%%%%%%%%%%%%%%%%%%%%%%%%%%%%%%%%%%%%%%%%%%%%%%
%
% EXPERIMENTAL UNCERTAINTIES
%
%%%%%%%%%%%%%%%%%%%%%%%%%%%%%%%%%%%%%%%%%%%%%%%%%%%%%%%%%%%%%%%%%%%
%%%%%%%%%%%%%%%%%%%%%%%%%%%%%%%%%%%%%%%%%%%%%%%%%%%%%%%%%%%%%%%%%%%
\subsection{Experimental Uncertainties}
\label{sec:syst_experimental}

\subsubsection{Event-wide Uncertainties}
Event-wide (process-independent) uncertainties affecting the overall normalisation of the processes
relate to both the luminosity and pileup measurements.
The uncertainty on the integrated luminosity used to normalise all MC simulated
processes is derived following the methodology described in Ref.~\cite{LumiUncert}.
For the 2015+2016 (full Run 2) dataset, relevant to the analysis described in Chapter~\ref{chap:search_stop} (\ref{chap:search_hh}),
this uncertainty was found to be 2.1\% (1.7\%).
The uncertainty on the luminosity measurement does not affect processes whose
SR normalisation is constrained by data in dedicated CRs.

An uncertainty is considered on re-weighting the pileup distributions in the
MC simulation.
The re-weighting is applied in order to correct for the differences in the actual pile-up
distributions observed in data and those assumed at the time of producing the MC simulation (Section~\ref{sec:pileup_sim}).

\subsubsection{Jet Uncertainties}
The systematic uncertainties on reconstructed jet objects are related to the jet energy
resolution (JER), jet energy scale (JES), and JVT.
There are many sources of uncertainties related to the JES and JER, each related to a specific
part of the JES and JER calibration measurements, as described in Section~\ref{sec:jet_calib}.
They arise from the techniques and corrections derived in MC, including statistical, detector,
modelling effects, jet flavor compositions, pileup corrections, and $\eta$-dependence effects.
The effects of the JES and JER uncertainties are among the more dominant sources of uncertainty
on the final analysis results in both analyses to be presented.
Given the complexity of the JES and JER calibrations, there are nearly 100 components associated with their uncertainties
that must be incorporated into an analysis' measurement uncertainty.
In general, the leading uncertainty in searches such as those to be presented in the current thesis
is statistical in nature --- related to the limited statistics in data and in the MC simulation used
for the final SR predictions ---  and not related to the systematic uncertainties.
For this reason, the searches described in Chapters~\ref{chap:search_stop} and \ref{chap:search_hh}
used a reduced set of JES and JER uncertainties that are derived following a Principal Component Analysis (PCA)
designed to capture only the dominant components --- and their correlations --- of the total set of JES and JER uncertainty components
that are relevant to the phase space being probed by the analyses.
In the analysis presented in Chapter~\ref{chap:search_stop} (\ref{chap:search_hh}), this uncertainty reduction
process leads to only 4 (34) separate components of the combined JES and JER systematic uncertainty.

\subsubsection{Flavor Tagging Uncertainties}
There are uncertainties in the jet flavor tagging efficiencies, as well as in
the measured mis-tagging efficiencies associated with $c$- and light jets.
They are a mixture of statistical, experimental, and modelling uncertainties
incurred during the flavor tagging calibration procedures.
The uncertainties enter into the analyses through their impact on the scale-factors,
described in Section~\ref{sec:ftag_calib},
that are applied in the analysis.
Given the importance of $b$-tagged jets in the final states of the signal processes
in the analyses described in Chapters~\ref{chap:search_stop} and \ref{chap:search_hh},
these uncertainties have non-negligible impact on the analyses' results.

\subsubsection{Lepton Uncertainties}
Uncertainties on the measurement of leptons correspond to the electron and muon reconstruction,
identification, trigger, and isolation efficiencies in a manner similar to the flavor tagging
in that systematic variations incurred in the associated scale-factor measurements are applied
in the analysis.
Additional uncertainties related to the lepton kinematics due to the resolution and scale of the
electron (muon) energy (momentum) measurement are considered.
The muon momentum measurement uncertainties are derived for both the ID and MS measurement
of the combined muons used in the analyses.

\subsubsection{Missing Transverse Momentum, \met}
Systematic variations of the \met are coherently incurred as a result of the
systematic variations, described above, being applied to the objects provided as input to the \met calculation:
the leptons and jets.
Additional uncertainties related to the scale and resolution of the soft-term of the \met calculation
are also considered.
Given that the analyses considered in Chapters~\ref{chap:search_stop} and \ref{chap:search_hh}
are characterised by real sources of \met, the soft-term component plays a small role in the magnitude
of the \met and therefore its uncertainties have negligible impact on the analyses.
Generally, the \met uncertainties are sub-dominant.

\subsubsection{Fake Lepton Estimate}

As will be seen in Chapters~\ref{chap:search_stop} and \ref{chap:search_hh}, the overall contribution
of the fake background processes to the analyses to be presented in this thesis is very small.
As a result, the systematic uncertainties related to this background are almost always negligible
in impact on the analyses' final results.
However, given the subtle nature of these data-driven estimates, we describe the methods by which
systematic uncertainties are derived for them.
They are as follows:

\begin{description}
    \item{Matrix Method (Section~\ref{sec:matrix_method}):} There are three primary sources of systematic
        uncertainty ascribed to the fake estimate derived from the Matrix Method in the analysis described
        in Chapter~\ref{chap:search_stop}.
        The first is related to the limited statistics in the region(s) used for the determination of the
        fake efficiencies, $\varepsilon_r$ and $\varepsilon_f$.
        The second is related to the `prompt subtraction': the evaluation of the portion of real events contaminating the
        regions in which the fake efficiencies are derived, which is evaluated using the MC simulation.       
        The component of real lepton sources is varied by $\pm 30$\% and the impact on the resulting
        fake efficiencies is propagated to to the final analysis. 
        The third component is related to the compositional differences of the background sources leading
        to fake leptons in the region in which the fake efficiencies are derived and in the regions (SRs)
        in which the fake estimate is applied.
        To assess the systematic related to compositional differences, alternative regions in which the leading
        components of the fake backgrounds (e.g. the heavy flavor component) are varied are defined and the impact on the measured fake efficiencies
        is used to define an uncertainty.
    \item{Same-sign Extrapolation Method (Section~\ref{sec:same_sign_extrap}):} As with the Matrix Method,
        there are three primary sources of uncertainty prescribed to the fake background estimate derived using
        the Same-sign Extrapolation Method. The first, as with the Matrix Method, is related to the prompt subtraction
        in which the rate of contamination of prompt processes is varied.
        In the analysis described in Chapter~\ref{chap:search_hh}, the real contamination in the same-sign
        regions is varied by $\pm 50$\%, and the impact on the final fake background estimate is used to quantify
        an uncertainty.
        Following the studies in Refs.~\cite{TOPQ-2015-09,TOPQ-2017-05}, an additional uncertainty (on top of the statistical component) on the
        extrapolation from the SS to the OS regions is included by varying the $f^{SS \rightarrow OS}$ factors
        by $\pm 20$\% and assessing the impact on the final fake background prediction.
        The statistical uncertainty related to the extrapolation over the $d_{hh}$ discriminant,
        described in Chapter~\ref{chap:search_hh}, is applied as an additional uncertainty on this fake background estimate.
        Uncertainties related to the composition of fakes in the SS and OS regions are found to be small,
        and are covered by the extrapolation factor uncertainties already described.
\end{description}



%%%%%%%%%%%%%%%%%%%%%%%%%%%%%%%%%%%%%%%%%%%%%%%%%%%%%%%%%%%%%%%%%%%
%%%%%%%%%%%%%%%%%%%%%%%%%%%%%%%%%%%%%%%%%%%%%%%%%%%%%%%%%%%%%%%%%%%
%
% TRANSFER FACTOR METHOD
%
%%%%%%%%%%%%%%%%%%%%%%%%%%%%%%%%%%%%%%%%%%%%%%%%%%%%%%%%%%%%%%%%%%%
%%%%%%%%%%%%%%%%%%%%%%%%%%%%%%%%%%%%%%%%%%%%%%%%%%%%%%%%%%%%%%%%%%%
\subsection{The Transfer Factor Method}
\label{sec:transfer_factor}

For estimating the impact of modelling uncertainties on the SM backgrounds, described in Section~\ref{sec:syst_bkg_modelling},  that
have dedicated CRs to constrain their overall normalisation in the SRs,
the so-called Transfer Factor (TF) Method is used.
Since the purpose of the CRs is to constrain the process' normalisation using
the observed data, we do not want the systematic variations to directly
impact the processes' normalisations within the SRs.
Instead, the impact of the systematic variations is assessed via their effect
on the SM processes' acceptance in the CRs and SRs.
If a given systematic variation for a given SM process affects the process
in a coherent manner across both the CR and SR, then the resulting affect of the systematic variation
on the analysis should be reduced.
This will nearly be guaranteed if the phase spaces being probed by the CR and SR
are similar.
The more dissimilar the phase space being probed by the CR and SR, the larger
the expected impact of a given systematic variation due to the larger kinematic
extrapolation required.
This can conceptually be seen by considering Equation~\ref{eq:cr_tf}:
\begin{align}
    N_{p}^{\text{SR}} &= \mu_p \times N_{p,\,\text{MC}}^{\text{SR}} \nonumber \\
        &= \left( \frac{N_{p, \text{data}}^{\text{CR}}}{N_{p,\,\text{MC}}^{\text{CR}}} \right) \times N_{p,\,\text{MC}}^{\text{SR}} \nonumber \\
        &= N_{p, \text{data}}^{\text{CR}} \times \left( \frac{ N_{p,\,\text{MC}}^{\text{SR}}  }{ N_{p,\,\text{MC}}^{\text{CR}} } \right) \label{eq:cr_tf} \\
        &= N_{p, \text{data}}^{\text{CR}} \times \underbrace{\tau_p}_{\substack{\text{Transfer} \\ \text{ Factor}}} \nonumber,
\end{align}
where `$p$' is the process for which the CR and normalisation factor are defined, $\mu_p$
is the CR-derived normalisation factor (c.f. Equation~\ref{eq:mu_fac}),
$N_{p, \text{data}}^{\text{CR}}$ is the observed data in the CR with the MC simulation
for all processes that are not process $p$ subtracted, and $N_{p,\,\text{MC}}^{\text{SR}}$
is the MC-based SR prediction of process $p$.
The quantity $\tau_p$ is the process' TF that extrapolates the observed data in the CR
to the SR.
It can be seen that if the MC simulation response for a given process for a given systematic variation is the same
across both the CR and SR, that the TF will be unchanged as a result of the systematic
variation and therefore the predicted contribution of this process in the SR will
be unaffected by the systematic uncertainty.
If kinematics differ across the CR and SR, the acceptance for a given process may vary
in going from the CR and SR (or vice versa) and therefore such a cancellation is not likely to
occur due to the larger extrapolation required.

It can be seen, then, that for processes whose SR normalisation is derived in dedicated CRs,
that the TF appearing in Equation~\ref{eq:cr_tf} quantifies the impact of acceptance
variations between the CR and SR.
Systematic variations of the SM backgrounds with dedicated CRs, then, are quantified
by their impact on the resulting TF values.
This is contrary to assessing their impact by measuring the change in a process' SR prediction
by simply comparing the SR predictions before and after a given systematic variation is applied.
The uncertainties ascribed to SM processes via the TF Method are computed as follows,
\begin{align}
    \Delta \tau = \frac{ \lvert \tau_{\,\text{nominal}} - \tau_{\,\text{variation}} \rvert} { \tau_{\,\text{nominal}} },
    \label{eq:tf_uncert}
\end{align}
where $\tau_{\,\text{nominal}}$ ($\tau_{\,\text{variation}}$) is the TF computed
using the nominal (systematically varied) prediction of the process.
The quantity $\Delta \tau$ is then taken as a fractional uncertainty on the corresponding
process' SR predicted yield.

Sources of uncertainty stated as following the TF approach in Table~\ref{tab:syst_summary} are
quantified following Equation~\ref{eq:tf_uncert}.
The others are assessed simply by taking the impact of the variation on the process'
predicted yields in each of the regions appearing in the analyses.

%%%%%%%%%%%%%%%%%%%%%%%%%%%%%%%%%%%%%%%%%%%%%%%%%%%%%%%%%%%%%%%%%%%
%%%%%%%%%%%%%%%%%%%%%%%%%%%%%%%%%%%%%%%%%%%%%%%%%%%%%%%%%%%%%%%%%%%
%
% BACKGROUND MODELLING
%
%%%%%%%%%%%%%%%%%%%%%%%%%%%%%%%%%%%%%%%%%%%%%%%%%%%%%%%%%%%%%%%%%%%
%%%%%%%%%%%%%%%%%%%%%%%%%%%%%%%%%%%%%%%%%%%%%%%%%%%%%%%%%%%%%%%%%%%

\subsection{Background Modelling Uncertainties}
\label{sec:syst_bkg_modelling}

Uncertainties in the modelling of specific processes, SM or otherwise, are typically
assessed by comparing the nominal MC simulation for the processes in question to
that of an MC simulation with certain theoretical or phenomenological parameters varied.
In this way, one can assess the impact of the underlying assumptions made
in the MC simulation on the analyses' final results.

\subsubsection{Top-quark Pair and Single-top $Wt$ Production}
The production of SM top-quark pairs, \ttbar, is by far the most dominant SM background
in both of the analyses described in Chapters~\ref{chap:search_stop} and \ref{chap:search_hh}.
Some of the largest uncertainties in both analyses described therein are related to the modelling
of the \ttbar~process.
The process in which a single top-quark is produced in association with a $W$-boson plays a large
role in the analysis presented in Chapter~\ref{chap:search_hh}.
The MC simulation of these two processes is done using the same MC generation steps and, as a result,
the methods by which their systematic evaluation is performed are the same.
Here we describe the systematic variations used in the analyses for both of these processes.

Variation of the hard-scatter generation is performed by comparing the \ttbar~and \wt~samples produced
using \textsc{Powheg} for the matrix element generation to that using \textsc{aMC@NLO}, but keeping
the showering and fragmentation model the same in both (\textsc{Pythia8}).
The effects of the choice of fragmentation and hadronization model is assessed by comparing
the use of \textsc{Pythia8}, used in the nominal \ttbar~and \wt~predictions, to that of \textsc{Herwig}, all the while
keeping \textsc{Powheg} for the hard-scatter generation in both cases.
The description of the ISR and FSR provided by the hard-scatter generation is ascribed an
uncertainty by varying the underlying parameters that describe the characteristic energy
scales at which additional radiation (beyond that described by the matrix-element hard-scatter)
is produced.
In \textsc{Powheg}, this is controlled primarily by the \texttt{hdamp} parameter, and to assess
the impact of this parameter's value on the analyses it is varied up and down by a factor of 2 relative
to the nominal scenario.
The impact of the choice of PDF used in the hard-scatter generation is assessed by varying the
choice of PDF from the nominal, which is that of the NNPDF collaboration~\cite{Ball:2014uwa},
to that of the MMHT~\cite{Harland-Lang:2014zoa}, CT14~\cite{Lai:2010vv}, and PDF4LHC~\cite{Butterworth:2015oua} PDF sets and taking
the envelope of the variations.
The PDF error set, comprised of 100 separate components and provided by the NNPDF collaboration, is used to assign an additional uncertainty
on the nominal PDF used.
To assess the impact of the finite order in QCD at which the MC simulation is taken, and
sensitivity to missing higher-order terms, the factorization and renormalization scales,
$\mu_F$ and $\mu_R$, are varied in all pairings possible in which either is varied by a factor of two
up or down.
The resulting \textit{scale uncertainties} are derived by taking the envelope of the resulting variations.

Only in the analysis presented in Chapter~\ref{chap:search_hh} is the uncertainty related
to the theoretical prediction of the \ttbar~and \wt~processes taken into account.
The former is taken to be $\pm 5.82\%$ and the latter $\pm 5.32$\%~\cite{Czakon:2013goa,ATLAS-CONF-2013-102,Kidonakis:2010ux}.
The analysis described in Chapter~\ref{chap:search_hh} considers the sum of the \ttbar~and \wt~processes
as a single background and constrains the normalisation of their sum using a dedicated CR.
The uncertainties on these processes' cross-section, therefore, are taken into account to allow
for the \textit{composition} of the combined estimate ($\ttbar+\wt$), as opposed to its normalisation, to vary within the theoretical uncertainties.

An additional uncertainty, considered in both analyses presented in the current thesis, is related
to the non-trivial quantum interference between the NLO predictions of the \ttbar~and $\wt+b$~processes relevant
to both analyses.
This uncertainty is assessed by comparing the estimates of the \wt~process simulated
under the so-called Diagram Removal (DR) and Diagram Subtraction (DS) schemes that
are used in the NLO calculation of the single-top \wt~process.
The DR and DS schemes, and their comparison as a means of assessing a systematic uncertainty
as just described, are fully described in Refs.~\cite{Frixione:2008yi,ATL-PHYS-PUB-2016-004}.
This \textit{interference uncertainty} is negligible in the analysis presented in Chapter~\ref{chap:search_stop},
given the relatively small contamination of the \wt~process.
However, it is an important uncertainty in the analysis presented in Chapter~\ref{chap:search_hh}.
In this latter analysis, the \wt~background contributes at a rate equal to that of the \ttbar~process, making
the assessment of their interference a subtle one that will be described in Chapter~\ref{chap:search_hh}.

\subsubsection{$Z$-boson Production in Association with Heavy-Flavor Jets}

Processes involving the production of a SM $Z$-boson in association with
two or more heavy-flavor jets are important for the analysis presented in Chapter~\ref{chap:search_hh}.
These processes are referred to simply as `\zhf'.

The PDF and scale uncertainties on the \zhf~process are computed in exactly the
same fashion as for the \ttbar~and \wt~processes described in the previous section.
Additional uncertainties on the parton shower merging scales are assessed by varying
the CKKW merging scales~\cite{Lonnblad:2012ix} used by the \textsc{Sherpa} MC generator
used for the \zhf~simulation.
To assess the impact of different MC hard-scatter generation and fragmentation models, the 
nominal sample produced using \textsc{Sherpa} is compared to a \zhf~simulation produced using
\textsc{Madgraph+aMC@NLO} for the hard-scatter generation.
This last source of uncertainty is important for the analysis described in Chapter~\ref{chap:search_hh}
and amounts to an uncertainty of $\pm 15$\% on the \zhf~background estimate in the analysis' SRs.
The other sources of uncertainty on the \zhf~background are minor in comparison.

\subsubsection{Diboson Production}

The SM production of boson pairs --- diboson production (`$VV$') --- plays an important role only in the
analysis described in Chapter~\ref{chap:search_stop}.
The nominal $VV$ background estimate is simulated using the \textsc{Sherpa} MC generator.
Uncertainties arising as a result of variations in the factorization and renormalization scales,
PDF variations, and uncertainties in the CKKW merging scales are considered.
The methods by which these uncertainties are quantified are described above.
These systematic uncertainties amount to about a $\pm 9$\% uncertainty in the SR prediction
in the analysis described in Chapter~\ref{chap:search_stop}, but are generally smaller than
the modelling uncertainties on the \ttbar~background process.

%%%%%%%%%%%%%%%%%%%%%%%%%%%%%%%%%%%%%%%%%%%%%%%%%%%%%%%%%%%%%%%%%%%
%%%%%%%%%%%%%%%%%%%%%%%%%%%%%%%%%%%%%%%%%%%%%%%%%%%%%%%%%%%%%%%%%%%
%
% SIGNAL MODELLING
%
%%%%%%%%%%%%%%%%%%%%%%%%%%%%%%%%%%%%%%%%%%%%%%%%%%%%%%%%%%%%%%%%%%%
%%%%%%%%%%%%%%%%%%%%%%%%%%%%%%%%%%%%%%%%%%%%%%%%%%%%%%%%%%%%%%%%%%%

\subsection{Signal Modelling Uncertainties}
\label{sec:syst_sig_modelling}

Uncertainties on the signal modelling are assessed in both analyses presented in Chapters~\ref{chap:search_stop}
and \ref{chap:search_hh}.
The methods by which the uncertainties arising due to variations in the $\mu_F$ and $\mu_R$ scales, as
well as due to variations in the PDF choice, are assessed in exactly the same manner
as for the SM backgrounds described in Section~\ref{sec:syst_bkg_modelling}.
In both searches, these uncertainties on the signal are negligible.
For the analysis described in Chapter~\ref{chap:search_stop}, the uncertainties
arising as a result of varying the fragmentation model are found to be negligible for the
phase space probed by the SUSY parameter space, described by the masses of the supersymmetric
top-quark and LSP, in which that search is performed.
In the analysis described in Chapter~\ref{chap:search_hh}, uncertainties arising as a result of
the choice in fragmentation model are assessed by varying the choice of parton shower generator
from the nominal \textsc{Herwig} to an alternative description provided by \textsc{Pythia}
and have little impact on the analysis' final result.
The uncertainties on the theoretical cross-sections on the signal models appearing in the two analyses
are generally small and are found to not have a large impact on either of the analyses' final results.



%%%%%%%%%%%%%%%%%%%%%%%%%%%%%%%%%%%%%%%%%%%%%%%%%%%%%%%%%%%%%%%%%%%%%%
%%%%%%%%%%%%%%%%%%%%%%%%%%%%%%%%%%%%%%%%%%%%%%%%%%%%%%%%%%%%%%%%%%%%%%
%
% STATISTICS AND HYPOTHESIS TESTING
%
%%%%%%%%%%%%%%%%%%%%%%%%%%%%%%%%%%%%%%%%%%%%%%%%%%%%%%%%%%%%%%%%%%%%%%
%%%%%%%%%%%%%%%%%%%%%%%%%%%%%%%%%%%%%%%%%%%%%%%%%%%%%%%%%%%%%%%%%%%%%%
\section{Hypothesis Testing and Statistics}
\label{sec:stat_hypo}

This section describes the statistical procedures used in the analyses to be
presented in Chapters~\ref{chap:search_stop} and \ref{chap:search_hh} that allow
for conclusions to be drawn about the compatibility of the observed data with theories of
BSM physics.
The statistical inference tools described are inherently Frequentist and
are, for the most part, the \textit{de facto} standard for physics analyses searching for evidence of BSM physics
at the large experiments at the LHC.
Their widespread adoption by the experiments at the LHC does not indicate the
philosophical merit of Frequentist inference methodology, but rather highlights the technically simple implementation
of Frequentist hypothesis testing that allows for physics analyses to not get
bogged down in some of the details associated with Bayesian analyses, computational
or otherwise.
Indeed, most people by default think and interact with the world around them
in a Bayesian manner.
Taking the path of least resistance, physicists have tended to opt for the simpler
implementation of reporting their results, which, at the end of the day,
tend to not lose out much in terms of the picture of the objective truth that they draw~\cite{CousinsBayes}.
Section~\ref{sec:hypo_test} will describe, in somewhat general terms, what
hypotheses tests are and the way in which they are performed in ATLAS.
Sections~\ref{sec:likelihood}-\ref{sec:asymptotics} describe the details by which the measurements and systematic
uncertainties of
an analysis are transcribed into the language of the hypothesis test described
in Section~\ref{sec:hypo_test} using a likelihood-based test statistic.


%%%%%%%%%%%%%%%%%%%%%%%%%%%%%%%%%%%%%%%%%%%%%%%%%%%%%%%%%%%%%%%%%%%
%%%%%%%%%%%%%%%%%%%%%%%%%%%%%%%%%%%%%%%%%%%%%%%%%%%%%%%%%%%%%%%%%%%
%
% HYPOTHESIS TESTING
%
%%%%%%%%%%%%%%%%%%%%%%%%%%%%%%%%%%%%%%%%%%%%%%%%%%%%%%%%%%%%%%%%%%%
%%%%%%%%%%%%%%%%%%%%%%%%%%%%%%%%%%%%%%%%%%%%%%%%%%%%%%%%%%%%%%%%%%%

\subsection{Hypothesis Testing and the \cls Construction}
\label{sec:hypo_test}

Hypothesis testing starts with the unambiguous formulation of the hypothesis being
tested.
In the search for evidence of BSM physics, there are two hypothesis pitted
against one another.
The first is the \textit{null hypothesis}, denoted $H_0$, which is the hypothesis
subject to the test and corresponds to the SM hypothesis. The null hypothesis is commonly
referred to simply as the background-only (B) hypothesis.
The second hypothesis is the \textit{alternate hypothesis}, denoted $H_1$, and corresponds
to the SM with the addition of the BSM physics process being sought out.
The hypothesis $H_1$ is commonly referred to as the signal-plus-background (S+B) hypothesis.
In both searches presented in Chapters~\ref{chap:search_stop} and \ref{chap:search_hh},
$H_0$ is taken to be the SM.
In the search presented in Chapter~\ref{chap:search_stop}, $H_1$ is taken to be a specific
instantiation of the MSSM (Section~\ref{sec:susy}), with specific masses of the
stop quark and LSP.
In the search presented in Chapter~\ref{chap:search_hh}, $H_1$ is taken to be the
non-resonant production of Higgs boson pairs.
In this latter case, the $H_1$ hypothesis is indeed a process predicted by the SM but
it is one that is not included in the $H_0$ hypothesis.

%%%%%%%%%%%%%%%%%%%%%%%%%%%%%%%%%%%%%%%%%%%%%%%%%%%%%%%%%%%%%%%%%%%
%%%%%%%%%%%%%%%%%%%%%%%%%%%%%%%%%%%%%%%%%%%%%%%%%%%%%%%%%%%%%%%%%%%
%
% TEST STATISTICS
%
%%%%%%%%%%%%%%%%%%%%%%%%%%%%%%%%%%%%%%%%%%%%%%%%%%%%%%%%%%%%%%%%%%%
%%%%%%%%%%%%%%%%%%%%%%%%%%%%%%%%%%%%%%%%%%%%%%%%%%%%%%%%%%%%%%%%%%%
\subsubsection{The Test Statistic and $p$-Values}

In order to perform a hypothesis test in the Frequentist arena, a \textit{test statistic}, $q(x)$,
is defined.
A test statistic is defined using the analysis' measurements $x$ alone and is
used in order to define metrics by which the observed data is said to agree with
one of the two hypothesis, either $H_0$ or $H_1$.
In Section~\ref{sec:likelihood}, the exact form of the likelihood used in modern LHC experiments,
and that used in the analyses discussed in Chapters~\ref{chap:search_stop} and \ref{chap:search_hh},
will be introduced.
Here we will discuss general features of Frequentist test statistics and introduce
some of the language that will be used later on when discussing the results of the
analyses.

The conclusions eventually drawn about a given hypothesis are based on the observed value
of $q(x)$ and where this value lies in relation to the pre-defined \textit{critical region}.
The critical region is defined by a cut value, $q_c$, on the distribution of $q(x)$ under a specified hypothesis.
In the one-sided tests to be considered in the present thesis, $H_1$
will tend to have larger values of $q(x)$ as compared to $H_0$.
The critical region defines two important parameters associated with the hypothesis test.
The first is the quantity $\alpha$, which is referred to as the \textit{significance level},
and is defined as follows,
\begin{align}
    \int\limits_{q_c}^{+\infty} \, f(q | H_0) \, \mathrm{d}q = \alpha,
    \label{eq:sig_level}
\end{align}
where $f(q|H_0)$ is the probability distribution for the test statistic under
the background-only hypothesis.
The quantity $\alpha$ reports the probability for the background-only hypothesis (the SM) to be rejected when it is actually
true. This is commonly referred to as the Type I error rate.
The second quantity is $\beta$ and is defined as,
\begin{align}
    \int\limits_{-\infty}^{q_c} \, f(q|H_1) \, \mathrm{d}q = \beta,
    \label{eq:power_level}
\end{align}
where $f(q|H_1)$ is the probability distribution for the test statistic under
the signal-plus-background hypothesis.
The quantity $\beta$ gives the probability to reject the signal-plus-background hypothesis
when it is actually true. This is commonly referred to as the Type II error rate.
The quantity $(1-\beta)$ is referred to as the \textit{power of the test}.
The better a given physics analysis is at being able to discriminate between the signal
and background, i.e. to have clear separation between the $H_0$ and $H_1$ hypotheses,
the smaller (larger) is $\beta$ (the power of the test).

For simplicity, the two hypotheses $H_0$ and $H_1$ can be generalised by introducing a so-called
`signal strength' parameter, $\mu$, which acts as a multiplicative factor on the signal cross-section
appearing in $H_1$.
The hypothesis $H_0$, then, corresponds to the case $\mu = 0$ and that of $H_1$ corresponds to
$\mu = 1$.
With this general notation, then, the test statistic under either hypothesis is labelled as $q_{\mu}$.

Once a test statistic is specified, and its expected distribution under a given hypothesis is obtained,
$p$-values can be defined in order to compute the probability that the observed data originates from the
considered hypothesis (value of $\mu$).
They are computed as follows,
\begin{align}
    p_{\mu} = \int\limits_{q_{\mu, \text{obs}}}^{+\infty} \, f(q_{\mu} | \mu) \, \mathrm{d}q_{\mu},
    \label{eq:test_stat_pvalue}
\end{align}
where $q_{\mu, \text{obs}}$ is the observed value of the test statistic in data and $f(q_{\mu} | \mu)$ is the probability
density function of $q_{\mu}$ assuming hypothesis $\mu$.
A particular case of Equation~\ref{eq:test_stat_pvalue} is that of $p_0$, which quantifies the agreement of the data with the background-only
hypothesis ($\mu = 0$).
The $p_{\mu}$-value associated with a given hypothesis ($\mu$-value) is typically converted into the equivalent corresponding Gaussian significance, $Z$, defined
as the number of standard deviations that correspond to an upper-tail probability of $p_{\mu}$.
This is illustrated in Figure~\ref{fig:pval_sig}.

As the value of $p_{\mu}$ gets smaller, the confidence that the assumed hypothesis (value of $\mu$) is true
decreases.
At a certain point, it becomes acceptable to say that the assumed hypothesis is incompatible with
reality and the hypothesis described by the particular value of $\mu$ is said to be \textit{excluded}.
In the particle physics community, the conventional threshold to take for the value of $p_{\mu}$
at which point a hypothesis is said to be excluded is $p_{\mu} = 0.05$, corresponding to $Z=1.64$ as
illustrated in Figure~\ref{fig:pval_sig}.
This choice of the $p_{\mu}$-value at which point exclusion is said to occur defines
the critical region, described above, of the test.
The value of $0.05$ corresponds to the significance level of the test (c.f. Equation~\ref{eq:sig_level}), and is referred
to a hypothesis test being performed at the $95\%$ confidence level (CL) (i.e. CL $\equiv (1-\alpha)$)

In order to claim that new physics has been seen, the null hypothesis ($\mu = 0$) must be rejected.
The thresholds at which new physics can be said to have been observed and discovered are
much more stringent than that used for the exclusion of a specified hypothesis.
Incompatibilities with the null-hypothesis at the level of $p_0 = 1.3 \times 10^{-3}$ and
$p_0 = 2.9\times 10^{-7}$ are required in order to state that observation and discovery, respectively,
of new phenomena has occurred.
These thresholds, illustrated in Figure~\ref{fig:pval_sig}, for claiming observation and discovery are the fabled `$3\sigma$' and `$5\sigma$'
$p_0$-value criterion adopted by the particle physics community.

\begin{figure}[!htb]
    \begin{center}
        \includegraphics[width=0.48\textwidth]{figures/common_ana/stat_hypo/pval_sig_lin}
        \includegraphics[width=0.48\textwidth]{figures/common_ana/stat_hypo/pval_sig_log}
        \caption{
            Gaussian tail significance levels corresponding to specific $p$-values.
            The significance level of $Z=1.64\sigma$ corresponds to $p = 0.05$,
            that of $Z=3\sigma$ to $p = 1.3 \times 10^{-3}$, and that of
            $Z = 5\sigma$ to $p = 2.9\times 10^{-7}$.
            The area under the tail to the right of each indicated significance level corresponds to
            the associated $p$-value.
            \textbf{\textit{Left}}: Linear $y$-scale. \textbf{\textit{Right}}: Logarithmic $y$-scale.
        }
        \label{fig:pval_sig}
    \end{center}
\end{figure}

%%%%%%%%%%%%%%%%%%%%%%%%%%%%%%%%%%%%%%%%%%%%%%%%%%%%%%%%%%%%%%%%%%%
%%%%%%%%%%%%%%%%%%%%%%%%%%%%%%%%%%%%%%%%%%%%%%%%%%%%%%%%%%%%%%%%%%%
%
% THE CLS METHOD
%
%%%%%%%%%%%%%%%%%%%%%%%%%%%%%%%%%%%%%%%%%%%%%%%%%%%%%%%%%%%%%%%%%%%
%%%%%%%%%%%%%%%%%%%%%%%%%%%%%%%%%%%%%%%%%%%%%%%%%%%%%%%%%%%%%%%%%%%

\subsubsection{The \cls Construction}
\label{sec:cls_method}

In searches for new physics, the statement that a given signal hypothesis has been excluded
is an important one.
Once made by the LHC experiments, the specific signal model is essentially considered
no longer important to be searched for.
Therefore, the metrics by which the experiments make claims of exclusion have surrounding them
a wide-ranging literature discussing the merits and drawbacks of the many such metrics
that have been proposed over the years.
The bare $p_{\mu}$-value, for example, extracted from the observed data is subject to statistical fluctuations
and it can lead to unphysical exclusions when a downward fluctuation in the observed
number of events occurs.
This could lead to a premature exclusion of a broad region of new physics
that would perhaps no longer be looked into by future analyses or experiments.

The standard metric used by the LHC experiments today is known as `\cls'~\cite{CLSReadI,CLSReadII},
and is constructed in such a way as to reduce the likelihood of excluding signal
hypotheses that a search is not a-priori sensitive to.
The \cls metric is given by,
\begin{align}
    \text{CL}_s = \frac{p_{\mu}}{1-p_0},
    \label{eq:cls_def}
\end{align}
where the quantities $p_{\mu}$ and $p_0$ quantify the compatibilities between the data and the signal-plus-background
and background-only hypotheses, respectively.
Downward fluctuations in data, as those described above, will lead to larger values of $p_0$; thus
leading to larger values of \cls that avoid premature exclusion.

At the LHC, the \cls metric is used primarily for performing hypothesis tests aimed at claiming exclusion.
The standard null-hypothesis $p_0$-value is still used for claiming observation and discovery, as described above.
A given signal hypothesis with $\mu = 1$ is considered excluded at 95\% CL when $\cls \le 0.05$.
Note that this prescription for exclusion, $\cls \le \alpha$, is generally a stronger requirement than the
standard prescription, $p_{\mu} \le \alpha$.
The \cls metric is also used to compute \textit{upper limits}.
A 95\% CL upper limit on a given signal hypothesis specified by $\mu$
is the largest value of $\mu$ satisfying $\cls \ge 0.05$.
The interpretation being that this corresponds to the largest possible signal cross-section
that is unable to be excluded and therefore smaller values of $\mu$, corresponding
to smaller signal cross-sections, are still consistent with the observed data and cannot therefore
be excluded.
The process of scanning $\mu$ hypotheses and computing the \cls in order to find
an upper limit on $\mu$ is illustrated in Figure~\ref{fig:upper_limit_scan_cartoon}.

\begin{figure}[!htb]
    \begin{center}
        \includegraphics[width=0.5\textwidth]{figures/common_ana/stat_hypo/upper_limit_scan_examplePDF}
        \caption{
            An upper limit scan on the signal strength parameter $\mu$ associated with a signal hypothesis.
            The \cls, given by Equation~\ref{eq:cls_def}, is recomputed for a range of $\mu$ values
            describing a given signal hypothesis.
            This is shown by the blue line.
            The $\mu$ value at which the \cls curve crosses the line $\cls = 0.05$, $\mu^{\text{UL}}$, is the
            upper limit on $\mu$ for the signal hypothesis.
            Values of $\mu$ smaller than $\mu^{\text{UL}}$ remain compatible with the observed data,
            while those values greater than $\mu^{\text{UL}}$ are excluded at $95\%$ CL.
        }
        \label{fig:upper_limit_scan_cartoon}
    \end{center}
\end{figure}

%%%%%%%%%%%%%%%%%%%%%%%%%%%%%%%%%%%%%%%%%%%%%%%%%%%%%%%%%%%%%%%%%%%
%%%%%%%%%%%%%%%%%%%%%%%%%%%%%%%%%%%%%%%%%%%%%%%%%%%%%%%%%%%%%%%%%%%
%
% PROFILE LIKELIHOOD
%
%%%%%%%%%%%%%%%%%%%%%%%%%%%%%%%%%%%%%%%%%%%%%%%%%%%%%%%%%%%%%%%%%%%
%%%%%%%%%%%%%%%%%%%%%%%%%%%%%%%%%%%%%%%%%%%%%%%%%%%%%%%%%%%%%%%%%%%

\subsection{The Profile Likelihood Ratio Test Statistic}
\label{sec:likelihood}

The test statistic associated with many of the LHC experiments, including ATLAS,
and the one used in the analyses to be presented in Chapters~\ref{chap:search_stop} and
\ref{chap:search_hh} is based on a likelihood ratio.
The construction of the test statistic is described in this section.

In the analyses to be presented, so-called `counting experiments' are performed wherein
only the numbers of events, from data and the predicted background and signal,
are used as input.
These numbers are taken from all relevant regions in the analysis: the CRs and the SRs.
The expected number of events populating each region is given by the following:

\begin{align}
    N_r^\text{exp}(\mu_{\text{sig}}, \bm{\mu_{\text{bkg}}}, \bm{\theta}) = \mu_{\text{sig}} \cdot N_{r,\,\text{sig}}^{\text{exp}}(\bm{\theta}) + \sum\limits_{b\,\in \,\text{bkg}} \, \mu_b \cdot N_{r,\,b}^{\text{exp}}(\bm{\theta}),
    \label{eq:test_stat_n_r}
\end{align}
where $\bm{\theta}$ are a set of fit nuisance parameters (NP) associated with the systematic
uncertainties described in Section~\ref{sec:common_systematics},
$\bm{\mu_{\text{bkg}}}$ are normalisation factors associated with the background processes (indexed by `$b$'),
$\mu_{\text{sig}}$ is the signal-strength modifier associated with the signal hypothesis,
$N_{r,\,\text{sig}}^{\text{exp}}$ is the predicted signal yield in region $r$,
and $N_{r,\,b}^{\text{exp}}$ is the predicted background yield in region $r$ for process $b$.
The predicted number of events for each process, signal or background, depend on the $\bm{\theta}$ parameter vector
since systematic variations can adjust the overall normalisation of a given process or adjust
the acceptance of a given process in the phase space probed by the regions indexed by $r$.
Both of these effects result in a change in the predicted rate of a process in a given region.

The observed data yield in each region of the analysis is expected to obey Poisson statistics.
Therefore, the likelihood function $L(\mu_{\text{sig}}, \bm{\mu_{\text{bkg}}}, \bm{\theta})$ is
constructed as a product of Poisson probability terms:
\begin{align}
    L_0(\mu_{\text{sig}}, \bm{\mu_{\text{bkg}}}, \bm{\theta}) = \prod\limits_{r\,\in\,\text{regions}} \, 
            \frac
            {
                \left[N_r^{\text{exp}} ( \mu_{\text{sig}}, \bm{\mu_{\text{bkg}}}, \bm{\theta}) \right] ^ {N_r^{\text{obs}}}
            }
            {
                N_r^{\text{obs}}!
            }
            \cdot
            \exp\left[ N_r^{\text{exp}} ( \mu_{\text{sig}}, \bm{\mu_{\text{bkg}}}, \bm{\theta}) \right],
    \label{eq:likelihood_main}
\end{align}
where $N_r^{\text{obs}}$ is the observed data yield in region $r$.

It is standard practice to parametrize the systematic uncertainties associated with the measurements
of the $N_r^{\text{exp}}$ in such a way that $\bm{\theta} = 0$ (all of the $\theta_i$ equal to zero) corresponds to the central (nominal) value
of the set of parameters associated with the uncertainty (e.g. the nominal value of JES).
The values $\theta_i = \pm 1$ represent shifts in the parameter values by their $\pm 1\sigma$
variation, as defined by the systematic uncertainty (e.g. systematic shifts in the value of the JES).
The systematic uncertainties described in Section~\ref{sec:common_systematics} are typically
computed such that, at the level of performing a physics analysis, only the $\pm 1 \sigma$ shifts about the central value
in the associated parameter are known.
Means of interpolation and extrapolation, needed to obtain a smoothly varying response in the
numbers of predicted events as the $\theta$ parameters vary continuously between their $\pm 1 \sigma$ values,
are provided by the \textsc{HistFactory} toolkit~\cite{HistFactory}.
The effects of the NP associated with each of the analysis' systematic uncertainties
are implemented as \textit{constraint terms} in the likelihood appearing in Equation~\ref{eq:likelihood_main}.
These terms are typically implemented as Gaussians centered on zero and with a width of one:
\begin{align}
    L(\mu_{\text{sig}}, \bm{\mu_{\text{bkg}}}, \bm{\theta}) = L_0(\mu_{\text{sig}}, \bm{\mu_{\text{bkg}}}, \bm{\theta})
        \, \cdot \, \underbrace{\prod\limits_{i = 1}^{n} \frac{1}{\sqrt{2 \pi}} \, \exp \left( - \frac{\theta_i^2}{2} \right)}_{\text{Constraints}}.
    \label{eq:full_likelihood}
\end{align}
The best estimates (maximum likelihood estimates, MLE) for the parameters ($\mu_{\text{sig}}$, $\bm{\mu_{\text{bkg}}}$, $\bm{\theta}$)
are determined in the fit to the observed data, $\bm{N^{\text{obs}}}$, via the maximization
of the likelihood function given by Equation~\ref{eq:full_likelihood}. 
Typically, and equivalently, the negative log likelihood, $-\ln L$, is minimized.
Technically, the minimization of Equation~\ref{eq:full_likelihood} in the analyses to be presented
in Chapters~\ref{chap:search_stop} and \ref{chap:search_hh} is done using \textsc{Miniuit}
via an interface provided by \textsc{RooFit}~\cite{MINUIT,RooFitI}.
The values of all parameters prior to (after) the minimization procedure are referred to as the
`pre-fit' (`post-fit') values.
The pre-fit values for the $\mu_{\text{bkg}}$ parameters are set to 1 and the $\theta_i$ are set to $0$, corresponding
to the central values of the Gaussian-shifted parameters associated with the systematic uncertainties.

The test statistic defined for the analyses to be presented is based on the following
\textit{profile likelihood ratio},
\begin{align}
    \lambda(\mu_{\text{sig}}) \equiv
        \frac{
            L(\mu_{\text{sig}}, \hat{\hat{\bm{\mu}}}_{\text{bkg}}, \hat{\hat{\bm{\theta}}})
        }
        {
            L(\hat{\mu}_{\text{sig}}, \hat{\bm{\mu}}_{\text{bkg}}, \hat{\bm{\theta}})
        },
    \label{eq:likelihood_ratio}
\end{align}
where in the numerator the parameters $(\bm{\mu}_{\text{bkg}}, \bm{\theta})$ are fit to their MLE values
for a given value of the assumed value of the signal-strength parameter $\mu_{\text{sig}}$.
In the denominator, the full set of parameters ($\mu_{\text{sig}}$, $\bm{\mu_{\text{bkg}}}$, $\bm{\theta}$) are
equal to their MLE values.
The likelihood ratio defined in Equation~\ref{eq:likelihood_ratio} is used to define the
final test statistic relevant to the analyses,
\begin{align}
    q_{\mu_{\text{sig}}} = - 2 \ln \lambda (\mu_{\text{sig}}),
    \label{eq:pll_test_stat}
\end{align}
which is used to perform the hypothesis tests described in Section~\ref{sec:hypo_test}.

\subsubsection{Profile Likelihood Fit Determination of Background Constraints}

We note here that the quantities described by $\bm{\mu_{\text{bkg}}}$ in Equations~\ref{eq:test_stat_n_r}--\ref{eq:pll_test_stat} correspond to normalisation correction
factors for a given SM background process.
In the analyses to be presented in Chapters~\ref{chap:search_stop} and \ref{chap:search_hh},
for the background processes for which a dedicated CR is not defined, the associated $\mu_b$ are
set to 1 and are not allowed to vary during the fit (i.e. their pre-fit and post-fit values are equal).
For those processes for which a dedicated CR is defined, the associated $\mu_b$ parameters
are analogous to the normalisation correction factors described in Section~\ref{sec:control_region_method} and
are unconstrained in the fit procedure.
They are therefore said to `freely float' during the fit and 
their post-fit value does not necessarily correspond to their pre-fit one.
Given the low numbers of events typically expected in the SRs, and the relatively
large numbers and purities expected in the CRs, the post-fit values of the $\mu_b$ typically
correspond to those values expected from the simple computations provided by Equation~\ref{eq:mu_fac}
and/or Equation~\ref{eq:mu_fac_expand}.
An additional note is that in the signal-plus-background hypotheses with $\mu_{\text{sig}} \ne 0$, if
the CRs have a non-negligible contamination of signal events, however, this correspondence will generally not be true
since the post-fit values of the $\mu_b$ will depend on a given value of the $\mu_{\text{sig}}$ parameter.
The desire to construct a robust background-only model is one of the motivating factors, then,
in designing analyses in such a way as to minimize the signal contamination in the CRs.

\subsubsection{Profiling}
\label{sec:profiling}

The Gaussian constraint terms in Equation~\ref{eq:full_likelihood}, centered on 0 and with widths of 1,
are equivalent to transforming the $\bm{\theta}$ parameters associated with the systematic uncertainties as follows,
\begin{align}
    \bm{\theta}^{\prime} = \frac{\bm{\theta} - \bm{\hat{\theta}}}{\bm{\hat{\sigma}}},
    \label{eq:theta_gaus}
\end{align}
where the quantity $\bm{\sigma}$ corresponds to the initial widths of the $\pm 1 \sigma$ variations
associated with the systematic variation represented by $\bm{\theta}$.
Equation~\ref{eq:theta_gaus} normalises all NP to facilitate the easy comparison of their pre- and post-fit values and uncertainties.
For a given NP, then, a post-fit value and uncertainty near 0 and 1, respectively, indicate
that the data was not able to adjust the NP.
Changes in the post-fit NP values (uncertainties) are referred to as NP `pulling' (`profiling').
NP values may be pulled far from zero in order to maximize the overall agreement of the background prediction
with data during the fit procedure.
NP may be profiled in such a way that the post-fit uncertainties on the NP are smaller than
the initial estimate of the uncertainty provided by auxiliary measurements and data.
Such a case indicates that the initial prior on the impact of the systematic uncertainty was too
large and that the uncertainties may be reduced so as to be more compatible  with the range allowed by
the data statistics relevant for the analysis at hand.
This potential for the profiling mechanism to reduce the overall impact of systematic uncertainties
on an analysis' results is seen as a general benefit of the profile likelihood prescription described
by Equations~\ref{eq:likelihood_ratio} and \ref{eq:pll_test_stat}.
The interpretation of this is that the profiling mechanism allows for the impact of the systematic uncertainties
to be more accurately characterised in the phase space in which they are being applied, as opposed
to that of the auxiliary measurements in which they are initially derived.


\subsection{Toy Examples of the Profile Likelihood Fit}
\label{sec:profile_examples}

The best way to get a feel for the profiling mechanisms just described is via a few simple
examples, to which we turn now.

\subsubsection{Profile Likelihood Fit Example 1a}
\label{sec:profiling_example_1a}

This example illustrates the mechanisms of both NP profiling and pulling.
We set up a dummy analysis, with two regions and two backgrounds only.
The regions are referred to as `Region 1' and `Region 2' and the backgrounds
are `Bkg 1' and `Bkg 2'.
A summary of the predicted and observed yields, as well as the uncertainties on
the predicted yields, is as follows:

\begin{itemize}
    \item \underline{\textbf{Region 1}}
        \begin{itemize}
            \item Background composition: 80 events from Bkg 1
            \item Observed data: 100 events
            \item Systematic Uncertainties:
            \begin{itemize}
                \item `Norm. Bkg. 1': A 50\% systematic uncertainty on the predicted yield of Bkg 1
            \end{itemize}
        \end{itemize}
    \item \underline{\textbf{Region 2}}
        \begin{itemize}
            \item Background composition: 100 events from Bkg 2
            \item Observed data: 100 events
            \item Systematic Uncertainties:
            \begin{itemize}
                \item `Norm. Bkg. 2$_1$': A 10\% systematic uncertainty on the predicted yield of Bkg 2
                \item `Norm. Bkg. 2$_2$': An additional 10\% systematic uncertainty on the predicted yield of Bkg 2
            \end{itemize}
        \end{itemize}
\end{itemize}

From this information, a likelihood as described by Equation~\ref{eq:full_likelihood} is constructed.
In this setup, all of the $\mu$ parameters are set to 1 for the backgrounds
and NP constraint terms parametrized by $\theta$ terms
for each of the three uncertainties are included.
The pre-fit situation is illustrated by the left side of Figure~\ref{fig:prof_ex_1_np50}.
A profile-likelihood fit to the observed data is performed and the results for the fitted parameters
are shown in Figure~\ref{fig:prof_ex_1_pulls} and by the
right side of Figure~\ref{fig:prof_ex_1_np50}.

Focusing first on Region 1, we see that after the fit the pre-fit 20 event excess is removed by
the increase in Bkg. 1's predicted yield.
This is explained by the NP on Bkg. 1 being pulled by nearly 0.5, as seen in Figure~\ref{fig:prof_ex_1_pulls}.
Using Equation~\ref{eq:theta_gaus}, a pull by 0.5 in this NP is precisely 20 events ($\theta^{\text{post-fit}} \times ( \text{pre-fit uncertainty} \times N^{\text{pre-fit}}) \rightarrow 0.5 \times (0.5 \times 80) = 20$).
The post-fit uncertainty has also been reduced to that allowed by the data statistics, which is driven
by the NP on Bkg. 1 being profiled from its initial uncertainty of $\pm 1$ to $\pm 0.25$ ($\Delta \theta^{\text{post-fit}} \times (\text{pre-fit uncertainty} \times N^{\text{pre-fit}}) \rightarrow 0.25 \times (0.5 \times 80) = 10$), also seen in Figure~\ref{fig:prof_ex_1_pulls}.

Looking to the post-fit results of Region 2, we see that, as with Region 1, the overall
background uncertainty has been profiled so as to be compatible with that allowed by the data statistics.
The main difference with respect to Region 1, however, is the fact that there are two NP constraints
on Bkg. 2.
Each of the NP constraint terms is already at the level of $10\%$ that corresponds to the data statistics.
There is therefore a redundancy in the NP constraints on Bkg. 2 and the fit develops a correlation between
them such that their combined post-fit impact on Bkg. 2 is as seen on the right side of Figure~\ref{fig:prof_ex_1_np50}.
The correlation between the two constraints is seen in Figure~\ref{fig:prof_ex_1_pulls} to be $\rho = -0.5$,
which, in this simple scenario, can be expected to be the case by the following:
\begin{align}
    \sigma_{1} &= \sigma_{2} = \sigma, \nonumber \\
    \sigma_{1 \oplus 2} &= \sqrt{ \sigma_1^2 + \sigma_2^2 + 2 \sigma_1 \sigma_2 \rho } = \sigma, \nonumber
\end{align}
where, in the second line, we expect $\sigma$ to be at the level of the data statistics.
This requires $\rho = -0.5$ and is indeed what the profile-likelihood fit ends up with.

\begin{figure}[!htb]
    \begin{center}
        \includegraphics[width=0.48\textwidth]{figures/common_ana/stat_hypo/profile_examples/profile_ex_1_NP50_pre}
        \includegraphics[width=0.48\textwidth]{figures/common_ana/stat_hypo/profile_examples/profile_ex_1_NP50_post}
        \caption{
            \textbf{\textit{Left}}: Pre-fit scenario for Example 1a, described in the text.
                The hatched areas indicate the uncertainty in the predicted yields.
            \textbf{\textit{Right}}: Post-fit scenario for Example 1a, described in the text.
                The hatched areas indicate the uncertainty in the predicted yields.
        }
        \label{fig:prof_ex_1_np50}
    \end{center}
\end{figure}

\begin{figure}[!htb]
    \begin{center}
        \includegraphics[width=0.48\textwidth]{figures/common_ana/stat_hypo/profile_examples/profile_ex_1_pulls}
        \includegraphics[width=0.48\textwidth]{figures/common_ana/stat_hypo/profile_examples/np_corr_ex_1}
        \caption{
            \textbf{\textit{Left}}: Post-fit values for the parameters entering the profile-likelihood fit
                described in Example 1a. The post-fit result for the NP `Norm. Bkg. 1' in pink corresponds
                to that of the fit configuration described in Example 1b, below.
                All other parameters (in black) correspond to those in the fit configuration described in Example 1a.
            \textbf{\textit{Right}}: Post-fit correlation matrix for the parameters entering the profile-likelihood
                fit described in Example 1a.
        }
        \label{fig:prof_ex_1_pulls}
    \end{center}
\end{figure}

\subsubsection{Profile Likelihood Fit Example 1b}
\label{sec:profiling_example_1b}

In the previous example we saw that, through the pulling of a constraint term,
the complete removal of a discrepancy between the observed data and the
predicted background was made possible.
This reinforces the idea that the predicted numbers of events are dependent
upon the NP entering the fit ($N^{\text{exp}} \rightarrow N^{\text{exp}}(\bm{\theta})$)
and that their post-fit values can change even if their is no explicit normalisation-correcting $\mu$ terms in the fit.
Of course, the large uncertainty on Bkg. 1's prediction in Example 1a equates to a loose constraint term and allows for a large degree of
flexibility in the fit for it to be pulled in such a way as to have the post-fit prediction perfectly line up with
the data being fit to.
In this current example we reproduce the fit configuration introduced in Example 1a, but instead the
uncertainty on Bkg. 1 in Region 1 is reduced to $10\%$.
The pre- and post-fit distributions of the observed and predicted events, with their uncertainties,
are shown in Figure~\ref{fig:prof_ex_1_np10}.
The results of the fit for Region 2 are exactly the same as in Example 1a since we have not altered
the parameters describing this region.

Given the tighter constraint on the predicted yield of Bkg. 1 in Region 1, as compared to Example 1a,
the NP describing Bkg. 1's normalisation uncertainty does not have enough freedom to be pulled to fully
cover the 20 event excess.
The post-fit value of the NP is shown in pink in Figure~\ref{fig:prof_ex_1_pulls}.
It is pulled to a value of 1.04, which corresponds to just above 8 events as seen in Figure~\ref{fig:prof_ex_1_np10}.
The smaller uncertainty on Bkg. 1's prediction in this example, as compared to Example 1a, corresponds
to a higher level of confidence in its pre-fit value.
The associated NP constraint term, therefore, should not be allowed the freedom to contradict this
high degree of confidence by shifting the prediction to the same extent as that in Example 1a.

\begin{figure}[!htb]
    \begin{center}
        \includegraphics[width=0.48\textwidth]{figures/common_ana/stat_hypo/profile_examples/profile_ex_1_NP10_pre}
        \includegraphics[width=0.48\textwidth]{figures/common_ana/stat_hypo/profile_examples/profile_ex_1_NP10_post}
        \caption{
            \textbf{\textit{Left}}: Pre-fit scenario for Example 1b, described in the text.
                The hatched areas indicate the uncertainty in the predicted yields.
            \textbf{\textit{Right}}: Post-fit scenario for Example 1b, described in the text.
                The hatched areas indicate the uncertainty in the predicted yields.
        }
        \label{fig:prof_ex_1_np10}
    \end{center}
\end{figure}

\subsubsection{Profile Likelihood Fit Example 2}
\label{sec:profiling_example_2}

In this example we again set up the same fit configuration as in Example 1a.
However, a signal process `Signal' with 20 events predicted in Region 1 is added.
The signal-strength parameter $\mu_{\text{sig}}$, acting as a freely floating normalisation
term for the signal process, is now included in the fit as well.
A profile-likelihood fit to the observed data is once again performed.
The pre- and post-fit distributions of the observed data and predicted backgrounds,
with their uncertainties, are shown in Figure~\ref{fig:prof_ex_2_pre}.

From the post-fit distributions shown in Figure~\ref{fig:prof_ex_2_pre}, we see the
expected result that the post-fit uncertainties on the predicted backgrounds
are such that they match the precision allowed by the data statistics.
As we have not changed anything in Region 2, relative to Example 1a and Example 1b,
we do not expect the post-fit values of the NP associated with Bkg. 2 to have changed relative
to those examples.
The post-fit value of the NP associated with Bkg. 1, however, differs quite substantially
relative to the earlier examples, as is seen in Figure~\ref{fig:prof_ex_2_pulls}.
We see that the NP term `Norm. Bkg. 1' is neither pulled nor profiled.
Instead, the unconstrained signal-strength parameter $\mu_{\text{sig}}$ is able to `eat up'
the remaining degrees of freedom needed for the reduction in the background prediction's uncertainty.
Since the pre-fit predicted yield of the signal process is such that the complete background
prediction in Region 1 matches the data, the post-fit value for $\mu_{\text{sig}}$ is equal to its
pre-fit value of 1.
It's uncertainty, however, is $\pm 1.90$ and corresponds to a post-fit uncertainty
on the signal process' prediction of nearly 40 events ($1.90 \times 20$).
This precision on the signal matches the pre-fit uncertainty of of Bkg. 1, with its predicted 80 events
and 50\% uncertainty.
This illustrates the fact that the $\mu_{\text{sig}}$ term is completely degenerate with the NP associated with Bkg. 1:
any variation in the NP term can be fully compensated by the unconstrained $\mu_{\text{sig}}$ term.
This compensation allows for the background precision in Region 1 to be reduced nearly to
that of the data statistics, as seen in Figure~\ref{fig:prof_ex_2_pre},
and is characterised by the nearly perfect anti-correlation between the $\mu_{\text{sig}}$ and
Bkg. 1 NP constraint terms, shown in the right side of Figure~\ref{fig:prof_ex_2_pulls}.
This ability for $\mu_{\text{sig}}$ to compensate the NP by preventing it from being pulled is a general
feature of the addition of unconstrained normalisation-scaling $\mu$ terms in these types of fits, regardless
of whether they are associated with signal or background processes.

\begin{figure}[!htb]
    \begin{center}
        \includegraphics[width=0.48\textwidth]{figures/common_ana/stat_hypo/profile_examples/profile_ex_2_pre}
        \includegraphics[width=0.48\textwidth]{figures/common_ana/stat_hypo/profile_examples/profile_ex_2_post}
        \caption{
            \textbf{\textit{Left}}: Pre-fit scenario for Example 2, described in the text. 
                The hatched areas indicate the uncertainty in the predicted yields only on the background processes.
            \textbf{\textit{Right}}: Post-fit scenario for Example 2, described in the text.
                The hatched areas indicate the uncertainty in the predicted yields only on the background processes.
        }
        \label{fig:prof_ex_2_pre}
    \end{center}
\end{figure}

\begin{figure}[!htb]
    \begin{center}
        \includegraphics[width=0.48\textwidth]{figures/common_ana/stat_hypo/profile_examples/profile_ex_2_pulls}
        \includegraphics[width=0.48\textwidth]{figures/common_ana/stat_hypo/profile_examples/np_corr_ex_2}
        \caption{
            \textbf{\textit{Left}}: Post-fit values for the parameters entering the profile-likelihood fit
                of Example 2,
                described in the text. 
            \textbf{\textit{Right}}: Post-fit correlation matrix for the parameters entering the profile-likelihood
                fit of Example 2, described in the text.
        }
        \label{fig:prof_ex_2_pulls}
    \end{center}
\end{figure}

It is interesting to make note of the fact, exemplified by the above discussion, that the precision on a given process' $\mu$ term,
if allowed to vary in the fit, is highly dependent on the precision of the prediction of the
other backgrounds.
This, of course, is intuitive and has meaningful consequences for searches for new physics, in which the signal-plus-background
hypotheses are characterised by $\mu_{\text{sig}}$ terms associated with the sought-for signal processes.
Large uncertainties in background predictions result in less precise statements about the presence
of signal in the analyses' regions.
This, then, reduces the statistical power (c.f. Equation~\ref{eq:power_level}) of the hypothesis tests
being performed and, if the uncertainties are large enough, prevent the analyses from making meaningful
statements about the compatibility of the signal-plus-background hypotheses with the observed data.

As an illustration, Figure~\ref{fig:cls_scan_uncert} shows the CL$_s$ computed as a function of varying signal-to-background ratio, for several background uncertainty hypotheses,
for a fixed number of predicted background and data events in a single-region analysis.
The number of signal events needed to cross into the critical CL$_s <0.05$ region,
at which point a signal hypothesis may be considered excluded at 95\% CL, is quite sensitive to the uncertainty
on the background prediction.
Again, this is intuitive mainly from the perspective that reduced levels of precision
in the background prediction correspond directly to increased levels of ambiguity 
as to the point at which a signal becomes apparent.

Figure~\ref{fig:cls_scan_uncert} also illustrates the increase in the upper limit
on a signal process' cross-section, as a result of increased background uncertainty,
in cases where no significant excess in data over the predicted background level is observed.
For example, imagine that the sought-for signal process has a predicted cross-section leading to an
$S/B$ value of 0.4 in Figure~\ref{fig:cls_scan_uncert}.
At a level of background uncertainty of $30\%$, with a CL$_s \approx 0.35$, this process can neither be said
to be excluded nor to actually exist.
At this background uncertainty, the analysis would find an upper-limit on the signal process' $\mu_{\text{sig}}$ term nearing a value of 2,
since the CL$_s$ crossing point is near $S/B = 0.8$.
Using this terminology, then, when no meaningful discrepancies (characterised by very low $p_0$-values) between the predicted background and observed event counts
are seen, the upper-limit value under the hypothesis for a
given signal process gives a measure of `how far' the analysis is from being sensitive to that process.
If the upper-limit value is $200$ times the predicted cross-section, the analysis is not sensitive to the process at all.
However, if the upper-limit value is $\mathcal{O}(1)$ times the process' predicted cross-section, the analysis
is entering the regime wherein it will be able to resolve the presence of the signal and can begin to make meaningful statements about
its likelihood of existence.
If an upper-limit value corresponds to a cross-section value that is \textit{less} than the predicted one,
the process --- as predicted --- is excluded.

\begin{figure}[!htb]
    \begin{center}
        \includegraphics[width=0.6\textwidth]{figures/common_ana/stat_hypo/cls_bkguncert_scan}
        \caption{
            Dependence of computed CL$_s$ on the signal-to-background ratio for a single-region
            analysis in which 20 events are predicted and 20 events are observed, shown
            for varying levels of uncertainty on the predicted background.
        }
        \label{fig:cls_scan_uncert}
    \end{center}
\end{figure}

\FloatBarrier
\subsection{Asymptotic Formulation of Likelihood Ratio Tests}
\label{sec:asymptotics}

In Section~\ref{sec:profile_examples} we have made use of the profile likelihood ratio fits and
even shown some examples of hypothesis testing.
We have not shown, however, how the probability density functions of the test statistics described
by Equation~\ref{eq:pll_test_stat} are obtained.
The probability density functions are needed in order to compute the $p_{\mu}$-values as indicated
by Equation~\ref{eq:test_stat_pvalue}.
There are two general approaches taken to obtain such distributions.
The first is by generating so-called `pseudo-experiments', in which the predictions of the hypotheses are sampled within their uncertainties many times,
with a value of the corresponding test statistic computed for each sample, such that a distribution
of the $q_{\mu}$ may be obtained.
This method can be computationally intensive if high-levels
of precision are required for the analysis.
The method used in the searches presented in Chapters~\ref{chap:search_stop} and \ref{chap:search_hh}
relies on asymptotic approximations of the sampling distributions of the 
profile-likelihood ratio test statistic of Equation~\ref{eq:pll_test_stat}.
The complete discussion and description of the asymptotic formulae used to represent
Equation~\ref{eq:pll_test_stat} is given in Ref.~\cite{AsymptoticFormula}.

The key takeaway from Ref.~\cite{AsymptoticFormula}, is that in the large $N$ limit, with $N$
corresponding to the data sample size, the sampling distributions of the profile likelihood test statistics
take the form of non-central chi-square distributions with single degrees of freedom,
\begin{align}
    f(q_{\mu} | \Lambda) = \frac{1}{2\sqrt{q_{\mu}}} \frac{1}{\sqrt{2\pi}} \left[ \exp \left( -\frac{1}{2} (\sqrt{q_{\mu}} + \sqrt{\Lambda})^2 \right) 
        + \exp \left( -\frac{1}{2}(\sqrt{q_{\mu}} - \sqrt{\Lambda} )^2 \right) \right],
    \label{eq:test_stat_asym}
\end{align}
where the non-centrality term, $\Lambda$, is given by the following:
\begin{align}
    \Lambda = \frac{ (\mu - \hat{\mu})^2 } {\sigma^2}.
    \label{eq:noncentrality}
\end{align}
Example distributions of the test statistic probability density function are shown in Figure~\ref{fig:qmu_dist},
using Equation~\ref{eq:test_stat_asym}, illustrating the differences between the background-only
and signal-plus-background hypotheses for a case in which there is a large predicted rate of signal
and for the case in which there is not.
As described by Equation~\ref{eq:test_stat_pvalue}, the $p_{\mu}$ and $p_0$ values needed, among other things, for the computation of \cls can be obtained from these distributions
for a given value of $q_{\mu}$ by integrating the areas to the right of the specified $q_{\mu}$ under the distributions associated
with the $S+B$ and $B$ hypotheses, respectively.

\begin{figure}[!htb]
    \begin{center}
        \includegraphics[width=0.48\textwidth]{figures/common_ana/stat_hypo/qmu_dist_large_s}
        \includegraphics[width=0.48\textwidth]{figures/common_ana/stat_hypo/qmu_dist_small_s}
        \caption{
            Probability densities of the profile likelihood test statistic of Equation~\ref{eq:pll_test_stat}
            obtained under the asymptotic formulae described in Ref.~\cite{AsymptoticFormula} and shown
            in Equation~\ref{eq:test_stat_asym}.
            The background-only hypothesis ($\mu = 0$) is shown in blue, and the signal-plus-background
            hypothesis ($\mu \ne 0$) is shown in red.
            \textit{\textbf{Left}}: The case in which the predicted amount of signal is large compared to the
                background rate.
            \textit{\textbf{Right}}:  The case in which the predicted signal rate is similar to that of the background.
        }
        \label{fig:qmu_dist}
    \end{center}
\end{figure}


