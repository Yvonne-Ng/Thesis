\section{Systematic Uncertainties}
\label{sec:common_systematics}

There are many sources of systematic uncertainty (`systematics') that affect the results of the analyses
to be presented in Chapters~\ref{chap:search_stop} and \ref{chap:search_hh}.
The considered sources of systematic uncertainty are listed in Table~\ref{tab:syst_summary}.
There are uncertainties related to the overall event (e.g. the uncertainty
on the luminosity measurement), on the reconstruction and identification of
the physics objects described in Section~\ref{chap:objects}, and in the MC
simulation of both the SM background and signal processes.
Brief descriptions of the sources of systematic uncertainty
appearing in Table~\ref{tab:syst_summary} are given in Sections~\ref{sec:syst_experimental}-\ref{sec:syst_sig_modelling}.

\begin{table}[!htb]
    \caption{
        Summary of the sources of systematic uncertainties affecting the measurements
        in the analyses discussed in Chapters~\ref{chap:search_stop} and \ref{chap:search_hh}.
        Sources of uncertainty in {\color{red}{red}} ({\color{blue}{blue}}) pertain only to the
        search presented in Chapter~\ref{chap:search_stop} (\ref{chap:search_hh}).
        Those in black are considered in both analyses.
        For the uncertainties related to the SM background modelling, it is indicated
        whether or not they are computed using the Transfer Factor Method (Section~\ref{sec:transfer_factor}).
    }
    \label{tab:syst_summary}
    \begin{footnotesize}
    \begin{center}
        %\begin{tabularx}{\textwidth}{@{\extracolsep{\fill}}c c c c}
        %\begin{tabularx}{\textwidth}{c c c c}
        \begin{tabular}{c c c c}
        \toprule
        \hline
        \multicolumn{4}{c}{\textbf{Event-level}} \\
        \hline
        \multicolumn{4}{c}{Luminosity Measurement} \\
        \multicolumn{4}{c}{Pile-up Modelling} \\
%        \multicolumn{3}{c}{Luminosity} & \multicolumn{2}{c}{ Pile-up }  \\
        \hline
        \midrule
        \multicolumn{4}{c}{\textbf{Object Reconstruction}} \\ \hline
        \hspace{-2cm} \underline{\textbf{Jets}} &     \hspace{-1.8cm} \underline{\textbf{Flavor Tagging}} & \hspace{0.5cm}\underline{\textbf{Leptons}} & \hspace{0.3cm} \underline{\textbf{\met}} \\
        \hspace{-2cm} Jet Energy Scale (JES) &        \hspace{-1.8cm} $b$-tag Eff. & \hspace{0.5cm}Reconstruction Eff. & \hspace{0.3cm} Soft-term Resolution \\
        \hspace{-2cm} Jet Energy Resolution (JER) &   \hspace{-1.8cm} Mis-tag Eff. & \hspace{0.5cm}Identification Eff. & \hspace{0.3cm}  Soft-term Scale \\
        \hspace{-2cm} Pile-up Suppression (JVT) &     \hspace{-1.8cm}   & \hspace{0.5cm}Isolation Eff. &\hspace{0.3cm}  \\
        \hspace{-2cm} &     \hspace{-1.8cm}   & \hspace{0.5cm}Trigger Eff. &\hspace{0.3cm}  \\
        \midrule
        \midrule
        \multicolumn{4}{c}{\textbf{Background Modelling}} \\
        \hline
        \multicolumn{1}{l}{\textbf{Souce of Uncertainty}} & \multicolumn{2}{c}{\textbf{Affected Background Processes}} & \textbf{Transfer Factor Approach?} \\
        \hline
        \multicolumn{1}{l}{Hard-Scatter Generation}  & \multicolumn{2}{c}{\ttbar, \wt, \color{blue}{\zhf}} & Yes \\
        \multicolumn{1}{l}{Fragmentation} & \multicolumn{2}{c}{\ttbar, \wt} & Yes \\
        \multicolumn{1}{l}{Additional Radiation (ISR and FSR variation)} & \multicolumn{2}{c}{\ttbar, \wt} & Yes \\
        \multicolumn{1}{l}{PDF Choice and Uncertainty} & \multicolumn{2}{c}{\ttbar, \wt, \color{blue}{\zhf}, \color{red}{\vv}} & Yes \\
        \multicolumn{1}{l}{Scale ($\mu_R$, $\mu_F$) Variations} & \multicolumn{2}{c}{\ttbar, \wt, \color{red}{\vv}, \color{blue}{\zhf}} & Yes \\
        \multicolumn{1}{l}{Cross-section Uncertainty} & \multicolumn{2}{c}{\color{blue}{\ttbar}, \color{blue}{\wt}} & No \\
        \multicolumn{1}{l}{\ttbar~Interference Uncertainty} & \multicolumn{2}{c}{\wt} & Yes \\
        \multicolumn{1}{l}{Prompt Subtraction \& Fake Composition} & \multicolumn{2}{c}{Fake Lepton Background} & No \\ 
        \multicolumn{1}{l}{SS-OS Extrapolation} & \multicolumn{2}{c}{\color{blue}{Fake Lepton Background}} & No \\ 
        \midrule
        \midrule
        \multicolumn{4}{c}{\textbf{Signal Modelling}} \\
        \hline
        \multicolumn{4}{c}{{\color{blue}{Fragmentation}}} \\
        \multicolumn{4}{c}{Scale ($\mu_R$, $\mu_F$) Variations} \\
        \multicolumn{4}{c}{PDF Choice and Uncertainty} \\
        \multicolumn{4}{c}{Cross-section Uncertainty} \\
        
%        %\multicolumn{1}{l|}{ }  & \underline{\ttbar} & \underline{$Wt$} & \underline{Diboson} & \underline{$Z$+jets} & \underline{Fake Lepton Estimate} \\
%        \multicolumn{1}{l|}{Uncertainty Source }  & \ttbar & $Wt$ & Diboson & $Z$+jets & Fake Lepton \\
%        \hline
%        \multicolumn{1}{l|}{Hard-Scatter Generation}  & \checkmark & \checkmark & & & n/a\\
%        \multicolumn{1}{l|}{Fragmentation}  & \checkmark & \checkmark  & & & \\
%        \multicolumn{1}{l|}{Additional Radiation} & \checkmark & \checkmark & & & \\
%        \multicolumn{1}{l|}{PDF} & \checkmark & \checkmark &  & \checkmark & \\
%        \multicolumn{1}{l|}{Scale ($\mu_R$, $\mu_F$) Variations} & \checkmark & \checkmark & & & \\
%        \multicolumn{1}{l|}{Cross-section} & \checkmark$_{HH}$& & & & \\
%        HS + Matching & HS + Matching & & &  \\
%        Frag. + Had. & Frag. + Had. & & &  \\
        \hline
        \bottomrule
        \end{tabular}
    \end{center}
    \end{footnotesize}
\end{table}


%%%%%%%%%%%%%%%%%%%%%%%%%%%%%%%%%%%%%%%%%%%%%%%%%%%%%%%%%%%%%%%%%%%
%%%%%%%%%%%%%%%%%%%%%%%%%%%%%%%%%%%%%%%%%%%%%%%%%%%%%%%%%%%%%%%%%%%
%
% EXPERIMENTAL UNCERTAINTIES
%
%%%%%%%%%%%%%%%%%%%%%%%%%%%%%%%%%%%%%%%%%%%%%%%%%%%%%%%%%%%%%%%%%%%
%%%%%%%%%%%%%%%%%%%%%%%%%%%%%%%%%%%%%%%%%%%%%%%%%%%%%%%%%%%%%%%%%%%
\subsection{Experimental Uncertainties}
\label{sec:syst_experimental}

\subsubsection{Event-wide Uncertainties}
Event-wide (process-independent) uncertainties affecting the overall normalisation of the processes
relate to both the luminosity and pileup measurements.
The uncertainty on the integrated luminosity used to normalise all MC simulated
processes is derived following the methodology described in Ref.~\cite{LumiUncert}.
For the 2015+2016 (full Run 2) dataset, relevant to the analysis described in Chapter~\ref{chap:search_stop} (\ref{chap:search_hh}),
this uncertainty was found to be 2.1\% (1.7\%).
The uncertainty on the luminosity measurement does not affect processes whose
SR normalisation is constrained by data in dedicated CRs.

An uncertainty is considered on re-weighting the pileup distributions in the
MC simulation.
The re-weighting is applied in order to correct for the differences in the actual pile-up
distributions observed in data and those assumed at the time of producing the MC simulation (Section~\ref{sec:pileup_sim}).

\subsubsection{Jet Uncertainties}
The systematic uncertainties on reconstructed jet objects are related to the jet energy
resolution (JER), jet energy scale (JES), and JVT.
There are many sources of uncertainties related to the JES and JER, each related to a specific
part of the JES and JER calibration measurements, as described in Section~\ref{sec:jet_calib}.
They arise from the techniques and corrections derived in MC, including statistical, detector,
modelling effects, jet flavor compositions, pileup corrections, and $\eta$-dependence effects.
The effects of the JES and JER uncertainties are among the more dominant sources of uncertainty
on the final analysis results in both analyses to be presented.
Given the complexity of the JES and JER calibrations, there are nearly 100 components associated with their uncertainties
that must be incorporated into an analysis' measurement uncertainty.
In general, the leading uncertainty in searches such as those to be presented in the current thesis
is statistical in nature --- related to the limited statistics in data and in the MC simulation used
for the final SR predictions ---  and not related to the systematic uncertainties.
For this reason, the searches described in Chapters~\ref{chap:search_stop} and \ref{chap:search_hh}
used a reduced set of JES and JER uncertainties that are derived following a Principal Component Analysis (PCA)
designed to capture only the dominant components --- and their correlations --- of the total set of JES and JER uncertainty components
that are relevant to the phase space being probed by the analyses.
In the analysis presented in Chapter~\ref{chap:search_stop} (\ref{chap:search_hh}), this uncertainty reduction
process leads to only 4 (34) separate components of the combined JES and JER systematic uncertainty.

\subsubsection{Flavor Tagging Uncertainties}
There are uncertainties in the jet flavor tagging efficiencies, as well as in
the measured mis-tagging efficiencies associated with $c$- and light jets.
They are a mixture of statistical, experimental, and modelling uncertainties
incurred during the flavor tagging calibration procedures.
The uncertainties enter into the analyses through their impact on the scale-factors,
described in Section~\ref{sec:ftag_calib},
that are applied in the analysis.
Given the importance of $b$-tagged jets in the final states of the signal processes
in the analyses described in Chapters~\ref{chap:search_stop} and \ref{chap:search_hh},
these uncertainties have non-negligible impact on the analyses' results.

\subsubsection{Lepton Uncertainties}
Uncertainties on the measurement of leptons correspond to the electron and muon reconstruction,
identification, trigger, and isolation efficiencies in a manner similar to the flavor tagging
in that systematic variations incurred in the associated scale-factor measurements are applied
in the analysis.
Additional uncertainties related to the lepton kinematics due to the resolution and scale of the
electron (muon) energy (momentum) measurement are considered.
The muon momentum measurement uncertainties are derived for both the ID and MS measurement
of the combined muons used in the analyses.

\subsubsection{Missing Transverse Momentum, \met}
Systematic variations of the \met are coherently incurred as a result of the
systematic variations, described above, being applied to the objects provided as input to the \met calculation:
the leptons and jets.
Additional uncertainties related to the scale and resolution of the soft-term of the \met calculation
are also considered.
Given that the analyses considered in Chapters~\ref{chap:search_stop} and \ref{chap:search_hh}
are characterised by real sources of \met, the soft-term component plays a small role in the magnitude
of the \met and therefore its uncertainties have negligible impact on the analyses.
Generally, the \met uncertainties are sub-dominant.

\subsubsection{Fake Lepton Estimate}

As will be seen in Chapters~\ref{chap:search_stop} and \ref{chap:search_hh}, the overall contribution
of the fake background processes to the analyses to be presented in this thesis is very small.
As a result, the systematic uncertainties related to this background are almost always negligible
in impact on the analyses' final results.
However, given the subtle nature of these data-driven estimates, we describe the methods by which
systematic uncertainties are derived for them.
They are as follows:

\begin{description}
    \item{Matrix Method (Section~\ref{sec:matrix_method}):} There are three primary sources of systematic
        uncertainty ascribed to the fake estimate derived from the Matrix Method in the analysis described
        in Chapter~\ref{chap:search_stop}.
        The first is related to the limited statistics in the region(s) used for the determination of the
        fake efficiencies, $\varepsilon_r$ and $\varepsilon_f$.
        The second is related to the `prompt subtraction': the evaluation of the portion of real events contaminating the
        regions in which the fake efficiencies are derived, which is evaluated using the MC simulation.       
        The component of real lepton sources is varied by $\pm 30$\% and the impact on the resulting
        fake efficiencies is propagated to to the final analysis. 
        The third component is related to the compositional differences of the background sources leading
        to fake leptons in the region in which the fake efficiencies are derived and in the regions (SRs)
        in which the fake estimate is applied.
        To assess the systematic related to compositional differences, alternative regions in which the leading
        components of the fake backgrounds (e.g. the heavy flavor component) are varied are defined and the impact on the measured fake efficiencies
        is used to define an uncertainty.
    \item{Same-sign Extrapolation Method (Section~\ref{sec:same_sign_extrap}):} As with the Matrix Method,
        there are three primary sources of uncertainty prescribed to the fake background estimate derived using
        the Same-sign Extrapolation Method. The first, as with the Matrix Method, is related to the prompt subtraction
        in which the rate of contamination of prompt processes is varied.
        In the analysis described in Chapter~\ref{chap:search_hh}, the real contamination in the same-sign
        regions is varied by $\pm 50$\%, and the impact on the final fake background estimate is used to quantify
        an uncertainty.
        Following the studies in Refs.~\cite{TOPQ-2015-09,TOPQ-2017-05}, an additional uncertainty (on top of the statistical component) on the
        extrapolation from the SS to the OS regions is included by varying the $f^{SS \rightarrow OS}$ factors
        by $\pm 20$\% and assessing the impact on the final fake background prediction.
        The statistical uncertainty related to the extrapolation over the $d_{hh}$ discriminant,
        described in Chapter~\ref{chap:search_hh}, is applied as an additional uncertainty on this fake background estimate.
        Uncertainties related to the composition of fakes in the SS and OS regions are found to be small,
        and are covered by the extrapolation factor uncertainties already described.
\end{description}



%%%%%%%%%%%%%%%%%%%%%%%%%%%%%%%%%%%%%%%%%%%%%%%%%%%%%%%%%%%%%%%%%%%
%%%%%%%%%%%%%%%%%%%%%%%%%%%%%%%%%%%%%%%%%%%%%%%%%%%%%%%%%%%%%%%%%%%
%
% TRANSFER FACTOR METHOD
%
%%%%%%%%%%%%%%%%%%%%%%%%%%%%%%%%%%%%%%%%%%%%%%%%%%%%%%%%%%%%%%%%%%%
%%%%%%%%%%%%%%%%%%%%%%%%%%%%%%%%%%%%%%%%%%%%%%%%%%%%%%%%%%%%%%%%%%%
\subsection{The Transfer Factor Method}
\label{sec:transfer_factor}

For estimating the impact of modelling uncertainties on the SM backgrounds, described in Section~\ref{sec:syst_bkg_modelling},  that
have dedicated CRs to constrain their overall normalisation in the SRs,
the so-called Transfer Factor (TF) Method is used.
Since the purpose of the CRs is to constrain the process' normalisation using
the observed data, we do not want the systematic variations to directly
impact the processes' normalisations within the SRs.
Instead, the impact of the systematic variations is assessed via their effect
on the SM processes' acceptance in the CRs and SRs.
If a given systematic variation for a given SM process affects the process
in a coherent manner across both the CR and SR, then the resulting affect of the systematic variation
on the analysis should be reduced.
This will nearly be guaranteed if the phase spaces being probed by the CR and SR
are similar.
The more dissimilar the phase space being probed by the CR and SR, the larger
the expected impact of a given systematic variation due to the larger kinematic
extrapolation required.
This can conceptually be seen by considering Equation~\ref{eq:cr_tf}:
\begin{align}
    N_{p}^{\text{SR}} &= \mu_p \times N_{p,\,\text{MC}}^{\text{SR}} \nonumber \\
        &= \left( \frac{N_{p, \text{data}}^{\text{CR}}}{N_{p,\,\text{MC}}^{\text{CR}}} \right) \times N_{p,\,\text{MC}}^{\text{SR}} \nonumber \\
        &= N_{p, \text{data}}^{\text{CR}} \times \left( \frac{ N_{p,\,\text{MC}}^{\text{SR}}  }{ N_{p,\,\text{MC}}^{\text{CR}} } \right) \label{eq:cr_tf} \\
        &= N_{p, \text{data}}^{\text{CR}} \times \underbrace{\tau_p}_{\substack{\text{Transfer} \\ \text{ Factor}}} \nonumber,
\end{align}
where `$p$' is the process for which the CR and normalisation factor are defined, $\mu_p$
is the CR-derived normalisation factor (c.f. Equation~\ref{eq:mu_fac}),
$N_{p, \text{data}}^{\text{CR}}$ is the observed data in the CR with the MC simulation
for all processes that are not process $p$ subtracted, and $N_{p,\,\text{MC}}^{\text{SR}}$
is the MC-based SR prediction of process $p$.
The quantity $\tau_p$ is the process' TF that extrapolates the observed data in the CR
to the SR.
It can be seen that if the MC simulation response for a given process for a given systematic variation is the same
across both the CR and SR, that the TF will be unchanged as a result of the systematic
variation and therefore the predicted contribution of this process in the SR will
be unaffected by the systematic uncertainty.
If kinematics differ across the CR and SR, the acceptance for a given process may vary
in going from the CR and SR (or vice versa) and therefore such a cancellation is not likely to
occur due to the larger extrapolation required.

It can be seen, then, that for processes whose SR normalisation is derived in dedicated CRs,
that the TF appearing in Equation~\ref{eq:cr_tf} quantifies the impact of acceptance
variations between the CR and SR.
Systematic variations of the SM backgrounds with dedicated CRs, then, are quantified
by their impact on the resulting TF values.
This is contrary to assessing their impact by measuring the change in a process' SR prediction
by simply comparing the SR predictions before and after a given systematic variation is applied.
The uncertainties ascribed to SM processes via the TF Method are computed as follows,
\begin{align}
    \Delta \tau = \frac{ \lvert \tau_{\,\text{nominal}} - \tau_{\,\text{variation}} \rvert} { \tau_{\,\text{nominal}} },
    \label{eq:tf_uncert}
\end{align}
where $\tau_{\,\text{nominal}}$ ($\tau_{\,\text{variation}}$) is the TF computed
using the nominal (systematically varied) prediction of the process.
The quantity $\Delta \tau$ is then taken as a fractional uncertainty on the corresponding
process' SR predicted yield.

Sources of uncertainty stated as following the TF approach in Table~\ref{tab:syst_summary} are
quantified following Equation~\ref{eq:tf_uncert}.
The others are assessed simply by taking the impact of the variation on the process'
predicted yields in each of the regions appearing in the analyses.

%%%%%%%%%%%%%%%%%%%%%%%%%%%%%%%%%%%%%%%%%%%%%%%%%%%%%%%%%%%%%%%%%%%
%%%%%%%%%%%%%%%%%%%%%%%%%%%%%%%%%%%%%%%%%%%%%%%%%%%%%%%%%%%%%%%%%%%
%
% BACKGROUND MODELLING
%
%%%%%%%%%%%%%%%%%%%%%%%%%%%%%%%%%%%%%%%%%%%%%%%%%%%%%%%%%%%%%%%%%%%
%%%%%%%%%%%%%%%%%%%%%%%%%%%%%%%%%%%%%%%%%%%%%%%%%%%%%%%%%%%%%%%%%%%

\subsection{Background Modelling Uncertainties}
\label{sec:syst_bkg_modelling}

Uncertainties in the modelling of specific processes, SM or otherwise, are typically
assessed by comparing the nominal MC simulation for the processes in question to
that of an MC simulation with certain theoretical or phenomenological parameters varied.
In this way, one can assess the impact of the underlying assumptions made
in the MC simulation on the analyses' final results.

\subsubsection{Top-quark Pair and Single-top $Wt$ Production}
The production of SM top-quark pairs, \ttbar, is by far the most dominant SM background
in both of the analyses described in Chapters~\ref{chap:search_stop} and \ref{chap:search_hh}.
Some of the largest uncertainties in both analyses described therein are related to the modelling
of the \ttbar~process.
The process in which a single top-quark is produced in association with a $W$-boson plays a large
role in the analysis presented in Chapter~\ref{chap:search_hh}.
The MC simulation of these two processes is done using the same MC generation steps and, as a result,
the methods by which their systematic evaluation is performed are the same.
Here we describe the systematic variations used in the analyses for both of these processes.

Variation of the hard-scatter generation is performed by comparing the \ttbar~and \wt~samples produced
using \textsc{Powheg} for the matrix element generation to that using \textsc{aMC@NLO}, but keeping
the showering and fragmentation model the same in both (\textsc{Pythia8}).
The effects of the choice of fragmentation and hadronization model is assessed by comparing
the use of \textsc{Pythia8}, used in the nominal \ttbar~and \wt~predictions, to that of \textsc{Herwig}, all the while
keeping \textsc{Powheg} for the hard-scatter generation in both cases.
The description of the ISR and FSR provided by the hard-scatter generation is ascribed an
uncertainty by varying the underlying parameters that describe the characteristic energy
scales at which additional radiation (beyond that described by the matrix-element hard-scatter)
is produced.
In \textsc{Powheg}, this is controlled primarily by the \texttt{hdamp} parameter, and to assess
the impact of this parameter's value on the analyses it is varied up and down by a factor of 2 relative
to the nominal scenario.
The impact of the choice of PDF used in the hard-scatter generation is assessed by varying the
choice of PDF from the nominal, which is that of the NNPDF collaboration~\cite{Ball:2014uwa},
to that of the MMHT~\cite{Harland-Lang:2014zoa}, CT14~\cite{Lai:2010vv}, and PDF4LHC~\cite{Butterworth:2015oua} PDF sets and taking
the envelope of the variations.
The PDF error set, comprised of 100 separate components and provided by the NNPDF collaboration, is used to assign an additional uncertainty
on the nominal PDF used.
To assess the impact of the finite order in QCD at which the MC simulation is taken, and
sensitivity to missing higher-order terms, the factorization and renormalization scales,
$\mu_F$ and $\mu_R$, are varied in all pairings possible in which either is varied by a factor of two
up or down.
The resulting \textit{scale uncertainties} are derived by taking the envelope of the resulting variations.

Only in the analysis presented in Chapter~\ref{chap:search_hh} is the uncertainty related
to the theoretical prediction of the \ttbar~and \wt~processes taken into account.
The former is taken to be $\pm 5.82\%$ and the latter $\pm 5.32$\%~\cite{Czakon:2013goa,ATLAS-CONF-2013-102,Kidonakis:2010ux}.
The analysis described in Chapter~\ref{chap:search_hh} considers the sum of the \ttbar~and \wt~processes
as a single background and constrains the normalisation of their sum using a dedicated CR.
The uncertainties on these processes' cross-section, therefore, are taken into account to allow
for the \textit{composition} of the combined estimate ($\ttbar+\wt$), as opposed to its normalisation, to vary within the theoretical uncertainties.

An additional uncertainty, considered in both analyses presented in the current thesis, is related
to the non-trivial quantum interference between the NLO predictions of the \ttbar~and $\wt+b$~processes relevant
to both analyses.
This uncertainty is assessed by comparing the estimates of the \wt~process simulated
under the so-called Diagram Removal (DR) and Diagram Subtraction (DS) schemes that
are used in the NLO calculation of the single-top \wt~process.
The DR and DS schemes, and their comparison as a means of assessing a systematic uncertainty
as just described, are fully described in Refs.~\cite{Frixione:2008yi,ATL-PHYS-PUB-2016-004}.
This \textit{interference uncertainty} is negligible in the analysis presented in Chapter~\ref{chap:search_stop},
given the relatively small contamination of the \wt~process.
However, it is an important uncertainty in the analysis presented in Chapter~\ref{chap:search_hh}.
In this latter analysis, the \wt~background contributes at a rate equal to that of the \ttbar~process, making
the assessment of their interference a subtle one that will be described in Chapter~\ref{chap:search_hh}.

\subsubsection{$Z$-boson Production in Association with Heavy-Flavor Jets}

Processes involving the production of a SM $Z$-boson in association with
two or more heavy-flavor jets are important for the analysis presented in Chapter~\ref{chap:search_hh}.
These processes are referred to simply as `\zhf'.

The PDF and scale uncertainties on the \zhf~process are computed in exactly the
same fashion as for the \ttbar~and \wt~processes described in the previous section.
Additional uncertainties on the parton shower merging scales are assessed by varying
the CKKW merging scales~\cite{Lonnblad:2012ix} used by the \textsc{Sherpa} MC generator
used for the \zhf~simulation.
To assess the impact of different MC hard-scatter generation and fragmentation models, the 
nominal sample produced using \textsc{Sherpa} is compared to a \zhf~simulation produced using
\textsc{Madgraph+aMC@NLO} for the hard-scatter generation.
This last source of uncertainty is important for the analysis described in Chapter~\ref{chap:search_hh}
and amounts to an uncertainty of $\pm 15$\% on the \zhf~background estimate in the analysis' SRs.
The other sources of uncertainty on the \zhf~background are minor in comparison.

\subsubsection{Diboson Production}

The SM production of boson pairs --- diboson production (`$VV$') --- plays an important role only in the
analysis described in Chapter~\ref{chap:search_stop}.
The nominal $VV$ background estimate is simulated using the \textsc{Sherpa} MC generator.
Uncertainties arising as a result of variations in the factorization and renormalization scales,
PDF variations, and uncertainties in the CKKW merging scales are considered.
The methods by which these uncertainties are quantified are described above.
These systematic uncertainties amount to about a $\pm 9$\% uncertainty in the SR prediction
in the analysis described in Chapter~\ref{chap:search_stop}, but are generally smaller than
the modelling uncertainties on the \ttbar~background process.

%%%%%%%%%%%%%%%%%%%%%%%%%%%%%%%%%%%%%%%%%%%%%%%%%%%%%%%%%%%%%%%%%%%
%%%%%%%%%%%%%%%%%%%%%%%%%%%%%%%%%%%%%%%%%%%%%%%%%%%%%%%%%%%%%%%%%%%
%
% SIGNAL MODELLING
%
%%%%%%%%%%%%%%%%%%%%%%%%%%%%%%%%%%%%%%%%%%%%%%%%%%%%%%%%%%%%%%%%%%%
%%%%%%%%%%%%%%%%%%%%%%%%%%%%%%%%%%%%%%%%%%%%%%%%%%%%%%%%%%%%%%%%%%%

\subsection{Signal Modelling Uncertainties}
\label{sec:syst_sig_modelling}

Uncertainties on the signal modelling are assessed in both analyses presented in Chapters~\ref{chap:search_stop}
and \ref{chap:search_hh}.
The methods by which the uncertainties arising due to variations in the $\mu_F$ and $\mu_R$ scales, as
well as due to variations in the PDF choice, are assessed in exactly the same manner
as for the SM backgrounds described in Section~\ref{sec:syst_bkg_modelling}.
In both searches, these uncertainties on the signal are negligible.
For the analysis described in Chapter~\ref{chap:search_stop}, the uncertainties
arising as a result of varying the fragmentation model are found to be negligible for the
phase space probed by the SUSY parameter space, described by the masses of the supersymmetric
top-quark and LSP, in which that search is performed.
In the analysis described in Chapter~\ref{chap:search_hh}, uncertainties arising as a result of
the choice in fragmentation model are assessed by varying the choice of parton shower generator
from the nominal \textsc{Herwig} to an alternative description provided by \textsc{Pythia}
and have little impact on the analysis' final result.
The uncertainties on the theoretical cross-sections on the signal models appearing in the two analyses
are generally small and are found to not have a large impact on either of the analyses' final results.

