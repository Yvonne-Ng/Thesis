\chapter{The Standard Analysis Method for Resonance finding}
\label{chapter:dimuon}

\epigraph{\textit{But the truth can be re-found; most often it has already been written elsewhere. .}}{--Jacques Lacan, Ecrits}
%Lacan, J. (2006). The function and field of speech and language in psychoanalysis. Écrits. (Trans. B. Fink). New York, NY: Norton.

\section{Introduction}
This chapter describes the analysis methods done in performing the analyses covered in chapter~\ref{chapter:dijet} and ~\ref{chapter:dimuon}. 
All three analyses presented in this thesis fall into the resonance search category, which are analyses that look for bumps like excesses on top of smooth backgrounds. The method is simple in experiemntal signature and theoretical calculation. It also had many Many particles has been discovered that way before, which include the W, Z and the Higgs boson.

The analysis searches in the resonance mass variable, which numerically is the addition of the 4-vector of the two candidate muons and reconstruct a resonance candidate and subsequently project them in the resonance mass dimension. The addition of the 4-vectors follows from the Einstein's Special Relativitiy relations:

In both the dimuon and dijetISR channels, the background size is large compared to what can be reliably generated by Monte Carlo, therefore the "smoothness" feature in the resonance mass projection is relied upon for to estimate the background by direct estimation on the data. The "smoothness" requirement refers to is described a events vs variable mass range fit that does not have a sharp trigger turn on curve or any features of a similar width to signal. 

The dijetISR method uses a fit function based method, whereas the dimuon analysis uses a Gaussian process based method. They are all discussed in section~\ref{section:backgroundest}.The dijetISR boosted analysis utilizes a mixed method of Gaussian process smoothing and a transfer function from a control region, as well as some MC generation for the less prominent processes. The method differs from the general approach and will be left out of the discussion here, however,
details can be found here in this paper and note.

The chapter below follow the run through of a typical physics data analysis of ATLA: first data preparation will be discussed, which include trigger mapping, the event selection, and the senstiviity test. After that, the resonance signature-based data-driven background estimation as well as its verification steps before unblinding will be discussed in~\ref{section:backgroundest}. Lastly, the final result of the finding will be presented in the statistics method. Excesses as a bump on top of the
smooth background will be searched for in the bumphunter method. If an excess is not seen, an upper limit on the signal strength that excludes a particular signal model. The result along with the treatment of the systematic uncertainties are discussed in the ~\ref{section:stats} section. Both the Bayesian method and the frequentist method are discussed in this chapter. 

\section{Data Preparation}
\label{section:dataprep}
This analysis uses the ATLAS 13 TeV proton-proton collision dataset generated between 2014-2019. The dataset goes through several steps of data preparation before getting the dimuon mass data spectrum for resonance search analysis. 
In the following section, the steps will be studied over
First, the trigger was analyzed, then the event is selected with cuts applied for the optimal sensitivity. 
%The data preparation is done by the SUSYtool framework. 

\subsection{Trigger}
Due to the large amount of data and limited computing bandwidth, in ATLAS, the data goes through a triggering procedure, where it was decided in real time whether the the collision event should be kept. An event passes the trigger and gets kept if it contains at least one or more online object that satisfy the trigger requirement, the event is called to be "kept in a stream" named by the requirements.

The analysis uses unprescaled trigger stream~\footnote{} through an "AND" trigger of the different streams to maximize events kept for the analysis. Pre-scaled triggers~\footnote{} and delayed streams~\footnote{} are not used in this analysis. 

The following table summarized the triggers map used in this analysis, the integrated luminosity shows the effective amount of data available
These triggers used are all fully efficient, meaning cuts are taken from $p_{T}$ values where 99.5\% events passes the trigger.

%\subsubsection{trigger weighting}


\subsection{Event Selection}
The trigger provide the first cut into the dataset by requiring muons pairs above a certain transverse momentum to be chosen. Further selection criteria are applied as the following. 

\subsection*{Good Run List}
Due to different running condition of different detector parts during the time of the run, not all recorded data that passes the trigger are of the quality worthy to be analyzed. The ATLAS Data Quality group has published data quality basing on the health of various subsystem during runs. 

Different event selection cuts are applied to the events to select objects that are fully optimized for this particular analysis.  

%In addition, as different events utilizes different subsystem, the data quality flag is also specific to the physics object being analysed. 
In addition, as part of the data integrity checks, events that are considered corrupted/bad with LAr burst or tile corruption are also rejected. 


%\subsection{Data Integrity Checks}



\subsection{Simulated Physics Events(Monte Carlo)}
The analysis is data driven, meaning that the background estimation is done directly on the data itself along with signal search. However, simulated physics events, or Monte Carlo, is still used to study the different background estimation strategies and their corresponding signal sensitivity, as well as the detector resolutions and binning. 
Using a parton-distribution function, an event generator is first used to simulate pp colision and their tree level outcomes. The parton showers are then simulated probabilistically through some hadronization models, the detector response to these events are then modeled by a GEANT4 detector simulator. 
In this analysis, as there isn't enough Monte Carlo for the background estimation, fast simulation is also used to avoid the problem from a low statistics, the fast simulation is generated from pythia and a smearing function produced afterwards. 

%and therefore different weighting are used to make sure that theMonte Carlo matches up to the data events being studied. 


\subsection{Object Preparation}
The analysis uses muons as objects, their detailed preparation is shown in chapter~\ref{dataprep:muon}. 

Here a list of the different calibration working points chosen for the analysis.


\subsection{Mass Spectrum Resolution} 
The mass resolution is found through a width-to-mean($\frac{\mu}{sigma}$) Gaussian fit on the $m^{reco}/m^{truth}$ on the four-vector addition of the two candidate final state particles. The truth particles came from PYTHIA generation whereas the reconstructed value came from the GEANT4 detector reconstruction. A Gaussian fit is performed to find the mean ($\mu$) and the width($\sigma$).  

\subsection{Binning Strategy} 
To prepare a mass spectrum, the data must be properly binned. The bins has to be narrower than the the width of the expected signal, as this would minimize the possibility of mistaking fluctuation as an excess and improve characterization of the signal. A uniform binning is chosen for the dimuon analysis, as it would make the background estimation more simplified. For the dijet analysis, to cope with the low statistics in the high mass region, variable bin size based on detector resolution is
used. 
%A fit between the detector resolution and the signal width is done here.  The binning value is chosen to be 0.25 GeV, the experimental resolution of the middle of the data spectrum. All the signal from the theory model will take up more than at least two bins.
% Remember argument for choosing this?


\subsubsection{MC/Data Comparison}
After the MC are properly prepared and weighted, it is then compared to data in control regions adjacent to signal region. This step reveals issues in object calibration, sub-detector functioning, incorrect trigger weighting, Monte Carlo event weight corrections as large discrepancies and therefore is a catch-all test for whether all the above steps are done correctly. If MC and data shows a reasonable agreement, the MC can be trusted to perform the test strategy on the
background estimation study in the folllowing part. 

\subsubsection{Background Modelling}
    The background modeling in both the dijetISR resolved analysis and the dimuon analysis are done through a direct fit of the background model on data directly. 
    As it's a smooth background test, it needs to 
    
    The fitting method is performed by Gaussian process and the fit is evaluated in the following way. 
    A Wilk test can be done to the fit. 
    \subsubsubsection{Gaussian Process} 
    \subsubsubsection{Signal Injection Test}
    \subsubsubsection{Spurious Signal Test}


 
\section{Statistics Testing}
In this section, the statsitical testing will be presented 
The statistics testing on the model can be presented as a test on the significance Z: 

\begin{equation}
\[ Z= \Phi^{-1}(1-p) \]
\end{equation}

Here, p is numeric representation of the agreement between the data and the model given a hypothesis, whereas the $\Phi$ is its conversion in terms of quantile of on a Gaussian distribution that is the cummulative probability. From here on, Z will be called the significance and p will be referred to as the p-value. 

There are two distinct statistics test that are done here, each test done on a threshold value of Z. The search and the upper limit setting. Before the Here, the null hypothesis is the hypothesis that include only known processes from the Standard Model hypothesis with no excess; the alternative hypotheis is the hypothesis that include both background and signal of a certain signal strength.

1.  The Search: 
\textit{Rejection of the null hypothesis}
This step is sometimes also referred to as the discovery test. This test is done on the null hypothesis alone. If the given data return a the Z > 5, the null hypothsis is rejected, something beyond the Standard Model is observed in the data to a statistically significant degree. The rejection test is also called the discovery test, note that the rejection of the null hypothesis is a necessary condition but not a sufficient condition for discovery. For example, in the Higgs boson analysis of 2012, a significance of Z> 5 was observed in the search, but more test
were required other than this test before discovery could be claimed. 
The search phase is done by in a frequentist manner through a bumphunter method for both the dimuon and dijetISR analysis. 
%()corresponds to $p = 2.87 \cdot 10^{-7}$.

2. The Upper Limit Setting
\textit{Finding signal strength value where the alternative hypothesis is rejected to a 95\% confidence level}
The upper limit setting is the finding of the signal strength where the alternative hypothesis is rejected at 95\% confidence level. The 95\% confidence level corresponds to a critical p-value of 0.05 or a Z value of 1.64. The signal strength can also be represented as the cross section. There is also a median significance + error band called the expected limit for the signal strength value when the data fluctuates with the systematics effect to add comparison with the upper limit. 
Both the frequentist and the Bayesian method on upper limit setting will be presented in the following. The Bayesian method is used for the dijetISR analysis and the frequentist method is used for the dimuon analysis. 
%the Bayesian statistics bethod, this is often only done when test 1 failed (when the null hypotheis of standard model is not rejected, or no excess is observed). In the frequentist method test, by using the likelihood ratio as the test statistics to calculate the p value, both of these tests can be done in the limit setting. 
%the Bayesian tool is used for the dijetISR analyses, whereas the frequentist method is used for the dimuon analysis. 

\subsection{The Search}
    The search is done by the bumphunter method. It is the step in the analysis where it's studied whether the a discrepancy from the Standard Model background prediction is seen, if yes, an excess is claimed to be found and a through testing on the statsitics procedure will be done. Statistically this is defined by an rejection of a null hypothesis scenario. 

    The statistics of the Standard Model prediction can be described by a probability distribution, as all other counting experiments, the this is described by a Poisson probability distribution. 

    \[ p(x|\lambda) = \frac{\lambda^{k}e^{-\lambda}}{k!} \]

    The test is done in a frequentist manner, it compares the observation outcome x with a fixed critical value $\alpha$ of a probability distribution generated from pseudo-experiment from a Poisson distribution. The value used to reject or accept the null hypothesis is a probability quantity called the p-value. Statistically, it is known to be a false-discovery probability. 
    If the observation compared with the distribution outcome is smaller than the critical p-value, the null hypothesis is rejected.  

    \[ P(x \ni w|H)<= \alpha \]

    As the test wants to see how much deviation is seen between the data and the prediction, the observable x is presented as a test statistics that quantifies deviation between the observation and predictable. The observable is then compared against the pseudoexperiment distribution in such a form. 
    
    \[ p = P(T>=t_{0}| H_{0}) \]


    \subsubsubsection*{Test Statistics}
    Different test statistics can be chosen to describe how well the data fits the background prediction,  
    Traditionally, how well the data fits the background estimation can be described by either the $\chi^{2}$ test or the log-likelihood that gets derived from the Poisson distribution of the 
    in the case of excess search, a test statistics chosen would need to fit the need of the signal shape searched for.
    However, these test statistics fail to take into account the excess from the neighboring bins, and therefore a different bumphunter statistics is defined, in the analysis below. 
    In each bumphunter search window(from bin m to n), the expected number of events in the window is: 

    \[ d= \sum_{i=m}^{n} d_i \]

    The Observed number of events is: 
    
    \[ b= \sum_{i=m}^{n} b_i \]
    
    Assuming the observed counts in each bin follows a Poissonian distribution, the cumulated count across all the bins is as the following:

    \begin{equation}
    \[ t= \Bigg\{  \sum_{n=0}^{d} \frac{b^{n}}{n!} e^{-b} for d < b \newline
        \sum_{n=d}^{\infinity} \frac{b^n}{n!} e^{-b} for b \leq d \]

    \label{test statistics}
    \end{equation}

    The above expression can be represented by the Gamma function: 

    \begin{equation}
    t = \Bigg\{  1-\gamma(d+1, b), for d < b \nextline
                 \gamma(d,b)       for b \leq d

    \label{test statistics}
    \end{equation}


    This quatity takes the overall deficit and or excess in the region compared to the \chi^{2} and loglikelihood calculation. Only excesses are used for this study due to the nature of the resonace signature being looked for. 

    In this analyses two kind of bump-hunter statistics are used, one is used to described the localized excess in each window and the other on whether an excess is found for the overall spectrum.

    Local excesses is represented by a set of bumphunter test statistics calculated for a window scan along the spectrum from 2 bin wide to half the spectrum size.

    An overall standard model agreement is defined by the negative log of the smallest p value calculated for any window scanned given the pseudo experiemntal. The test statstiscs can be represented as such 
    \[ t_{0} = - log t_{min} \]

    
    In addition to the bumphunter test statistics, there is the tailhunter, KS, Jeffreys and no sidebands bumphunter that can also be chosen for different signal searched for in other physics analyses. More information can be found in the Bumphunter paper here: . 


    \subsubsection{The Bumphunter}
    % Look up the bumphunter fit. 
    The search in the analysis is defined as the statistical process to look for an excess beyond the Standard Model null hypothesis using the bumphunter test statistics. 

    The procedure is describe bumphunting procedure to look for the search test.
    The bumphunter test statistics are performed all the windows along the spectrum starting two bin wide to half the spectrum size. This is assumed to be true for all of the following steps illustrated. 

    1.  First a fit is performed, according to the method from the last section.
    2.  A set of bumphunter window are defined across the spectrum. Bumphunter statistics and their p values are calculated. 
    If all window has a bumphunter statistics p-value above 0.01, stop, no excesses is found. Alternatively, if any one of the window goes below p< 0.01, the window most discrepant window (the window with the lowest p-value) is removed, and the fit is reperformed.
    3. After the fit is reperformed, the bumphunter statistics and the p value is recalculated for every window, if all p-value is above 0.01, stop, otherwise, see if the most discrepant region is adjacent to the fit, if yes, add the adjacent two bin (left and right) to the fit. 
    Three outcomes are possible, if there is one excluded window and a background fit with a bumphunter statistics p-value of >0.01, the null hypothesis is rejected, an excess may be seen; if no window is excluded and the fit has a bumphunter p-value statistics above 0.01, the null hypothesis is accepted; in all other sceanrios, more tests needs to be done.

    A summary of test on a background only test and a data spectrum with signal injected is shown here.

    
\section{Limit Setting}

The previous section performed a test on whether the null hypothesis of there is only known process in the Standard Model is rejected. In the section of limit setting, the alternative hypothesis is looked at, and an alternative hypothesis on the mass or cross section of the signal model the data has excluded the information of interest. 

The limit is a hypothesis testing procedure that test both the 
compares the null hypothesis of absence of signal ($H_{0}$) to the hypothesis where there is a signal($H_{1}$). 
The limit is a projection of the $H_0$ and $H_1$ hypothesis in either the cross-section or the signal strength axis. This provide another way to discuss signal discovery, and information on an 95\% confidence limit in the If $H_1$ shows agreement with $H_{0}$ when fluctuation is taken into account, the 
Even in the absence of a signal, a statement can be made about the signal strength value that is ruled out by the signal+ background. 

The limit setting procedure can be done in both the frequentist and Bayesian way on ATLAS, and in the analyses presented, the dimuon analysis is done in the frequentist way, whereas the dijet analysis is done in the Bayesian way. 

\subsection{Bayesian limits}

\subsection{Frequentist Limits}
The frequentist limit follows from the frequentist philosophy, it creates 
The limit setting with the frequentist method does both rejection of the null hypothesis and the exclusion of the signal+background hypothesis up to 95\% for each mass point either in terms of signal strength or cross se

The probability distribution of the observable (x) histogram can be given as a Poissonian distribution described by both the signal strength and other nuissance parameters. The likelihood from the Poissonian distribution, when maximized, gives the best fit parameters that would describe the data. 

\begin{equation}
\[ L(\mu, \theta) =  \prod_{i=0}^{N} \frac{(\mu s_{j} + b_{j})^{n_j}}{n_{j}!}e^{-\mu s_j + b_j} \prod_{k=1}^{M}\frac{u_{k}^{m_{k}}}
{m_{k}!} e^{-u_{k}}
\end{equation}

The parameters that would influence the model's prediction fit to data include both the signal strength and . Since the only parameter we are interested in learning about is the signal strength, the influence from the unknown truth value of the nuissance parameters can be taken into account/eliminated from the optimization process by considering the profile likelihood: 

\begin{equation}
\[ \lambda(\mu) = \frac{L(\mu, \hat{\hat{\theta}})}{L(\hat{\mu}, \hat{\theta})}
\end{equation}

Here, the denominator is the maximized unconditional likelihood function, where $\hat{\mu}$ and $\hat{\theta}$ is the parameter values that would maximize the unconditional likelihood function; the nominator is the conditional likelihood function, that would be maximized under the maximum-likelihood estimator of $\theta$ and $\mu$. 

To aid the optimization process and also to reflect the nature of the excess search experiment that neglects negative signal event value, the test statistics defined as the following: 
\begin{equation}
    \[ q_{\mu} = -2 ln \frac{L(\mu, \hat{\hat{\theta}})}{L(\hat{\mu}, \hat{\theta})}
\]
\end{equation}


\subsection{Asymptotic distribution}


Using the Wald theorem the test statistics can be approximated: 

\begin{equation}
\[-2ln(\lambda(\mu))= \frac{(\mu- \hat{\mu})^{2}}{\sigma^{2}} +O(1/\sqrt{N})}\]
\end{equation}

If the terms in O are neglected, it follows that the test statistics will then asymtotically follow a non-central chi-square distribution: 

\begin{equation}
\[f(q_{\mu}) = \frac{1}{2\sqrt{q_{\mu}}} \frac{1}{\sqrt{2\pi}} [exp(-\frac{1}{2}(\sqrt{q_{\mu}}+ \sqrt{\Lambda})+ exp(-\frac{1}{2}(\sqrt{q_{\mu}-\sqrt{\Lambda}})^{2})]
\end{equation}

\subsection{The Asimov dataset for a median experimental sensitivity}

The finding of \hat\hat{\mu} in the denominator is CPU intensive, the dataset is large with many dimensions. The asimov dataset reduces the calcuation here by showing that \hat\hat{\mu} can be approximated by \mu', the expected value of the strength parameter. From this, CPU intensive calcuation of the p-value, significance and thus the upper limit and its fluctuation can be reduced to simple formulae that only look at the representative Asimov dataset. 
A proof is given as the following section, and the formulae used for upper limit and expected limit calcuation in this thesis is included in the end. 

From the definition, true values of parameters will result in a maximum likelihood and therefore the derivative of the $ln_{L}$ will be equal to 0. 

\[ \frac{\partial{L}}{\partial{\theta_{j}}} = \sum_{i=0}^{N}(\frac{n_{i}}{\nv_{i}}-1) \frac{\partial{\nv_{i}}}{\partial{\theta_{j}}}+ \sum_{i=0}^{N}(\frac{m_{i}}{u_{i}}-1) \frac{\partial{u_{i}}}{\partial{\theta_{j}}} =  0 \], 

where n is the number of count in the bin in the signal region variable x, and N is the total number of bins, and m is the number of counts in the bin in the control region may help further constrain the background, M is the maximum number of bins. 

The maximal likelihood value for $n_{i,A}$ and $m_{i,A}$ is associated with their expectation value. 

\[ n_{i,A} = E[n_{i}] = \nv_{i} = \mu's_{i}(\theta) + b_{i}(\theta), 

    m_{i,A} = E[m,i] = u_{i}(\theta) \]

    These optimal parameters \theta can be estimated through large size Monte Carlo generation. And it's called the asimov dataset. 

Therefore, it is shown that 

\[ \lamda_{A}(\mu) = \frac{L_{A}(\mu, \hat{\hat{\theta}})}{L_{A}(\hat{\mu}, \hat{\theta})} = \frac{L_{A}{\mu, \hat{\hat{\theta}}}}{L_{A}(\mu', \theta)} \]

\sigma_{A} can be approximated as the following:

\[ 2ln(\lambda_{A}(\mu)) \approx \frac{(\mu-\mu')^{2}}{\sigma^{2}}=\Lambda
\]

\[ \sigma_{A}^{2} = (\mu-\mu')^{2} \]

, when there is no signal, $\mu'$ is taken to be zero. 


Once that's found, the likelihood profile of the search can be estimated through the Asimov likelihood as such:

Following from the above, the test statistics, the upper limits, the discovery median significance can all be found through substituting \hat{\mu} as \mu' by the asimov method. 
Only the results are quoted here, more details are found here. 

\bullet{Test statistics distribution}
\begin{equation}
\[ F(t_{\mu}| \mu) = 2\Phi(\sqrt{t_{\mu}})-1 \]
\end{equation}

\bullet{p-value of a hypothesized \mu for an observed value t_\mu}

\begin{equation}
\[ p_\mu = 1-F(t_{\mu}| \mu})=2(1-\Phi(\sqrt(t_{\mu})))\]
\end{equation}

\bullet{The significance that corresponds to the above p-value}
\begin{equation}
    \[ \Z_{\mu} = \Phi^{-1}(1-p_{\mu})  = \Phi^{-1}(2\Phi(\sqrt{t_{\mu}})-1)
\end{equation}

%\subsection{The Search (discovery test)}
%A hypothesis is said to be rejected if its p-value is below a certain threshold z value of alpha. This is the basis of the discovery test.
%Using the test statsitics:
%
%
%
%\begin{equation}
%\[ Z_{0}= \Phi(1-p_{0})= \sqrt(q_{0}) \]
%\end{equation}

To find the upper limit at a certain 1-\alpha confidence interval, \mu where p_{\mu}=\alpha needs to be found.

\begin{equation}
\[ \mu_{up/lo} = \hat{\mu} +- \sigma\Phi^{-1}(1-\alpha/2) \]
\end{equation}

In Asimov a median distribution 

In the experiment, statistical fluctuation need to be taken into account, it can either be taken as that the data has room to fluctuate, or that the expected value from the model can fluctuate and change. Either way, the sigificance will fluctuate even if \mu' is assumed to be true and hold fixed. In the analysis, it's taken that data will be fluctuating. 

In the case where there is assumed that there is no background, 

\begin{equation}
\[Z_{0}= \Big{\hat{\mu}\sigma    \hat{\mu}\gt 0, \newline 
         0    \hat{\mu}<0. }\]
\end{equation}

Using the fluctuation, the significance can be calculated as the following: 

% this is for the discovery test
%\begin{equation}
%    \[ Z_{0}(\mu'+N\sigma) = med[Z|\mu'] +N \] 
%\end{equation}
%
%\begin{equation}
%    \[ max[Z_{0}(\mu'+N\sigma) = med[Z|\mu'] -N, 0] \] 
%\end{equation}

And the associated expected limit error band can be given as such, where the strenght parameter of \mu' is substitute by the \mu^{'} due to fluctuation and the sigma from the test statistics of the asimov set. 

The median upper limit:
\begin{equation}
    \[ med[\mu_{up}|\mu'] = \mu' + \sigma\Phi^{-1}(1-\alpha) 
\end{equation}

The error band: 
\begin{equation}
    \[ band_{N\sigma} = \mu' + \sigma(\Phi(1-\alpha)+-N) \]
\end











% \hat{\hat{\theta}} is the value that maximizes the respective \mu
% \\hat{mu} and \hat{\theta} are the value with the maximal likelihood. 

% \{mu} is the value that we want to guess, if mu is of a value where in the test statistics distribution at 95%, it can be said that it's excluded. 

\[ \lambda_{A}(\mu) = \frac{L_{A}(\mu, \hat{\hat{\theta}})}{L_{A}(\hat{\mu}, \hat{\theta})}


Following certain estimation, the non-centrality term, \Lambda, can be given by the following:


Following from this, it can be shown mathmatically the median significance, the error bands, as well as the upper limit exclusion can be given as the following: 


The upper limit is given by: 
\[ \

The median significance of the expected limit is given by
\[ \Lambda = \frac{(\mu- \hat{\mu})^{2}}{\sigma^2}=-2ln{lambda} \], where mu is the expected value, this is known as the asimov dataset. 


\begin{figure}[!htb]
    \begin{center}
        \includegraphics[width=0.75\textwidth]{figures/chapter_analysismethod/asimovApproximation}
        \caption{
			Generation showing convergence of the asimov dataset and the median significance dataset from full generation. .
        }
        \label{fig:Model_figure}
    \end{center}
\end{figure}


\[ Z^{2}_{median} = -2ln{\lambda_{median}} \]

As can be shown from the MC generation plot here, the median conversed 

Using the above approximation, one distribution can be used instead of a large amout of ensemble data, the above is known as the asimov dataset. 




There are some special cases where the 

From the Wilk's theorem, the upper limit on the signal strength \mu_{95} can be given as the following:



%\begin{equation}
%\[ f(t_{\mu}; \Lambda)=\
%\end{equation}


%\subsection{The Data spectrum}
%
%The frequentist limit is done in the Asimov way to save on computing resources for the calculation. 
%
%The limit setting procedure is done by an Asimov method, 

\section{Systematics}
\section{Future Extensions}

