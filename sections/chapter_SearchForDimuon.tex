\chapter{The Resonance Search on the Dimuon Signature}
\label{chapter:dimuon}

\epigraph{\textit{But the truth can be re-found; most often it has already been written elsewhere. .}}{--Jacques Lacan, Ecrits}
%Lacan, J. (2006). The function and field of speech and language in psychoanalysis. Écrits. (Trans. B. Fink). New York, NY: Norton.

\section{Introduction}

    % Why this is interesting  
    Owing to its unprecedented high energy, the run I and early run II of ATLAS has long focused on the high mass resonances searches. This has left some gaps in the low mass region that are unexplored. In the two lepton final state, there has been an analysis after the Z boson peak since the beginning of Run 1, however the mass region below the Z mass peak has left a region totally uncovered on ATLAS.

\begin{figure}[!htb]
    \begin{center}
        \includegraphics[width=0.75\textwidth]{figures/chapter_dimuon/
        dimuonStudies}        
        \caption{
            This cartoon illstrate the target signal region of the analysis and how it has not been covered by the previous high-mass dilepton analysis. 
            \label{fig:dimuonstudies}
    \end{center}
\end{figure}
    
    % Why this signature is something good to be looked for.
    Since ATLAS collide partons, there are fewer background on leptons than jet processes, this allows lower transverse momentum lepton events to be saved. Using two muons as the resonance final state items, lower mass resonances can be searched for, going below the limits of the dijet resonances. The signal significance would be higher compared to jet. 

    % Why this result is intersting to look for
    This search results is interesting to the theoretical community can be reinterpreted in the dark matter summary. 

\section{Historical searches}    
    Previous searches has been done directly in Tevatron experiment. Indirect constraints has been made in LEP as well. 

    CMS results and tevatron results: 
    
    This study will be an independent finding from ATLAS and the first search done in this region.  

\section{The Search Channels}
Since there is a trigger turn-on feature near 45 GeV in the distribution due to the cut in muon transverse momentum in trigger, 
This analysis is based on a pheno-study first done here: 

There are two search channels in the analysis, each covering a different mass region, the boosted and the resolved analysis:

\begin{figure}[!htb]
    \begin{center}
        \includegraphics[width=0.75\textwidth]{figures/chapter_atlas/turnedon}
        \caption{
            this figure shows the trigger turn on curve in the inclusive dimuon channel, and how utilizing the boosted channel a smooth background is possible in the lower mass region from 10-45 gev. samples used here are from the monte carlo sample in section~\ref{}, a minimal cut of muon pt >14 gev are done on the two leading muons. 
        \label{fig:cernacceler
    \end{center}
\end{figure}






    \section{Signal Theoretical Model}

    The analysis uses the dark matter LHC benchmark model and dark photon model outlined in~\ref{sec:LHCDM}.
Other models that are of interest and can be searched for in this analysis includes the W', the quantum blackholes,and axion models. 
The following search focuses on the search on the vector dilepton signature from the dark photon model and the dark matter benchmark. But as Gaussian limits are also set, the results will be reinterpretable for many other models that predicts scalar or axion-vector resonances. 

    The signal channel is shown here 


\section{Data preparation}

\subsection{Samples Used for the Analysis}

\begin{table}[!htb]
    \begin{center}
    \caption{
        The table shows the Monte Carlo dataset used for the analysis. 
    }
\label{tab:MC samples}
\begin{tabular}{|l|l|}
\hline
\textbf{MC Type}   & \textbf{DSID}                                                         \\ \hline
Z+ jets $\mu\mu$   & 364100 - 364113 , 364198-364203                                       \\ \hline
Z+jets $\tau \tau$ & 364128 - 364141 , 364210-364215                                       \\ \hline
$t\bar{t}$         & 410472                                                                \\ \hline
Diboson Sherpa     & 364253 - 364255 , 363355 - 363360 ; 363489 ; 364250 ; 364288 - 364290 \\ \hline
Top decay          & 410644 - 410645 , 410658 - 410659 ; 410648 - 410649                   \\ \hline
W + jets munu      & 364156 - 364169                                                       \\ \hline
$b\bar{b}$         & 363833                                                                \\ \hline
$c\bar{c}$         & 363834                                                                \\ \hline
\end{tabular}
\end{center}
\end{table}

\subsection{Trigger Chain}

\subsection{Event Selection}

\subsection{Sensitivity Test}

\subsection{Dimuon Mass Spectrum Resolution}

\subsection{Binning Strategy}
The strategy to the binning strategy can be found here: 

\subsubsection{MC/Data Comparison}
After the MC are properly prepared and weighted, it is then compared to data to see if a proper comparison is seen. 


\section{Statistics Testing}


\section{The Bayesian Search Phase}


    \subsubsection{The Search Test: the Bumphunter}
    
\section{Limit Setting}

\subsection{Asimov frequentist limits}
\section{Systematics}
\section{Future Extensions}

