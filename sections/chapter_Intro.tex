
\chapter*{Introduction}

% Searching for truth is part of what makes us human. 
An eagerness for the truth is part of what makes us human. This has been true across cultures through different times: ancient Greeks who toyed with mathematics of trigonometric objects saw an eternal truth in their beauty; in the Eastern Zhou Dynasty of China\footnote{Eastern Zhou is a Chinese dynasty that existed between 771 to 476 BCE. Influential scholars include \textit{Confucious}, \textit{Lao-tse}, and \textit{Mo-tse}.}, influential scholars debated the truthful nature of humans and,
therefore, their subsequent duty; driven by the same thirst, in modern days humans took to space, went on the moon and began exploring the extra-terrestrial frontier for beyond earth. The aspiration for truth formed knowledge. It has advanced technology, medicine, law, science, and human psychology. \textit{Everything} in our civilization derives from this ambition in humans to seek truth.

%advance different technology, allow human to live for longer with greater convenience and provide a deeper understanding on the human nature. 
%different streams of Indian philosophers debates and discuss the Satya (loosly translate to truth) and its practice.
%distinction between the world of phenomenon and nomenon, the latter describe a part of world that is beyond our senses but describe the fundamental nature of things. 

Human's endeavor for the truth is always to uncover something more than what is already known: over two thousand years ago, Plato saw the truth in the \textit{ideal forms} beyond the shadow projections of everyday things~\cite{plato1961republic}; Amongst the mechanical industrial world view 19th century, Immanuel Kant postulated a world of the \textit{transcendental},  a world of the fundamental
nature of things, behind the world of \textit{phenomena}, of human senses~\cite{kant1908critique}.\footnote{Kant in his famous book the Critque of Pure Reasons~\cite{kant1908critique} call this world of the transcedentals the \textit{noumenon} (contrasting with the world
of \textit{phenomenon} that can be perceived by human senses.
Kant believes this \textit{transcendental} realm that we have no access to is actually the true nature of things. The \textit{transcendental} is sometime referred to as the \textit{thing-in-itself}.)}. In contemporary times of the $21^{st}$ century, faced with the clash of cultures and perspectives facilitated by the internet and modern transportation, truth is no longer seen as a static set of doctrine or statements of facts in the older days\footnote{The view of truth as a
direct correspondence to facts is known as the Correspondence theory
of truth~\cite{sep-truth-correspondence}. Detail objections to the theory can be found in the reference.}, but humans' endeavor to the truth has not stopped.

Combined with the idea of subjectivity, contemporary philosopher Alain Badiou has formulated the logic for the search for truth as "truth procedures": truth is the subsequent discourses and knowledge re-development that follows from a rupture or an accident\footnote{Badiou called such rupture or accident an \textit{Event}} that shatters our existing knowledge system~\cite{badiou2007being}. This "truth procedure" echoes ideas from Thomas Kuhn on scientific \textit{paradigm
shifts}~\cite{kuhn2021structure}, but also describes truth seeking as part of individual \textit{ethics}. "There is always only one question in the ethics of truth", Badiou wrote, "how will I, as someone, continue to exceed my own being?"\cite{badiou2002ethics}. The search for truth is not seen merely a human desire, but also a moral duty to remain authentic to himself/herself.

%The ingrained search of truth in human nature shows a fundamental quest to authenticity, and that human can be more than what we believe we are. 
% philosophy no longer treat truth as an ideal like Plato, or mechanical statics like in the early 20th century, but it's nonetheless a human drive to

Particle physics is a search for truth in the most elementary form. It seeks to understand fundamental particles and their interaction as building blocks to the world around. Not only does it expands knowledge of the smallest possible observable universe, but it's also a study of the things that constitute chemicals, biological cells, animals, highrise structures, and \texit{everything else} under the physical realm. Knowledge in particle physics is also tightly related to objects of the largest scale, namely cosmology, knowledge in particle physics affects the structure of the entire universe, its evolution and is essential to answering questions on the origin of the human.

Particle Physics as a field was developed in the 1960s together with advances in Quantum Field Theory. Through new experimental findings and new theoretical model-building techniques, the Standard Model of Particle Physics was established. The theory is the epitome of human's understanding on the fundamental building block of the universe. The model enjoyed many successes: many predictions made were later confirmed by experimental findings. The W, Z, and the Higgs Boson are all discovered that way. Human
understanding of matter, its origin, and the governing laws between the interactions has since been greatly advanced. 

However, even with its success, there remain many open questions to the Standard Model of Particle Physics. %Or in Badiou's term, many "accidents" have ruptured the current understanding of particle physics, a truth procedure is in need repaired the tear in understanding.
For example, the Standard Model in its current form does not describe gravity or any of its interactions; there are many mathematical un-naturalness that shows up in the different orders of the hierarchy problem in the theory; in addition, the
Standard Model also does not include dark matter, which makes up the majority of the matter (~85\% ) of the universe or dark energy, which in 
Einstein's theory explained the expanding universe seen today.  

One way of fixing the inconsistency is by finding new particles predicted by different theoretical hypotheses. One way to find these new particles  is through high-energy particle collisions. If dark matter interacted weakly with Standard Model particles in the early universe like many theoretical models have come to predict, it would be possible for them to be produced through high-energy collisions of Standard Model particles.

This thesis offers a few humble attempts to resolve some of these standing problems in the Standard Model of particle physics by discovering new particles that would offer extra insights to future theory building. These are done by attempting to create previously unobserved conditions in high-energy particle collisions via the Large Hadron Collider. It uses data collected by the ATLAS experiment in the Large Hadron Collider. These new particles, if discovered, as little inconsistencies to the Standard Model of Particle
Physics, are nuggets to truth. It has great potential to add to our understanding of fundamental particle physics.

The thesis is organized as the following: chapter~\ref{chapter:SM} describes the history and the description of the Standard Model; chapter ~\ref{chapter:DM} discusses the theory of dark matter; chapter~\ref{chapter:ATLAS} presents the experimental setup of the Large Hadron Collider and the ATLAS experiment; chapter~\ref{chapter:NSW} touches the upgrade in the New Small Wheel of ATLAS and its impact on muon reconstruction performance;
chapter~\ref{chapter:CommonAnalysisItems} covers the common analysis items (reconstruction
of objects )and their preparation before the analyses; Chapter~\ref{chapter:ISR}-~\ref{chapter:dimuon} outline the three analyses that I played a major part in, namely the dijet boosted and resolved analysis, and the low mass dimuon analysis, along with its reinterpretation to the dark matter model; this thesis also covers a chapter on other resonance finding techniques~\ref{chapter:weird},  that discusses data-driven, experimental-signature based search methods. Including some done through machine learning techniques. 
These novel techniques, together with theory-driven approaches, could hold keys to unlocking the next finding that can revolutionalize our current understanding of particle physics. 

The pursuit of \textit{Aletheia}(truth) is a nessessity to human \textit{authenticity}, as philosopher Martin Heideigger famously put. This thesis could be seen as a humble response in face of such human condition, or a fulfillment of an ethical duty to remain \textit{authentic}, even if nothing more was achieved. But perhaps there is still more to the pursuit than any eloquent philosopher can ever formulate: it's an adventure of
joy and pleasure to uncovering all things unknown. It is a journey that reminds us, what we are is always more than what we know.

%To summarize the section with a a reason that drives this attempt and many other that would follow: what's in the \textit{real} behind the surface, is always worth uncovering, not only as a ethical duty as stated by many philospher, but this is also a pursue done out of a sense of joy and pleasure in truth finding: we are always more than what we _know_ we are.

%Truth(\textit{Aletheia}) is not mere correspondence to facts, but rather a human(\textit{Dasein})-involving process to to _unconceal_ what is in the real.

%The search of truth is the uncovering of facade in front for authenticity and admitting of our ignorance, there is something more than what we currently know to be uncovered:  
% As Martin Heidigger would put it. 

%Plato in his allegory of the cave illustrates truth as ideal forms behind the shadows that human sees.
%The search of truth is an act of authenticity

%Truth is no longer viewed as a static, but still something that human strive to get close to in our different endeavour in the realm of particle. In search for the strange little resonances, ooking for such rupture in the truth fabric, 

%Searching for truth is the search for that there is more than what is before us. 



%- Cannot know to what we can know
%Extending the frontier to what human know 

%Particle physics is the study of the most fundamental particles in the universe and their interaction. It's at the boundary of what it's known

%This thesis focuses on a series of searches done over resonance hunting 
%they are well motivated initially on the dark matter model, which is detailed in chapter . 

%Chapter discuss and review resonance hunting over smooth background, and give a discussion on how it is possible to look for new particle in dark matter model and beyond with this method.
%Literature on this will be further discussed and some pilot studies on the different methods of looking for new particle is reviewed here. Some inital dimuon results is shown. 

%Search for truth and something more is something fundamental and 

%The signature itself and i

%2. 

%An introduction to why what we are is more than, the overview of standard model strive to give an account for the history of the standard model of particle physics, 
%In search of truth, theory is perfected in the process, and that it is in this process that we gain tools to look for further truth.
%It's possible for human to be more than what we know. We are more than what we are. 

%* Psychoanalysis 
%Lacan in lack in jouissance. 
%The only unexplored region is the science of jouissance, the science of pleasure. 

%* Other unexpected signatures of particle physics not driven by models, the unexplored landscape of two body resonances. 



