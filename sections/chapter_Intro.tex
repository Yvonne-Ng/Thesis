1. Human's desire to search for truth is fundamental . Truth procedure. 
* Plato 
* Kant 
* Heidegger  Dasein is a being towards death.
Authenticity. 
* Alain Badiou 

Searching for truth is part of what makes us human. This has been true across different cultures, spaces and time: ancient Greeks philosopher Plato contends there is truth that exists as ideals behind shadows in a cave, ancient chinese philosophers during ~\textit{Eastern Zhou}\footnote{Eastern Zhou is a Chinese dynasty that existed between 771 to 476 BCE. Influential scholars in this period, which include \textitConfucious}, \textit{Lao-tse}, and \textit{Mo-tse} have influenced understanding of
human nature, ethics and societal laws up to this day} endlessly debated the truthful nature of human; Knowledge is developed in such process of truth finding has advanced technology, medicine, law and psychology. Almost everything in our civilization are derivatives to this desire in human to seek truth. The search for truth drives progress, 

%advance different technology, allow human to live for longer with greater convenience and provide a deeper understanding on the human nature. 
%different streams of Indian philosophers debates and discuss the Satya (loosly translate to truth) and its practice.
%distinction between the world of phenomenon and nomenon, the latter describe a part of world that is beyond our senses but describe the fundamental nature of things. 


Human's endeavor for the truth is always to uncover something that is more than what we currently know, this stands the test of time across different generations: In ancient Greece, equipped with the knowledge of elegant trigonometric objects, Plato saw truth as ideal elegant \textit{forms} beyond the mere projections of everyday things. Amongst the mechanical world view and industrial development of the early 20th century, Immanuel Kant saw a world of the \textit{trancendental}, the fundamental
nature of things, behind the world of \textit{phenomena}, of human sense.\footnote{Kant in his famous book the Critque of Pure Reasons ~\cite{} call this world of the transcedentals the \textit{noumenon} (contrasting with the world
ofphenomenon that can be perceived by human senses.Kant believes this transcendental realm that we have no access to is actually the true nature of things. The transcendental is sometime referred to as the \textit{thing-in-itself}.)}. In contemporary times, faced with the crash of cultures and perspectives, truth is no longer seen as a static set of doctrine or statements of facts\footnote{This is known as the Correspondence theory of truth}, but human's endeavor to truth has not stopped here.
Combined with the idea of subjectivity, Alain Badiou has elegantly formulated the logic for the search for truth as "truth procedures": truth is the subsequent discourses and knowledge re-development that follows from an a rupture or an accident\footnote{Badiou called such rupture or accident an \textit{Event}} that shatters our existing knowledge system. This "truth procedure" echos ideas from Thomas Kuhn in the Scientific Revolution on paradigm shifts. "There is always only one question in the ethics of truth", Badiou said, "how will I, as some-one, continue to exceed my own being?"\cite{Ethics: An Essay on the Understanding of Evil}. Under such a view, The search of truth is not merely a human drive, but a moral duty.

%The ingrained search of truth in human nature shows a fundamental quest to authenticity, and that human can be more than what we believe we are. 
% philosophy no longer treat truth as an ideal like Plato, or mechanical statics like in the early 20th century, but it's nonetheless a human drive to

Particle physics is a search for truth in the most elemantary form. It seeks to understand fundamental particles and their interaction as building blocks to the entire world. Not only does it expands knowledge of the smallest possible observable universe for human, it's also what constitute chemicals, biological cells, human-beings, highrise structures and therefore \texit{everything} in the physical realm in the world; knowledge in particle physics closely impacted the study of the largest scale, namely cosmology, knowledge in particle physics affect the the structure of the entire universe and its evolution and is essential to answering questions on the origin of human-being.

Particle Physics as a field was developed in the 1960s together with advances in Quantum Field Theory. Through new experimental findings and new model building techniques, the Standard Model of Particle Physics was established. The theory is the epitome of human's understanding of fundamental building block of the world around. It has enjoyed many sucesses, as it made many accurate prediction that was later confirmed by experiment: notably, the prediction of the W, Z and the Higgs Boson. Human understanding of the interaction between elementary particles has been greatly advanced with the establishment of the Standard Model. 

With its success there remains many open questions to the theory of However, there remain many open questions. 

[On the open questions of standard model of particle physics]

One major aspect is the dark matter, the standard model of particle physics does not account for dark matter, matter that is believed to make up about 85\% of the the existing universe, other include 

One way of finding new particles and study the fundamental particle interaction is through high energy particle collision. If dark matter interacted weakly with Standard Model particles in the early universe like many theoretical model has come to believe, it would be possible for them to be produced through high energy collision of Standard Model particles.

This thesis offer a few attempt in searching for loopholes in the Standard Model, in particularly the ones driven by a particle dark matter model, using data collected by the ATLAS experiment in the Large Hadron Collider. These inconsistencies are nuggets of , that could tear apart existing fabric of the current undertstanding of fundamental physics, and thereby introduce means for a truth finding procedure to begin. 

The thesis is organized as the following: the first chapter describes the history and the description of the Standard Model; the second chapter discusses the theory of dark matter; the third chapter presents the experimental set up of the Large Hadron Collider and the ATLAS experiment; the fourth chapter touches the upgrade in the New Small Wheel of ATLAS and its impact on muon reconstruction performance; chapter ~\ref{chapter}- ~\ref{chapter} covers the common analysis items (reconstruction
of objects )and their preparatio before analyses. ~\ref{} outline the three analyses that I played a major part in, namely the dijet boosted and resolved analysis, and the low mass dimuon analysis; this thesis also cover a chapter that discuss data driven experimental signature based search method that include some done through machine learning techniques in chapter ~\ref{}, these techniques, together with theory driven approaches, could hold keys to unlocking the next finding that can revolutionalize our current understanding of particle physics. 

The search of \textit{Aletheia} (truth) for human is necessary to the search for \textit{authenticity}, as philosopher Martin Heideiger famously put. ~\cite{} This thesis can be seen as a humble response in face of such human condition, if I have achieve nothing more. To summarize a reason that drives this attempt and many other that would follow: what's in the \textit{real} behind the surface, is always worth uncovering, because as human beings, we are always more than what we _know_ we are.

%Truth(\textit{Aletheia}) is not mere correspondence to facts, but rather a human(\textit{Dasein})-involving process to to _unconceal_ what is in the real.

%The search of truth is the uncovering of facade in front for authenticity and admitting of our ignorance, there is something more than what we currently know to be uncovered:  
% As Martin Heidigger would put it. 

%Plato in his allegory of the cave illustrates truth as ideal forms behind the shadows that human sees.
%The search of truth is an act of authenticity

%Truth is no longer viewed as a static, but still something that human strive to get close to in our different endeavour in the realm of particle. In search for the strange little resonances, ooking for such rupture in the truth fabric, 

%Searching for truth is the search for that there is more than what is before us. 



%- Cannot know to what we can know
%Extending the frontier to what human know 

%Particle physics is the study of the most fundamental particles in the universe and their interaction. It's at the boundary of what it's known

%This thesis focuses on a series of searches done over resonance hunting 
%they are well motivated initially on the dark matter model, which is detailed in chapter . 

%Chapter discuss and review resonance hunting over smooth background, and give a discussion on how it is possible to look for new particle in dark matter model and beyond with this method.
%Literature on this will be further discussed and some pilot studies on the different methods of looking for new particle is reviewed here. Some inital dimuon results is shown. 

%Search for truth and something more is something fundamental and 

%The signature itself and i

%2. 

%An introduction to why what we are is more than, the overview of standard model strive to give an account for the history of the standard model of particle physics, 
%In search of truth, theory is perfected in the process, and that it is in this process that we gain tools to look for further truth.
%It's possible for human to be more than what we know. We are more than what we are. 

%* Psychoanalysis 
%Lacan in lack in jouissance. 
%The only unexplored region is the science of jouissance, the science of pleasure. 

%* Other unexpected signatures of particle physics not driven by models, the unexplored landscape of two body resonances. 



