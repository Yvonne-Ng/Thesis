\chapter{Theory of Dark Matter}

%\epigraph{\textit{So it goes...}}{---Kurt Vonnegut, \textit{Slaughterhouse
%		Five}}
	
%\epigraph{\textit{Science is a miracle.}}{--Ron Swanson}

\epigraph{\textit{If you wish to make an apple pie from scratch, you must first invent the universe.}}{--Carl Sagan, \textit{Cosmos: A Personal Voyage}}
\epigraph{\textit{All models are wrong, but some are useful.}}{--George Box}


\section{Introduction}
Dark matter is arguably one of the most solid evidence for beyond the standard model physics. In 1933, Fritz Zwicky first conceptualized the existence of a dunkle Materie ("dark matter") holding the galaxies to the cluster center and thus preventing them from flying apart. The proposition was later confirmed in differen findings by Horace W. Babcock and Jan Oort. The existence of such matter also supported evidence across different cosmological scales including the cosmic
microwave background, and bullet cluster collision.

Dark matter is known for interacting gravitationally and at most weakly (though recent evidence has shown weak scale interaction to be unlikely). It does not interact electromagnetically. Its known properties does not match any known standard model particle. Very little is known about its composition and and interaction mechanism. However, as dark matter makes up about 85\% of the matter in the universe, five times more than the ordinary matter covered in previous chapter. Dark matter also plays a major role in astro-object mechanics, galactic cluster formation and the evolution of the universe. Without a proper understanding of dark matter and its composition, human understanding of the universe will be incomplete.

Experiments and observations over the years have contricted some dark matter properties, but much of the physics is still remains a mystery. It is generally accepted that dark matter is a particle. In the LHC, it is believed that dark matter can be studied through direct production in the collider. If dark matter can be produced in a lab, it will make probing its properties possible. 

In this chapter, evidence for the existence of dark matter will first be reviews in 3.1, dark matter and the current properties that restrict its nature will be covered in 3.2 , effective model in the LHC that studied dark matter is discussed in 3.3. An overview of search methods is covered in 3.4 and 3.5. Lastly, the latest LHC searches result and its implication on future of dark matter searches is covered in 3.6. 

\Section{Evidence for Dark Matter}

\subsection{Galactic Rotational Curve}
The first proposition of dark matter in 1930s in the Coma cluster is later confirmed by further systemic galaxy rotational curve studies by Vera Rubin, Kent Ford and Ken Freeman. It was found that ordinary luminous matter only could account for one-sixth of the matter in most galaxies. 

[Add in virial theorem explanation]

\subsection{Gravitational Lensing}

General relativity shows that massive objects would curve and modify the space-time curvature around them. Light ray travelling through the curved space-time around the massive object would therefore be bent. The effect mirrors that of angled lenses bending light rays due to a velocity differece of light ray across different media and is called gravitational lensing. There are different forms of lensing effect, roughly arranged by the strength of their effects. Strong lensing
effect happens when bending of light result in either multiple images, a light ring or an arc, more typically observed around massive centers of galaxy clusters and galaxies; whereas weak lensing usually involves statistical analysis of many objects over a large region; micro-lensing is detected when there is an apparent change in the brightness of the source.


Figure of the bullet cluster collision of CL0024\+17 shown in here is produced from gravitational lensing effect. The red part of the diagram denotes ordinary matter and the blue part reflects dark matter. In the cluster collision, ordinary matter bend light around them and became luminous from the collision; the blue part shows a large part of the cluster did not interact with each other as ordinary matter would and simply passes through each other. This is a clear evidence for the existence of dark matter in the clusters.

\subsection {Cosmic Microwave Background}

In the beginning of the universe, ordinary matter and dark matter all exist in a hot plasma soup with frequent interaction between charged particles and photons through thomson scattering. There comes a very important period in the universe called the recombination period, where the expansion of the universe has cool the plasma soup enough that charged particles began to form neutral atoms. Photons stopped scattering on the charged particles and went on unhindered. Due to red
shifting effect, they form a microwave background of the universe that can still be observed today. This is the Comic Microwave Background. 
This original background is nearly a black body and is therefore very uniform, but there exists small temperature variations. The variation can be decomposed into an angular power spectrum, as shown in figure []. 
Due to the different interaction between dark matter and ordinary matter with photon and each other, simulation shows that different dark matter and ordinary matter make-up of the universe would result in different angular spectroscopy shape. 
Study from Planck on cosmic microwave background gives a clear composition and percentage abundance of dark matter. Dark matter is not only an essential part of the universe, its composition is approximately 5 times as large as ordinary matter. Giving evidence that dark matter makes up the majority of the universe. 

\section{Dark matter properties}
While dark matter itself has never been directly observed, many studies in the cosmological, astrophysical and particle scale has constrained much of its properties. In this section and the remaining thesis, the following properties of dark matter will be assumed along with its justifications.

\subsection{Dark}
Dark matter got its name from its little to no interaction with light compared to ordinary matter in cosmological observation. Collision of the bullet cluster has also greatly constrained its interaction with itself.  
It is taken that it will not interact with collider detector and its interaction with ordinary matter would be rare in the energy scale of the LHC. 

\subsection{Long Life Time}
In the particle model of dark matter, many requires a Z_2 symmetry (such as the R parity of supersymmetry). This prevent dark matter particles from decaying into lighter Standard Model objects. In most LHC analyses, it will be treated as a single object that does not decay further into other detectable SM particles. 

\subsection{Cold}
In cosmology, hot/cold refers to the relativistic mass of an object and thus their relativistic velocity in cosmological scale evolution. Different assumption of the relativistic mass of dark matter has great impact on the cosmological evolution/ galactic formation in cosmology.
As a hot relativistc model of dark matter will lead to cotradiction in galactic formation and universe evolution calcuation. Dark matter is taken to be non-relativistic by most physicists. 

\subsection{Single Particle}
While there are more and more composite dark matter models being proposed these days, dark matter is still taken to be a single kind of particle/ a class of particle with the same physical properties in most effective model/simplified model building. 

\section{Dark Matter Candidate}

There exist a wide range of candidates for what dark matter could be. In here, I outline a few possible candidate of dark matter. , other than candidates include sterile neutrinos, axion dark matter weakly interacting  

\subsection{Sterile Neutrino}
Standard model does not predict that neutrino has a mass, however, contradicting with experimental knowledge. By replacing right-handed neutrinos in the standard model with gauge singlet fermions that has no interaction other than mixing with normal neutrinos, sterile neutrino is formed by theoretical models. Tuning parameters such its interaction rate with normal neutrino, mass and mixing angle, sterile neutrino can be a possible candidate for dark matter, as it can lead to a right
density of dark matter with stability consistent with the scale of the universe. 
However, their measurement proven to be challenging. 


Daya Bay Collaboration, F. P. An et al., “Search for a Light Sterile Neutrino at Daya Bay,” Phys. Rev. Lett. 113 (2014) 141802, arXiv:1407.7259 [hep-ex].

1310.8642
\subsection{Axion Dark Matter}
Another exisitng problem in the standard model is the strong CP problem. The strong CP problem points to the unnaturalness of the exceptionally small experimental measure value of the term that govern strong CP violation. 
The problem can be solved by introducing an additional axion particle and its associated fields to the standard model. With the proposition of such field, the CP violation term in strong interaction will be cancelled out naturally, thereby giving an explanation to the unnaturalness to the experimental value. 
The axion proposed is a dark matter candidate, as its calculated life time is much greater than the age of the universe. 
The axion dark matter is currently not a favored model for dark matter due to its instability, and also the fact that most dedicated searches has returned empty handed. 

TARGET italy 

\subsection{Modified Newtonian Gravity}

\subsection{Weakly Interacting Massive Particle}
A very attractive candidate for dark matter is called the weakly interacting massive particle(WIMP). It appears naturally with many beyond-the-standard-physics model that aim to solve other physics problems, including theories of supersymmetry with R-parity and some extra dimensional theory. 
Another reason that makes WIMP an attractive candidate, is something called the WIMP miracle. Assuming dark matter is a thermal relic, 
https://indico.cern.ch/event/473000/contributions/1993414/attachments/1209863/1764345/tait-Aspen.pdf

\section{Models of Dark Matter in the LHC}

While there exist a wide range of speculation on the identity of dark matter, the use of LHC as a dark matter searching tool constrains a unique set of dark matter candidate accessible phenomenologically. 
Focusing on the theoretical simplicity and what is phenomenologically observable, different modelling approaches are used to develop models searching for dark matter. 

\subsection{Simple Portal Models}
Simple extensions of the standard model, where dark matter is mediated either through the Higgs or the Z Boson. As the Z portal models are contrained by LEP and DD experiments, dark matter that are mediated through a heavier version of Z boson ( the Z') and additional scalar is also searched for. 

\subsection{Effective Field Theory}
While there exist a wide range of dark matter models, the majority of dark matter models can be simplfied by effective field theory where all observables are described by a Lorentz structure and a rate parameter; high energy correction and details are integrated out for "effectiveness". 
This approach allow for few simple models to be searched for in experiment and a structure for a wide range of theretical physics to be studied. 

\subsection{Simplified Model}
While the underlying physics could be much more complex, models neglecting the higher energy physics than is accessible by the collider by focusing on models on particle phenomenology. 
Some simplified dark matter model used in the LHC include the Two-Higgs Doublet Model (2HDM), where Higgs or a non-standard model exotic higgs could serve as a mediator to dark matter; other examples include Dark photon or 
Other than these, simplified model of SUSY the MSSM also predict candidate dark matter like particles called 

\subsection{Less-simplified models}
Since its 

A detail list of models used by the LHC by both the CMS and ATLAS can be found in this reference https://arxiv.org/pdf/1507.00966.pdf


Search for Dark Matter Mod


7. 
Effective field theories
BSM mediator is heavy compared

1. Introduction --Importance of dark matter 
2. Evidence for its existence 
3. What is dark matter (restricted by current evidence) 
4. Models of dark matter
4. Search for dark matter 
5. Searching for dark matter in the LHC 
6. Summary of LHC searches
7. Other theory of Dark matter


%%%%%%%%%%%%%%%%%%%%%%%%%%%%%%%%%%%%%%%%%%%
%%%%%%%%%%%%%%%%%%%%%%%%%%%%%%%%%%%%%%%%%%%
%% sub section describing SM particle content and lagrangian
%%%%%%%%%%%%%%%%%%%%%%%%%%%%%%%%%%%%%%%%%%%
%%%%%%%%%%%%%%%%%%%%%%%%%%%%%%%%%%%%%%%%%%%
%\input{sections/chapter_SM_sub/sm_description}

%%%%%%%%%%%%%%%%%%%%%%%%%%%%%%%%%%%%%%%%%%%
%%%%%%%%%%%%%%%%%%%%%%%%%%%%%%%%%%%%%%%%%%%
%% sub section describing electroweak theory
%%%%%%%%%%%%%%%%%%%%%%%%%%%%%%%%%%%%%%%%%%%
%%%%%%%%%%%%%%%%%%%%%%%%%%%%%%%%%%%%%%%%%%%
%\input{sections/chapter_SM_sub/ewk_description}

%%%%%%%%%%%%%%%%%%%%%%%%%%%%%%%%%%%%%%%%%%%
%%%%%%%%%%%%%%%%%%%%%%%%%%%%%%%%%%%%%%%%%%%
%% sub section describing the higgs mechanism
%%%%%%%%%%%%%%%%%%%%%%%%%%%%%%%%%%%%%%%%%%%
%%%%%%%%%%%%%%%%%%%%%%%%%%%%%%%%%%%%%%%%%%%
%\input{sections/chapter_SM_sub/higgs_description}

%%%%%%%%%%%%%%%%%%%%%%%%%%%%%%%%%%%%%%%%%%%
%%%%%%%%%%%%%%%%%%%%%%%%%%%%%%%%%%%%%%%%%%%
%% sub section describing the standard model finally and its successes
%%%%%%%%%%%%%%%%%%%%%%%%%%%%%%%%%%%%%%%%%%%
%%%%%%%%%%%%%%%%%%%%%%%%%%%%%%%%%%%%%%%%%%%
%\input{sections/chapter_SM_sub/final_sm_description}


%\section{Particles and Forces}
%
%Here we introduce the SM particle content and provide a description of the interactions that
%link the particles together.
%
%\input{tables/tab_sm_content}
%\input{tables/tab_sm_content_EWSB}


%\subsection{Gauge Theories}

%\subsubsection{The Electroweak Theory}


