\chapter{Theory of Dark Matter}

%\epigraph{\textit{So it goes...}}{---Kurt Vonnegut, \textit{Slaughterhouse
%		Five}}
	
%\epigraph{\textit{Science is a miracle.}}{--Ron Swanson}

\epigraph{\textit{If you wish to make an apple pie from scratch, you must first invent the universe.}}{--Carl Sagan, \textit{Cosmos: A Personal Voyage}}
\epigraph{\textit{All models are wrong, but some are useful.}}{--George Box}


[Introduction]

[What is dark matter]

Dark matter is arguably one of the most solid evidence for beyond the standard model physics. In 1933, when Fritz Zwicky first proposed the existence of a "dark matter" holding the galaxies to the cluster center, when luminous matter failed to account for the gravity that keeps the galaxy from esacping.  The proposition was later supported by evidence in difference scales of cosmology and astrophysics. 

Known for interacting gravitationally and at most only weakly, it does not match any standard model particle profile. However, dark matter is known to make up about five times ordinary matter in the universe. Without a proper understanding of dark matter, most of the universe will remain elusive, understanding of universe evolution and galactic formation will also be incomplete.

Experiments and observations has contricted some dark matter properties over the years, but much of the physics is still unknown to us. In the LHC, dark matter can be produced and probed through various methods. If dark matter can be produced in a lab, its understanding could be improved. 

In the following chapter, evidence for the existence of dark matter will first be reviews in 3.1, dark matter and the current properties that restrict its nature will be covered in 3.2 , effective model in the LHC that studied dark matter is discussed in 3.3. An overview of search methods is covered in 3.4 and 3.5. Lastly, the latest LHC searches result and its implication on future of dark matter searches is covered in 3.6. 

1. Introduction --Importance of dark matter 
2. Evidence for its existence 
3. What is dark matter (restricted by current evidence) 
4. Models of dark matter
4. Search for dark matter 
5. Searching for dark matter in the LHC 
6. Summary of LHC searches


$18^{th}$
%%%%%%%%%%%%%%%%%%%%%%%%%%%%%%%%%%%%%%%%%%%
%%%%%%%%%%%%%%%%%%%%%%%%%%%%%%%%%%%%%%%%%%%
%% sub section describing SM particle content and lagrangian
%%%%%%%%%%%%%%%%%%%%%%%%%%%%%%%%%%%%%%%%%%%
%%%%%%%%%%%%%%%%%%%%%%%%%%%%%%%%%%%%%%%%%%%
\input{sections/chapter_SM_sub/sm_description}

%%%%%%%%%%%%%%%%%%%%%%%%%%%%%%%%%%%%%%%%%%%
%%%%%%%%%%%%%%%%%%%%%%%%%%%%%%%%%%%%%%%%%%%
%% sub section describing electroweak theory
%%%%%%%%%%%%%%%%%%%%%%%%%%%%%%%%%%%%%%%%%%%
%%%%%%%%%%%%%%%%%%%%%%%%%%%%%%%%%%%%%%%%%%%
\input{sections/chapter_SM_sub/ewk_description}

%%%%%%%%%%%%%%%%%%%%%%%%%%%%%%%%%%%%%%%%%%%
%%%%%%%%%%%%%%%%%%%%%%%%%%%%%%%%%%%%%%%%%%%
%% sub section describing the higgs mechanism
%%%%%%%%%%%%%%%%%%%%%%%%%%%%%%%%%%%%%%%%%%%
%%%%%%%%%%%%%%%%%%%%%%%%%%%%%%%%%%%%%%%%%%%
\input{sections/chapter_SM_sub/higgs_description}

%%%%%%%%%%%%%%%%%%%%%%%%%%%%%%%%%%%%%%%%%%%
%%%%%%%%%%%%%%%%%%%%%%%%%%%%%%%%%%%%%%%%%%%
%% sub section describing the standard model finally and its successes
%%%%%%%%%%%%%%%%%%%%%%%%%%%%%%%%%%%%%%%%%%%
%%%%%%%%%%%%%%%%%%%%%%%%%%%%%%%%%%%%%%%%%%%
\input{sections/chapter_SM_sub/final_sm_description}


%\section{Particles and Forces}
%
%Here we introduce the SM particle content and provide a description of the interactions that
%link the particles together.
%
%\input{tables/tab_sm_content}
%\input{tables/tab_sm_content_EWSB}


%\subsection{Gauge Theories}

%\subsubsection{The Electroweak Theory}


