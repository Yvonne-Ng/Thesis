\newcommand{\HRule}{\rule{\linewidth}{0.5mm}}
% ----------------------------------
%  Document begin
% ----------------------------------

\newenvironment{myindentpar}[1]%
{\begin{list}{}%
          {\setlength{\leftmargin}{#1}}%
          \item[]%
}
{\end{list}}



%\vspace{0.25in}

% -------------------------
%  SECTION:
%     Personal Details
% -------------------------
%{\Large Personal Details and Contact Information}\\
%\HRule
%\vspace{0.25in}

%\begin{minipage}{0.45\linewidth}
%	\hspace{0.18in}
%  \begin{tabular}{rl}
%    Place and Date of Birth: & \hspace{0.5em} Omaha, Nebraska, United States $\mid$ September 11$^{th}$, 1989 \\
%    Personal Address: & \hspace{0.5em} 1439 Avenue du Jura, Sergy, 01630, France \\
%%    Work Address: & \hspace{0.5em} CERN CH-1211, B\^{a}timent 104/R-C29, 23 Gen\`{e}ve, Switzerland \\
%    Work Address: & \hspace{0.5em} CERN, Esplanade des Particules 1, 1211 Gen\`{e}ve, Switzerland \\ 
%                & \hspace{0.5em} B\^{a}timent 104/R-C29 \\
%%    Permanent Address: & \hspace{0.5em} 3 Aldergrove, Irvine, California, 92604, United States \\
%%    Current Address:   & \hspace{0.5em} 3 Rue de Champ-Fusy, Saint-Genis-Pouilly, 01630, France \\
%    Phone:             & \hspace{0.5em} ($+$33) 06 33 19 30 61 \\
%%    Phone:             & \hspace{0.5em} +1 (949) 294-3045 \\
%    E-mail:            & \hspace{0.5em} daniel.joseph.antrim@cern.ch \\
%                       & \hspace{0.5em} dantrim@uci.edu
%  \end{tabular}
%\end{minipage}

% -------------------------
%  SECTION:
%     Education
% -------------------------
%\vspace{0.35in}
%{\Large Education}\\
%\HRule
%\vspace{0.25in}
%
%\hspace{0.18in}
%\begin{tabular}{c|l}
%     2013--2019 & {\bf{University of California, Irvine}} --- Ph.D. in Physics \\
%%    2013--~~~~~~~ & {\bf{University of California, Irvine}} --- Ph.D. in Physics \\
%%			  & \hspace{0.5cm}Degree Received: Ph.D. in Physics \\
%			  & \hspace{0.5cm}Degree emphasis: Experimental High Energy Particle Physics \\
%              & \hspace{0.5cm}Advisor: Professor Anyes Taffard
%\end{tabular}
%
%\vspace{0.1in}
%\hspace{0.18in}
%\begin{tabular}{c|l}
%    2008--2013 & {\bf{University of California, Los Angeles}} --- B.Sc. in Physics \\
%%		   	  & \hspace{0.5cm}Degree Received: B.Sc. in Physics, with a Minor in Mathematics \\
%              & \hspace{0.5cm}Degree emphasis: Mathematical Physics \\
%			  & \hspace{0.5cm}Distinguishments:\\
%              & \hspace{0.8cm}Departmental honors (Physics and Astronomy) \\
%              & \hspace{0.8cm}Latin honors (Cum laude) \\
%\end{tabular}
%\vspace{0.11in}
%
%\hspace{0.18in}
%\begin{tabular}{c|l}
%    2004--2008 & {\bf{Woodbridge High School}} --- High School Diploma \\
%\end{tabular}

{\Large Education}\\
\HRule
\vspace{0.25in}

\hspace{0.18in}
\begin{tabular}{c|l}
     2013--2019 & {\bf{University of California, Irvine}} --- Ph.D. in Physics \\
%    2013--~~~~~~~ & {\bf{University of California, Irvine}} --- Ph.D. in Physics \\
%			  & \hspace{0.5cm}Degree Received: Ph.D. in Physics \\
			  & \hspace{0.5cm}Degree emphasis: Experimental High Energy Particle Physics \\
              & \hspace{0.5cm}Advisor: Professor Daniel Whiteson
\end{tabular}

\vspace{0.1in}
\hspace{0.18in}
\begin{tabular}{c|l}
    2008--2013 & {\bf{University of Texas, Arlington}} --- B.Sc. in Physics \\
%		   	  & \hspace{0.5cm}Degree Received: B.Sc. in Physics, with a Minor in Mathematics \\
              & \hspace{0.5cm}Degree emphasis: Physics, Mathematics and Philosophy \\
			  & \hspace{0.5cm}Distinguishments:\\
              & \hspace{0.8cm}Departmental honors (Physics and Astronomy) \\
              & \hspace{0.8cm}Latin honors (Cum laude) \\
\end{tabular}
\vspace{0.11in}

\hspace{0.18in}
\begin{tabular}{c|l}
    2004--2009 & {\bf{True Light Middle School of Hong Kong}} --- High School Diploma \\
\end{tabular}

% -------------------------
%  SECTION:
%     Employment
% -------------------------

\vspace{0.35in}
{\Large Employment}\\
\HRule
\vspace{0.25in}
\hspace{0.18in}
\begin{tabular}{c|l}
     2015--2021 & {\bf{University of California, Irvine Dept. of Physics and Astronomy}} \\
%    2014--~~~~~~\,\,& {\bf{University of California, Irvine Department of Physics and Astronomy}} \\
              & Graduate Researcher \\
              & University of California, Irvine (UCI) %, Irvine, CA, USA
\end{tabular}

\vspace{0.1in}
\hspace{0.18in}
\begin{tabular}{c|l}
    2015--2016 & {\bf{University of California, Irvine Dept. of Physics and Astronomy}} \\
              & Teaching Assistant \\
              & University of California, Irvine (UCI) %, Irvine, CA, USA
\end{tabular}


\vspace{0.1in}
\hspace{0.18in}
\begin{tabular}{c|l}
    2012--2015 & {\bf{University of Texas, Arlington Dept. of Physics}} \\
              & Special author of the Dune collaboration\\
              & Assistant to Professor Andrew White, Jaehoon Yu \\
              & G.E.M. Chamber building \\
              & Seasonal dependence and noise studies of the G.E.M \\
              & detector as a beam aligner for the experiment \\
              & Managed and oversaw functioning shifts for 5 other \\ 
              & undergrads in the Gas Electron Multiplier lab 
\end{tabular}

\vspace{0.1in}
\hspace{0.18in}
\begin{tabular}{c|l}
   Summer 2013 & {\bf{Chinese University of Hong Kong}} \\
             & Research Student \\
             & Comsmological simulation of galactic cluster collision at different expansion at different rates of universe
\end{tabular}




% -------------------------
%  SECTION:
%      Research Experience
% -------------------------
\vspace{0.35in}
{\Large Significant Research Experience}\\
\HRule
\vspace{0.15in}


\hspace{0.25in}
As a member of the ATLAS Collaboration, 2015--Present
%\vspace{0.1in}

\hspace{0.45in}
\begin{minipage}{0.8\textwidth}

(Note: Labels to follow in \textbf{\textit{bold italics}} denote official ATLAS designations/roles.)
\vspace{0.15in}

ATLAS Physics Analysis Roles
\vspace{0.1in}

\hspace{0.2in}
\begin{minipage}{\textwidth}
\textbf{\textit{Analysis Contact \& Paper Editor}} Started and led the full Run-2 ATLAS search for dimuon low mass resonances between 12-70 GeV
\end{minipage}

\vspace{0.1in}
\hspace{0.2in}
\begin{minipage}{\textwidth}
\textbf{\textit{Main Analyzer}} Main analyzer for the resolved dijetISR analysis in 2019% (Involvement: May 2015--Present, Contact Role: November 2017--April 2019)
\end{minipage}


\vspace{0.1in}
\hspace{0.2in}
\begin{minipage}{\textwidth}
\textbf{\textit{Significant Contributor}} Significant contributor to the boosted dijetISr analysis of 2019% (Involvement: May 2015--Present, Contact Role: November 2017--April 2019)
\end{minipage}

\vspace{0.15in}
Technical Roles within the ATLAS Collaboration

\vspace{0.15in}
\hspace{0.2in}
\begin{minipage}{\textwidth}
    Built the prototype New Small Wheel online monitoring program for the ATLAS analysis with activities at a cosmic ray stand and final site of installation.
\end{minipage}

\end{minipage}


%Sole developer of the data-acquisition, and electronics configuration \& calibration
%software known as \texttt{VERSO} responsible for test-beam, cosmic-stand, test-bench, and full-chamber integration operation
%of the MicroMegas chambers of the Phase-1 New Small Wheel upgrade for the ATLAS muon spectrometer


%\end{minipage}

% -------------------------
%  SECTION:
%    Publications
% -------------------------
\newpage
\vspace{0.25in}
{\Large Publications}\\
\HRule

\vspace{0.15in}
\hspace{0.25in}\begin{minipage}{0.8\textwidth}

%247 published papers as part of the ATLAS Collaboration (9300+ citations \href{https://inspirehep.net/author/profile/D.J.Antrim.1}{[Inspire]})

131 published papers as part of the ATLAS Collaboration (1230+ citations \href{https://inspirehep.net/authors/1409040}{[Inspire]})

%\hspace{0.35in}\begin{minipage}{1.0\textwidth}
\vspace{0.15in}
(Note: Items that are marked with a star (``$\bigstar$'') denote publications in which I held a leading analyzer, analysis contact, and/or paper-editor role)
%\end{minipage}


\vspace{0.15in}
\hspace{0.2in}Peer-Reviewed Publications from the ATLAS Collaboration

\hspace{0.35in}\begin{minipage}{1.0\textwidth}
%\vspace{0.15in}
%\href{https://arxiv.org/abs/1908.06765}{\textit{Search for the non-resonant Higgs boson pair production in the $bb\ell\nu\ell\nu$ final state with the ATLAS detector in $pp$ collisions at $\sqrt{s} = 13$\,TeV}}, Physics Letters B. (submitted) $\bigstar$

\vspace{0.15in}
%\href{https://arxiv.org/abs/1901.10917}{\textit{Search for low-mass resonances decaying into two jets and produced in association with a photon using p p collisions at $\sqrt{s} = 13$ TeV} with ATLAS detector}, Phys. Lett. B (2019) 56, 2019 $\bigstar$
\href{https://arxiv.org/abs/1807.09477}{\textit{In situ calibration of large-R jet energy and mass in $\sqrt{s} = 13$ TeV with the ATLAS detector}}, Eur. Phys. J. C 78 (2018) 995 $\bigstar$

\vspace{0.15in}
%\href{https://arxiv.org/abs/1803.02762}{\textit{Search for the electroweak production of supersymmetric particles in final states with
%two or three leptons at $\sqrt{s} = 13$ TeV with the ATLAS detector}}, Eur. Phys. J. C 78 (2018) 995
\href{https://arxiv.org/abs/1708.03247}{\textit{Search for light resonances decaying to boosted quark pairs and produced in association with a photon or a jet in proton-proton collisions at $\sqrt{s} = 13$s =13TeV with the ATLAS detector}}, Phys.Lett. B788 (2019) 316 $\bigstar$

\vspace{0.15in}
\href{https://arxiv.org/abs/1610.09392}{\textit{The unexplored landscape of two body resonance }}, PITT-PACC-1610 $\bigstar$

\vspace{0.15in}
\href{https://arxiv.org/abs/1601.05471}{\textit{Long-Baseline Neutrino Facility (LBNF) and Deep Underground Neutrino Experiment (DUNE)}}, Conceptual Design Report Vol.1-2 $\bigstar$




\end{minipage}

\end{minipage}

% -------------------------
%  SECTION:
%    Conferences and talks
% -------------------------
\newpage
%\vspace{0.45in}
%\vspace{0.35in}
{\Large Conferences and Seminar Presentations}\\
\HRule

\vspace{0.15in}

\hspace{0.25in}\begin{minipage}{0.8\textwidth}

    \href{}{\textit{On Background Estimation Techniques}}

\vspace{0.05in}
\hspace{0.2in}
\begin{minipage}{1.0\textwidth}
Virtual due to Covid
ATLAS Higgs And Diboson Searches Workshop(Virtual)
Aug, 2020
\end{minipage}

\vspace{0.1in}
\href{https://www.physics.uci.edu/node/13108}{\textit{Searches for SUSY and Higgs pair production in dilepton events at ATLAS}}
\vspace{0.05in}
\hspace{0.2in}
\begin{minipage}{1.0\textwidth}
University of California, Irvine

Departmental Particle Physics Seminar

January 17$^{th}$, 2018
\end{minipage}

\vspace{0.1in}
\href{https://indico.ihep.ac.cn/event/6387/session/48/contribution/122}{\textit{Development of Trigger and Readout Electronics for the ATLAS New Small Wheel Detector Upgrade}}

\vspace{0.05in}
\hspace{0.2in}
\begin{minipage}{1.0\textwidth}
Beijing, China

Int`l Conference on Technology and Instrumentation in Particle Physics 2017

May 22$^{nd}$, 2017
\end{minipage}

\end{minipage}

% -------------------------
%  SECTION:
%     Honors and Awards
% -------------------------
%\vspace{0.15in}
%\hspace{0.2in}Peer-Reviewed Publications from the ATLAS Collaboration
%
%\hspace{0.35in}\begin{minipage}{1.0\textwidth}
%\vspace{0.15in}
%\href{https://arxiv.org/abs/1803.02762}{\textit{Search for the electroweak production of supersymmetric particles in final states with
%two or three leptons at $\sqrt{s} = 13$ TeV with the ATLAS detector}}, Eur. Phys. J. C 78 (2018) 995
%
%\vspace{0.15in}
%\textbf{*} \href{https://arxiv.org/abs/1708.03247}{\textit{Search for direct top squark pair production in final states with two leptons
%in $\sqrt{s} = 13$ TeV $pp$ collisions with the ATLAS detector}}, Eur. Phys. J. C77 (2017) 898

%\newpage
\vspace{0.2in}
{\Large Honors and Awards}\\
\HRule

%\vspace{0.15in}
\hspace{0.25in}
\begin{minipage}{0.8\textwidth}

    \vspace{0.15in}
    University of California, Irvine (Graduate)
    
	\vspace{0.05in}
    \hspace{0.15in}
    \begin{minipage}{1.0\textwidth}
        \vspace{0.1in}
        US-ATLAS Outstanding Graduate Student Achievement Award, 2019

        \vspace{0.1in}
        US-LHC Users Organization Lightning Round Talk Award, 2018
    \end{minipage}
\end{minipage}

%\vspace{0.25in}
%\begin{tabular}{ll}
%    & US-ATLAS Outstanding Graduate Student Achievement Award, 2019
%\end{tabular}

		


%\vspace{0.25in}
%\begin{tabular}{ll}
%    & Official CERN Tour Guide \\
%    & \hspace{0.7em} Lead tours through many of the CERN facilities, as well as underground to the \\
%    & \hspace{0.7em} ATLAS experimental cavern, to engage and educate the public
%\end{tabular}


% -------------------------
%  SECTION:
% 		Technical Skills
% -------------------------
\vspace{0.35in}
{\Large Technical Skills} \\
\HRule
\vspace{0.15in}

\hspace{0.2in}
General and Data-Analysis Oriented
\vspace{0.1in}

\hspace{0.2in}
\begin{minipage}{0.8\textwidth}

    \vspace{0.1in}
    \hspace{0.15in}
    \begin{minipage}{\textwidth}
    Programming Languages: C/C++ (low- and high-level), Python, bash, Qt (graphical user-interface development)
    \end{minipage}


    \vspace{0.1in}
    \hspace{0.15in}
    \begin{minipage}{\textwidth}
    Data formats: ROOT, HDF5
    \end{minipage}

    \vspace{0.1in}
	\hspace{0.15in}
	\begin{minipage}{\textwidth}
	Machine learning (skilled in the use of the \textsc{Keras} and \textsc{Tensorflow} libraries)
    \end{minipage}

	\vspace{0.1in}
	\hspace{0.15in}
    \begin{minipage}{\textwidth}
	Statistics and Frequentist limit setting procedures for ATLAS analyses
    \end{minipage}

	\vspace{0.1in}
	\hspace{0.15in}
    \begin{minipage}{\textwidth}
	Analysis framework development and maintenance (skilled in and familiar with the ATLAS analysis software and xAOD--based event data model)
    \end{minipage}

    \vspace{0.1in}
    \hspace{0.15in}
    \begin{minipage}{\textwidth}
    \end{minipage}

\end{minipage}

\hspace{0.2in}
Technical/Hardware Oriented

\vspace{0.1in}
\hspace{0.2in}
\begin{minipage}{0.8\textwidth}


    \vspace{0.1in}
    \hspace{0.15in}
    \begin{minipage}{\textwidth}
    Trigger/DAQ software, as related to High Energy Particle Physics experiments (within the ATLAS environment and standalone)
    \end{minipage}

    \vspace{0.1in}
    \hspace{0.15in}
    \begin{minipage}{\textwidth}
    Network-based computing (experience in communication over both the UDP and TCP protocols)
    \end{minipage}

    \vspace{0.1in}
    \hspace{0.15in}
    \begin{minipage}{\textwidth}
    Asynchronous and multi-threaded programming principles (experience primarily within C++)
    \end{minipage}

    \vspace{0.1in}
    \hspace{0.15in}
    \begin{minipage}{\textwidth}
    Readout electronics and detector instrumentation, as related to particle detectors
    \end{minipage}

    \vspace{0.1in}
    \hspace{0.15in}
    \begin{minipage}{\textwidth}
    Familiar hardware protocols: HDLC, SPI, I2C
    \end{minipage}



\end{minipage}

% -------------------------
%  SECTION:
%     Advising
% -------------------------
\vspace{0.35in}
{\Large Advising}\\
\HRule

\hspace{0.25in}
\begin{minipage}{0.8\textwidth}
	\vspace{0.15in}
	University of California, Irvine

	\hspace{0.15in}
	\begin{minipage}{1.0\textwidth}
		\vspace{0.1in}
		Alexander Armstrong (Graduate)

		\vspace{0.1in}
		Syed Faizanul Haque (Undergraduate)
	\end{minipage}
\end{minipage}

%\newpage
% -------------------------
%  SECTION:
%       Outreach
% -------------------------
%\vspace{0.1in}
{\Large Outreach and Education} \\
\HRule

%\vspace{0.15in}
\hspace{0.25in}
\begin{minipage}{0.8\textwidth}
	\vspace{0.15in}
	Official CERN Tour Guide

	\hspace{0.15in}
	\begin{minipage}{1.0\textwidth}
		\vspace{0.15in}

		Leading tours through the CERN facilities, including underground to the ATLAS experimental cavern, to engage with and educate the public
	\end{minipage}
\end{minipage}

% -------------------------
%  SECTION:
%     Teaching Experience
% -------------------------
%\newpage
\vspace{0.2in}
%\vspace{0.25in}
{\Large Teaching Experience}\\
\HRule
%\vspace{0.25in}

\hspace{0.25in}
\begin{minipage}{0.8\textwidth}
	\vspace{0.15in}
	Teaching Assistant (University of California, Irvine)

	\hspace{0.15in}
	\begin{minipage}{1.0\textwidth}
		\vspace{0.1in}
		Physics 3LC (2 academic quarters): Laboratory on optics, radioactivity, \& atomic physics

		\vspace{0.1in}
		Physics 7D (1 academic quarter): Electromagnetism for Physicists \& Engineers
	\end{minipage}
\end{minipage}
