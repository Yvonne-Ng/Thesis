That ``we don't accomplish anything in this world alone'' is never more true than in the
case of attempting a doctorate in physics, especially in the sub-field of experimental
high energy particle physics that, nowadays, likely has you stretched across the globe
in your work on large-scale experiments operated by international collaborations.

I have had the opportunity to be based at CERN for essentially the entirety of my doctorate.
This would not have been possible without the many people who have helped me along the way.
Their number is many, too far to count or list here.
They know who they are, though, and without them I would not have succeeded.
Not professionally and most definitely not personally.

As a result, the importance of my friends and family has never felt stronger.
Without them none of this means a thing and I am surely indebted to them forever.
They have given me the opportunity to climb mountains (figuratively and literally), not feel alone in the world,
and to establish a true home in France and Switzerland.
The pain and suffering that a young researcher sometimes feels, whether in pushing out a physics analysis or in confronting the
rather unscientific bureaucracies inherent in a large collaboration like ATLAS, are far outweighed by their measure.
The biggest takeaways from my time at CERN are therefore not technical at all and boil down to
a few of the ``strongest of all warriors'' that are requisite in our dealings and goings-on about
the world: patience and compassion.
I have been damn lucky to have been surrounded by exemplars of these two during my doctorate.
I hope to follow along in their footsteps as we move forward.

Also thanks to Dan.
