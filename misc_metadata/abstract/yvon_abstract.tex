In search of weird resonances in the LHC

Many empirical evidence points to the existence of beyond-the-standard-model physics. In the LHC, different theoretical and experiemntal tehcniques are used to look for new particles as for new physics. One method with theoretical elegance and clean experimental reconstruction, is the resonance finding method. The method has enjoyed much sucesses in the past, W, Z and the Higgs boson are some of the few particles discovered this way. However, the early beyond-the-standard-model resonance search program of the LHC has returned empty handed. 
In this thesis, I ask the question of whether the standard approach could be modified to look for "weird resonances", particles that exist in the LHC dataset but goes beyond the standard program. Their discovery can provide important clue to beyond-the-standard-model physics. 

Three physics analyses beyond the usual search program are covered in this thesis. The first two on the dijetISR final state and the last one on dimuon final state. 

In the dijetISR searches, an ISR object is used to boost the final state. By triggering on the ISR object rather than the resonance formming final state, lower mass resonances can be covered. These two analysis set the lowest ATLAS low mass limit below the standard dijet search.

In the dimuon search, as previous searches has neglected the low mass region below the Z peak, the search seeks to cover the region from 12-70 GeV. A novel statistical method driven by Gaussian Process is used. The method allow for a simple background estimation for a similar class of future analyses where Monte Carlo is limited.

All of these analyses provide competitive exclusion in the dark matter benchmark model of the LHC in places not previously covered.

As a future discussion, a new stream from trigger level analysis could be implemented to the dimuon search to improve the low mass sensitivity. Data driven approach like CWOLA can also be used to look for resonances free of an assumed signal model.

The thesis also covers involvement jet in-situ calibration as well as the new small wheel online monitoring software development.

"Weird resonances" are the "out-of-ordinary" ones: They are not a part of the current knowledge model; they break existing laws; they defies our knowledge about the world and ourselves; they are already somewhere out there, but only from its discovery it can be realized that we are more than what we think we are. The search for weird resonanances is therefore a search for authenticity.
